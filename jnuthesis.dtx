% \iffalse meta-comment
%
% Copyright (C) 2013-2016, Haixing Hu.
% Department of Computer Science and Technology, Nanjing University.
%
% This file may be distributed and/or modified under the conditions of the
% LaTeX Project Public License, either version 1.2 of this license or (at your
% option) any later version. The latest version of this license is in:
%
% http://www.latex-project.org/lppl.txt
%
% and version 1.2 or later is part of all distributions of LaTeX version
% 1999/12/01 or later.
%
% Home Page of the Project: http://haixing-hu.github.io/nju-thesis/
%
% \fi
%
% \iffalse
%<*driver>
\ProvidesFile{jnuthesis.dtx}
%</driver>
%<cls>\NeedsTeXFormat{LaTeX2e}[1995/12/01]
%<cls>\ProvidesClass{jnuthesis.cls}
%<cfg>\ProvidesFile{jnuthesis.cfg}
%<*cls>
 [2016/02/24 v1.1.19 Document Class for the Degree Papers of Nanjing University]
%</cls>
%<*driver>
\documentclass[10pt,a4paper,oneside]{ltxdoc}
\usepackage{dtx-style}
\EnableCrossrefs
\CodelineIndex
\GetFileInfo{jnuthesis.dtx}
\begin{document}
  \DocInput{jnuthesis.dtx}
\end{document}
%</driver>
% \fi
%
% \CheckSum{0}
% \CharacterTable
%  {Upper-case    \A\B\C\D\E\F\G\H\I\J\K\L\M\N\O\P\Q\R\S\T\U\V\W\X\Y\Z
%   Lower-case    \a\b\c\d\e\f\g\h\i\j\k\l\m\n\o\p\q\r\s\t\u\v\w\x\y\z
%   Digits        \0\1\2\3\4\5\6\7\8\9
%   Exclamation   \!     Double quote  \"     Hash (number) \#
%   Dollar        \$     Percent       \%     Ampersand     \&
%   Acute accent  \'     Left paren    \(     Right paren   \)
%   Asterisk      \*     Plus          \+     Comma         \,
%   Minus         \-     Point         \.     Solidus       \/
%   Colon         \:     Semicolon     \;     Less than     \<
%   Equals        \=     Greater than  \>     Question mark \?
%   Commercial at \@     Left bracket  \[     Backslash     \\
%   Right bracket \]     Circumflex    \^     Underscore    \_
%   Grave accent  \`     Left brace    \{     Vertical bar  \|
%   Right brace   \}     Tilde         \~}
%
% \DoNotIndex{\begin,\end,\begingroup,\endgroup}
% \DoNotIndex{\ifx,\ifdim,\ifnum,\ifcase,\else,\or,\fi}
% \DoNotIndex{\let,\def,\xdef,\newcommand,\renewcommand}
% \DoNotIndex{\expandafter,\csname,\endcsname,\relax,\protect}
% \DoNotIndex{\Huge,\huge,\LARGE,\Large,\large,\normalsize}
% \DoNotIndex{\small,\footnotesize,\scriptsize,\tiny}
% \DoNotIndex{\normalfont,\bfseries,\slshape,\interlinepenalty}
% \DoNotIndex{\hfil,\par,\vskip,\vspace,\quad}
% \DoNotIndex{\centering,\raggedright}
% \DoNotIndex{\c@secnumdepth,\@startsection,\@setfontsize}
% \DoNotIndex{\@plus,\@minus,\p@,\z@,\@m,\@M,\@ne,\m@ne,\@@par,\@dottedtocline}
% \DoNotIndex{\ ,\,,\.,\\}
% \DoNotIndex{\|}
% \DoNotIndex{\@dottedtocline}
% \DoNotIndex{\@afterindenttrue,\@arabic,\@biblabel,\@clubpenalty}
% \DoNotIndex{\@empty,\@highpenalty,\@ifnextchar,\@latex@warning,\@listI,\@listi}
% \DoNotIndex{\@mainmatterfalse,\@mainmattertrue,\@mkboth,\@nobreakfalse}
% \DoNotIndex{\@nobreaktrue,\@noitemerr,\@openbib@code,\@pnumwidth,\@restonecolfalse}
% \DoNotIndex{\@restonecoltrue,\@starttoc,\@tempcnta,\@tempdima,\@tocrmarg}
% \DoNotIndex{\@afterindenttrue,\@arabic,\@biblabel,\@clubpenalty,\@dottedtocline}
% \DoNotIndex{\@empty,\@highpenalty,\@ifnextchar,\@latex@warning,\@listI,\@listi}
% \DoNotIndex{\@mainmatterfalse,\@mainmattertrue,\@mkboth,\@nobreakfalse,\@nobreaktrue}
% \DoNotIndex{\@noitemerr,\@openbib@code,\@pnumwidth,\@restonecolfalse}
% \DoNotIndex{\@restonecoltrue,\@starttoc,\@tempcnta,\@tempdima,\@tocrmarg}
% \DoNotIndex{\abovedisplayshortskip,\abovedisplayskip,\addpenalty,\addvspace}
% \DoNotIndex{\advance,\alph,\arabic,\arraybackslash,\arraystretch,\AtBeginDocument}
% \DoNotIndex{\belowdisplayshortskip,\belowdisplayskip,\bf,\blacksquare,\bottomfraction}
% \DoNotIndex{\bullet,\c@enumiv,\c@page,\c@tocdepth,\captionsetup,\cdot,\CJKfamily}
% \DoNotIndex{\CJKglue,\CJKnumber,\CJKunderline,\CJKunderlinecolor,\ClassError}
% \DoNotIndex{\clearpage,\CurrentOption,\dagger,\day,\ddagger}
% \DoNotIndex{\DeclareGraphicsExtensions,\DeclareMathSizes,\DeclareOption}
% \DoNotIndex{\DeclareRobustCommand,\defaultfontfeatures,\DefineFNsymbolsTM}
% \DoNotIndex{\Diamondblack,\endlist,\ensuremath,\efill,\equal,\everypar}
% \DoNotIndex{\fontsize,\global,\hb@xt@,\hbox,\hfill,\hline,\hskip,\hspace,\hss}
% \DoNotIndex{\if@mainmatter,\if@restonecol,\if@twocolumn,\if@twoside,\ifodd}
% \DoNotIndex{\ifthenelse,\ignorespaces,\includegraphics,\input,\it,\item}
% \DoNotIndex{\itemsep,\kern,\l@chapter,\l@part,\labelsep,\labelwidth,\leaders}
% \DoNotIndex{\leavevmode,\leftmargin,\leftmargini,\leftmark,\leftskip,\list}
% \DoNotIndex{\LoadClass,\m@th,\makebox,\MakeUppercase,\markboth,\markright}
% \DoNotIndex{\mathparagraph,\mathsection,\mkern,\month,\multicolumn}
% \DoNotIndex{\newcolumntype,\newenvironment,\newif,\newline,\newlist,\newpage}
% \DoNotIndex{\newtheorem,\nobreak,\normalcolor,\null,\number,\onecolumn,\p@enumiv}
% \DoNotIndex{\pagestyle,\parbox,\parfillskip,\parindent,\parsep,\PassOptionsToClass}
% \DoNotIndex{\pdfbookmark,\penalty,\ProcessOptions,\punctstyle,\raisebox,\renewenvironment}
% \DoNotIndex{\RequirePackage,\RequireXeTeX,\restylefloat,\rightmargin,\rightmark,\rightskip}
% \DoNotIndex{\roman,\rule,\selectfont,,\setcounter,\setfnsymbol}
% \DoNotIndex{\setlength,\setlist,\settowidth,\sfcode,\sloppy,\square,\stretch,\tabcolsep}
% \DoNotIndex{\textasteriskcentered,\textbardbl,\textbf,\textdagger,\textdaggerdbl}
% \DoNotIndex{\textfraction,\textnormal,\textparagraph,\textsection,\textwidth}
% \DoNotIndex{\theenumiv,\theoremseparator,\theoremstyle,\theoremsymbol,\thispagestyle}
% \DoNotIndex{\titleformat,\titlespacing,\topsep,\twocolumn,\ULthickness}
% \DoNotIndex{\usecounter,\widowpenalty,\year,\color,\clubpenalty,\chaptermark}
% \DoNotIndex{\chaptertitlename,\geometry,\l@chapter,\l@part}
%
% \MakeShortVerb{\|}
% \newcommand*{\jnuthesis}{\texttt{JNU-Thesis}}
% \newcommand*{\texlive}{{\TeX}\ Live\ 2012}
%
% \pagestyle{empty}
% \title{\jnuthesis:江南大学学位论文\\
%        {\XeLaTeX}模板}
% \author{{胡海星}\\
%         {\texttt{starfish.hu@gmail.com}}\\
%         {江南大学计算机科学与技术系}}
% \date{\fileversion\ (\filedate)}
% \maketitle
% \thispagestyle{empty}
%
% \begin{abstract}
%
% \noindent\hspace{2em}文档类{\jnuthesis}提供了一个江南大学学位论文的{\XeLaTeX}
% 模板。该文档类严格按照江南大学对学位论文的格式要求排版学位论文,底层通过
% |xeCJK|宏包支持中文。
%
% 目前{\jnuthesis}可用于排版学士学位、硕士学位论文和博士学位论文。对于学士学位论文,
% 采用和硕士学位论文一样的格式进行排版。
%
% 虽然{\jnuthesis}主要用于排版江南大学的学位论文,但亦可经过简单的设置或修改用于排版
% 国内其他大学的学位论文。
%
% 本文档是{\jnuthesis}的说明文档,其中包含模板文件的设置说明以及其源代码的完全注释。
% \end{abstract}
% \clearpage
% \newpage
% \section*{{\hfill}修订历史{\hfill}}
% \begin{center}
% \noindent
% \begin{longtable}[C]{C{1.1cm}
%                      C{1.8cm}
%                      C{1.5cm}
%                      p{\textwidth-6.2cm}}
% \toprule
%   \textbf{版本}
%   & \textbf{日期}
%   & \textbf{修订者}
%   & \textbf{修订内容} \\
% \midrule
%  v1.0.0 & 2013/08/24 & 胡海星 & 完成第一个可工作版本 \\
%  v1.0.1 & 2013/08/25 & 胡海星 & 增加了一些PDF文档的元信息\\
%  v1.0.2 & 2013/08/25 & 胡海星 & 合并了国家图书馆封面上的导师信息\newline
%                                使用\cs{makebox}\oarg{width}\oarg{s}\marg{text}
%                                使得封面标签可以自动两端对其分布\newline
%                                修改了学位论文和用户手册的样式\\
%  v1.0.3 & 2013/08/25 & 胡海星 & 增加了《学位论文出版授权书》页面\\
%  v1.0.4 & 2013/08/25 & 胡海星 & 按照江南大学《博士 (硕士) 学位论文编写格式
%                                规定 (试行)》的要求修改了学位论文的一些排版格式\newline
%                                修复了页眉页脚的问题\\
%  v1.0.5 & 2013/08/25 & 胡海星 & 按照dtx的写作规范修改了手册的排版\newline
%                                增加了|preface|环境\newline
%                                删除了脚注样式的设置\newline
%                                删除了|algorithm|环境的定义,因为很多情况下
%                                其他算法和代码排版宏包都会定义该环境\\
%  v1.0.6 & 2013/08/25 & 胡海星 & 修复了生成学士学位论文的bug\newline
%                                增加了对国家标准的参考\newline
%                                增加了\std{GB/T 7156-2003}标准所定义的文献
%                                密级\newline
%                                将论文密级设置命令\cs{confidential}更名为
%                                \cs{securitylevel}\newline
%                                修改了国家图书馆封面和英文封面的样式\newline
%                                将生成国家图书馆封面的命令独立了出来\\
%  v1.0.7 & 2013/08/26 & 胡海星 & 修改了国家图书馆封面和论文出版授权书的一些细节\newline
%                                完善了|jnuthesis|手册\\
%  v1.0.8 & 2013/08/28 & 胡海星 & 修复了拼写错误和文件重命名导致的bug\\
%  v1.1.0 & 2013/08/31 & 胡海星 & 重写了符合\std{GB/T 7714-2005}规范
%                                的{\BibTeX}样式文件\\
%  v1.1.1 & 2013/09/01 & 胡海星 & 增加了警告图标\newline
%                                修改了参考文献引用的样式\newline
%                                去除了\cs{paragraph}及以下级别章节的编号\newline
%                                修改了章节编号的样式\newline
%                                修改了一些本手册的排版细节\\
%  v1.1.2 & 2013/09/02 & 胡海星 & 重新定义\cs{ref}命令,
%                                使其前面自动加一个``\textasciitilde''\\
%  v1.1.3 & 2013/09/02 & 胡海星 & 修复了公式自动编号多了一个括号的bug\\
%  v1.1.4 & 2013/09/02 & 胡海星 & 修复了附录的节编号错误\newline
%                                按照\std{CY/T 35-2001}规范修改了图、表、公式、
%                                定理的编号样式\\
%  v1.1.5 & 2013/09/03 & 胡海星 & 修改了参考文献引用编号和脚注编号的样式,使其符合
%                                \std{GB/T 7714-2005}规范\newline
%                                去除了用处不大的批注功能\\
%  v1.1.6 & 2013/09/15 & 胡海星 & 修改了项目主页网址 \\
%  v1.1.7 & 2013/10/18 & 胡海星 & 根据相应的国家规范,将``目录''改为``目次'',
%                                将``插图索引''改为``插图清单'',
%                                将``表格索引''改为``附表清单''\\
%  v1.1.8 & 2014/12/22 & 胡海星 & 修改|tabular|环境中的行间距为正文行间距\\
%  v1.1.9 & 2015/01/28 & 胡海星 & 改用|longtable|作为``修订历史''表格环境\\
%  v1.1.10 & 2015/01/28 & 胡海星 & 修复Tex Live 2014引入的|xeCJK|使用|CJKnumber|选项后
%                                 找不到|CJKnumber|命令的bug\\
%  v1.1.11 & 2015/04/14 & 胡海星 & 增加|notation|环境\\
%  v1.1.12 & 2015/06/14 & 胡海星 & 增加情况分类列表|caseenum|环境和步骤列表|stepenum|环境\\
%  v1.1.13 & 2015/06/14 & 胡海星 & 增加公设环境|postulate| \\
%  v1.1.14 & 2015/06/14 & 胡海星 & 恢复了|algorithm|环境的定义 \\
%  v1.1.15 & 2015/06/16 & 胡海星 & 为|url|宏包添加|hyphens|选项 \\
%  v1.1.16 & 2015/06/22 & 胡海星 & 增加假说环境|hypothesis| \\
%  v1.1.17 & 2015/06/27 & 胡海星 & 修复了段落首行缩进不是恰好两个中文字符的bug,
%                                 修正了|proof|环境最后的QED符号,
%                                 修改了|solution|环境的实现方式 \\
%  v1.1.18 & 2015/07/16 & 胡海星 & 将公设环境|postulate|和定律环境单独编号\\
%  v1.1.19 & 2016/02/24 & 胡海星 & 提供了论文标题很长时的解决方案;
%                                 修复了新版|xeCJKfntef|中|\CJKunderline|的颜色
%                                 设置方式不再兼容|CJKfntef|所引起的问题\\
% \bottomrule
% \end{longtable}
% \end{center}
% \clearpage
%
% \tableofcontents
% \clearpage
%
% \pagestyle{fancy}
% \section{简介}
%
% 文档类{\jnuthesis}是为了帮助江南大学的同学撰写学位论文而编写的{\XeLaTeX}模板。
% 该模板提供了一个江南大学学位论文的{\XeLaTeX}文档类,用于生成符合江南大学学位
% 论文格式要求进行的学位论文。该宏包的底层通过|xeCJK|宏包支持中文。目前该宏包可
% 用于排版硕士学位论文和博士学位论文;对于学士学位论文,由于作者未能找到江南大学
% 对学士学位论文的格式要求,因此目前采用和硕士学位论文一样的格式进行排版。
%
% 虽然{\jnuthesis}主要用于排版江南大学的学位论文,但亦可经过简单的设置或修改用于排版
% 国内其他大学的学位论文。
%
% 本文档将尽量完整的介绍{\jnuthesis}的使用方法,如有不清楚之处可以参考示例文档或
% 者与作者联系。由于作者水平有限,虽然现在的这个版本基本上满足了学位论文的撰写需
% 求,但难免还存在不足之处,欢迎大家积极反馈意见。
%
% 本模板的编写过程中参考了以下代码和文档,这里一并向这些代码和文档的作者表示感谢:
%
% \begin{itemize}
% \item 杨文博. \textsl{江南大学学位论文{\LaTeX}模板}. \url{https://code.google.com/p/jnuthesis/}.
% \item 薛瑞尼. \textsl{清华大学学位论文{\LaTeX}模板}.
% \item 胡卫谊. \textsl{武汉理工大学学位论文{\LaTeX}模板}.
% \item 吴凯. \textsl{GBT7714-2005NLang.bst}. v1.0 beta 2. 2006.
% \item \textsl{CTeX宏包}. \url{http://www.ctex.org}.
% \item The {\LaTeX}3 Project. \textsl{{\LaTeXe} for class and package writers}.
% \item Frank Mittelbach, Michel Gooseens. \textsl{The {\LaTeX} Companion}. 2nd ed.
% \item Scott Pakin. \textsl{How to Package Your {\LaTeX} Package}.
% \url{http://www.iitg.ernet.in/trivedi/LatexHelp/Latex%20Manual/dtxtut.pdf}.
% \item Oren Patashnik. \textsl{Designing \BibTeX Styles}. 1988.
% \end{itemize}
%
% \section{遵循的要求和标准}
%
%% {\jnuthesis}所遵循的江南大学研究生院的要求如下:
%% \begin{itemize}
%% \item \textsl{江南大学硕士答辩流程及相关材料下载},\\
%% \url{http://gs.nju.edu.cn/content/xw/ss3.htm}
%% \item \textsl{江南大学博士答辩流程及相关材料下载},\\
%% \url{http://gs.nju.edu.cn/content/xw/bs3.htm}
%% \item \textsl{江南大学博士(硕士)学位论文编写格式规定(试行)}\\
%% \url{http://grawww.nju.edu.cn/content/xw/lwgf.doc}
%% \end{itemize}
%
% {\jnuthesis}所遵循的中华人民共和国国家标准如下:
% \begin{itemize}
% \item \std{GB/T 7714-2005}\textsl{文后参考文献著录规则}
% \item \std{GB/T 7713.1-2006}\textsl{学位论文编写规则}
% \item \std{GB/T 7713.3-2009}\textsl{科技报告编写规则}
% \item \std{GB/T 7713-1987}\textsl{科学技术报告、学位论文和学术论文的编写格式},
% 该标准已被\std{GB/T 7713.1-2006}和\std{GB/T 7713.3-2009}部分替代
% \item \std{GB/T 7156-2003}\textsl{文献保密等级代码与标识}
% \item \std{GB/T 16159-2012}\textsl{汉语拼音正词法基本规则},
% 该标准取代了\std{GB/T 16159-1996}
% \item \std{CY/T 35-2001}\textsl{科技文献的章节编号方法}
% \end{itemize}
%
% 上述要求和标准的电子版可在{\jnuthesis}项目主页的源码库中找到。
%
% \begin{note}
% 如江南大学研究生院对学位论文的格式要求和国家标准之间有冲突,以国家标准的要求为准。
% \end{note}
%
% \section{安装}
%
% \subsection{下载}
%
% 可在{\jnuthesis}项目主页上下载最新版本,亦可在代码库主页上反馈bug和意见建议:
% \begin{itemize}
% \item 项目主页:\url{http://haixing-hu.github.io/nju-thesis/}
% \item 代码库主页:\url{https://github.com/Haixing-Hu/nju-thesis}
% \end{itemize}
%
% \subsection{模板的组成部分}
%
% \begin{table}
%   \centering\noindent
%   \begin{tabular*}{\textwidth}{p{4cm}p{\textwidth-4.5cm}}
%     \toprule
%     \textbf{文件(夹)}        & \textbf{功能描述}\\
%     \midrule
%     |jnuthesis.ins|             & 模板驱动文件 \\
%     |jnuthesis.dtx|             & 模板文档代码的混合文件\\
%     |jnuthesis.cls|             & 模板类文件\\
%     |jnuthesis.cfg|             & 模板配置文件\\
%     |gbt7714-2005.bst|          & 符合国标\std{GB/T 7714-2005}的参考文献样式文件\\
%     |dtx-style.sty|             & 用户手册样式文件\\
%     |jnulogo.eps|               & 江南大学校徽图片\\
%     |jnuname.eps|               & 江南大学校名图片\\
%     \hline
%     |sample.tex|                & 示例文档,亦可作为学位论文的基本模板 \\
%     |sample.bib|                & 示例文档的参考文献数据库 \\
%     |figures/|                  & 示例文档图片目录\\
%     \hline
%     |Makefile|                  & make 脚本 \\
%     |get_texmf_dir.sh|          & 获取本地|textmf|目录路径的脚本\\
%     \hline
%     |README.md|                 & 说明文件 \\
%     |jnuthesis.pdf|             & 用户手册(本文档)\\
%     \bottomrule
%   \end{tabular*}
%   \caption{{\jnuthesis}的主要文件及其功能}\label{table:component}
% \end{table}
%
% 表\ref{table:component}列出了{\jnuthesis}的主要文件及其功能。其中|jnuthesis.cls|,
% |jnuthesis.cfg|和|dtx-sty.sty|可以由|jnuthesis.ins|和|jnuthesis.dtx|生成,但为
% 了降低新手用户的使用难度,故将其一起发布。
%
% \subsection{准备工作}
%
% 表\ref{table:dependence}列出了{\jnuthesis}模板用到的宏包。这些包在常见的{\TeX}系
% 统中都有(推荐使用{\texlive}),如果没有请到\url{www.ctan.org}下载。
%
% \begin{table}
%   \centering\noindent
%   \begin{tabular*}{\textwidth}{@{\extracolsep{\fill}}*{6}{l}}
%   \hline
%     |ifxetex|  & |indentfirst| & |xeCJK|    & |lastpage| & |geometry|  & |graphicx| \\
%     |subfig|   & |caption|     & |float|    & |array|    & |longtable| & |booktabs| \\
%     |multirow| & |hyperref|    & |enumitem| & |xcolor|   & |amsmath|   & |amsfonts| \\
%     |amsthm|   & |amssymb|    & |bm|        & |mathrsfs| & |txfonts|  & |pifont|  \\
%     |setspace| & |wasysym|    & |hypernat| & |fancyhdr| & |natbib|    & |tabularx| \\
%     |titlesec| & |glossaries|  & |ifthen|   & |makeidx| & |footmisc| &  |CJKnumb| \\
%     |url|      & |etoolbox| &  &  &  & \\
%   \hline
%   \end{tabular*}
%   \caption{{\jnuthesis}用到的宏包}\label{table:dependence}
% \end{table}
%
% \subsection{推荐的{\TeX}系统}
%
% 本模板当前版本v{\fileversion}{\ }({\filedate})在{\texlive}下编写,尚未在其他
% {\TeX}系统上测试。因此推荐用户使用{\texlive}。其安装包可以在下述网址下载:
% \begin{center}
% \url{http://tug.org/texlive/}
% \end{center}
%
% \begin{note}
% 由于本模板采用{\XeLaTeX}引擎处理,所以{\TeX}源文件应使用\textbf{UTF-8}编码。
% \end{note}
%
% \subsection{开始安装}
%
% \subsubsection{生成模板}
%
% 默认的发行包中已经包含了所有文件,可以直接使用。如果对如何由|*.dtx|生成模板文件以及模板文
% 档不感兴趣,请跳过本小节。
%
% 模板解压缩后生成文件夹|jnuthesis-VERSION|,其中|VERSION|为版本号。该文件夹中包括:
% \begin{itemize}
% \item 模板源文件:|jnuthesis.ins|和|jnuthesis.dtx|
% \item 参考文献样式:|gbt7714-2005.bst|
% \item 江南大学校徽和校名图片:|jnulogo.eps|和|jnuname.eps|
% \item 示例文档:|sample.tex|、|sample.bib|和|figure|目录
% \end{itemize}
%
% 在使用之前需要先生成模板文件和配置文件,具体命令细节请参考|README|和|Makefile|。下面是
% 在Linux或Mac系统中生成模板所需执行的|shell|命令:
%
% \begin{shell}
% $ cd jnuthesis-VERSION
% # 清理以前执行make生成的旧文件
% $ make clean
% # 生成 jnuthesis.cls 和 jnuthesis.cfg
% $ make cls
% # 生成文档类手册
% $ make doc
% # 生成样例文档
% $ make sample
% \end{shell}
%
% \subsubsection{安装到{\TeX}系统中}
%
% 假设当前{\TeX}系统的texmf-local目录为|${TEXMFLOCAL}|。下面是在Linux或Mac系统中将模
% 板安装到本机的{\TeX}系统中所需执行的|shell|命令:
%
% \begin{shell}
% $ cd jnuthesis-VERSION
% # 建立jnuthesis文档类目录
% $ mkdir -p ${TEXMFLOCAL}/tex/latex/jnuthesis
% # 复制jnuthesis文档类文件
% $ cp jnuthesis.cls ${TEXMFLOCAL}/tex/latex/jnuthesis/
% $ cp jnuthesis.cfg ${TEXMFLOCAL}/tex/latex/jnuthesis/
% $ cp jnulogo.eps  ${TEXMFLOCAL}/tex/latex/jnuthesis/
% $ cp jnuname.eps  ${TEXMFLOCAL}/tex/latex/jnuthesis/
% # 复制jnuthesis文档类的源码,此过程可选
% $ cp jnuthesis.ins ${TEXMFLOCAL}/tex/latex/jnuthesis/
% $ cp jnuthesis.dtx ${TEXMFLOCAL}/tex/latex/jnuthesis/
% # 创建本地的BibTeX样式文件目录
% $ mkdir -p ${TEXMFLOCAL}/bibtex/bst
% # 复制GB/T 7714-2005参考文献样式
% $ cp gbt7714-2005.bst ${TEXMFLOCAL}/bibtex/bst/
% # 建立jnuthesis文档类手册目录
% $ mkdir -p ${TETEXMFLOCALXMF}/doc/latex/jnuthesis
% # 复制jnuthesis文档类手册和示例文档
% $ cp jnuthesis.pdf ${TEXMFLOCAL}/doc/latex/jnuthesis/
% $ cp sample.pdf ${TEXMFLOCAL}/doc/latex/jnuthesis/
% # 刷新tex文件名数据库
% $ texhash
% \end{shell}
%
% \begin{note}
% 上面的某些命令可能需要管理员权限,或通过|sudo|执行。
% \end{note}
%
% 当然,也可以直接使用|Makefile|提供的|install|操作进行安装:
% \begin{shell}
% $ sudo make install
% \end{shell}
%
% \begin{note}
% |Makefile|使用了脚本|get_texmf_dir.sh|来获取当前机器上所安装的{\TeX}系统的本地
% |textmf|目录(通常是{\TeX}安装目录下的|textmf-local|目录)。用户最好在执行
% |make install|之前先执行一下|get_texmf_dir.sh|脚本,看看输出的目录路径是否正确。
% 如不正确,可以手工修改|Makefile|中对|TEXMFLOCAL|变量的定义。
% \end{note}
%
% \section{使用说明}
%
% 本手册假定用户已经能处理一般的{\LaTeX}文档,并对{\BibTeX}有一定了解。如果你从来没有接
% 触过{\TeX}和{\LaTeX},建议先学习相关的基础知识。
%
% \subsection{\jnuthesis{} 示例文件}
%
% 模板核心文件只有三个:|jnuthesis.cls|,|jnuthesis.cfg|和 |gbt7714-2005.bst|,但
% 是如果没有示例文档用户会发现很难下手。所以推荐新用户从模板自带的示例文档入手,
% 里面包括了文档写作用到的所有命令及其使用方法,只需要用自己的内容进行相应替换就
% 可以。对于不清楚的命令可以查阅本手册。具体内容可以参考模板附带的|sample.tex|和
% |sample.bib|。
%
% \subsection{选项}
%
% 本文档类提供了一些选项以方便使用:
% \begin{description}
% \item[winfonts, linuxfonts, macfonts, adobefonts] |winfonts|选项使得文档使
%   用Windows系统提供的字体;|linuxfonts|选项使得文档使用Linux系统提供的字
%   体;|macfonts|选项使得文档使用Mac系统提供的字体;|adobefonts|选项使得文档使
%   用Adobe提供的OTF中文字体,一般来说OTF字体的显示效果要优于ttf字体。
%   默认选项是|adobefonts|。
% \begin{example}
% \documentclass[winfonts]{jnuthesis}
% \end{example}
%   表\ref{table:fontnames}中列出了默认配置下使用不同字体选项时所采用的实际字体
%   名称。系统中必须正确安装了相应的字体才能正常编译文档。\\
%   Adobe的宋体和黑体可以在其公司网站免费下载:
%   \begin{center}
%   \url{http://www.adobe.com/support/downloads/detail.jsp?ftpID=4421}
%   \end{center}
%   楷体无免费下载,但在网上可以找到。下面的网址提供了一个打包下载的地址:
%   \begin{center}
%   \url{http://tinker-bot.googlecode.com/files/cfonts.tar.gz}
%   \end{center}
%   \begin{table}
%     \centering\noindent
%     \begin{tabular}[t]{ccccc}
%     \toprule
%           & \textbf{adobefonts} &  \textbf{winfonts} & \textbf{linuxfonts} & \textbf{macfonts} \\
%     \midrule
%     \textbf{宋体} & {Adobe Song Std}  & {SimSun} & {AR PL SungtiL GB} &  {STSong} \\
%     \textbf{黑体} & {Adobe Heiti Std} & {SimHei} & {WenQuanYi Zen Hei Mono} &  {STHeiti} \\
%     \textbf{楷书} & {Adobe Kaiti Std} & {KaiTi}  & {AR PL KaitiM GB} & {STKaiti} \\
%     \textbf{仿宋体} & {Adobe Fangsong Std} & {FangSong} & {STFangSong} & {STFangSong} \\
%     \bottomrule
%     \end{tabular}
%     \caption{默认配置下不同字体选项所使用的实际字体名称}
%     \label{table:fontnames}
%   \end{table}
%
% \item[nobackinfo] 该选项用于控制是否在封面背面打印导师签名信息。如果设置了此选
%   项,则不在封面背面打印导师签名信息。此选项默认不被设置,一般情况下也无需设置
%   该选项。
% \begin{example}
% \documentclass[winfonts,nobackinfo]{jnuthesis}
% \end{example}
%
% \item[phd, master, bachelor] 用于设置申请的学位级别。当选择|phd|时,生成江南大学博
% 士学位论文,包含国家图书馆格式的封面,但不包括书脊,书脊需要单独制作;选择|master|时,
% 生成江南大学硕士学位论文;选择|bachelor|时,生成江南大学学士学位论文。
% \begin{example}
% \documentclass[winfonts,phd]{jnuthesis}
% \end{example}
% \begin{note}
% 这三个选项必须设置一个且只能设置一个。
% \end{note}
%
% \end{description}
%
% 本文档类不再提供对字号、字体和单双面打印的选择选项。因为国内各大学的学位论文基本上都要求
% 使用小四号宋体,双面打印。
%
% \subsection{命令和环境}
%
% 文档类中的命令和环境分为三类:一是格式控制,二是内容替换,三是文档结构。格式控制如字体、字
% 号、字距和行距等;内容替换如文档名称、作者、项目、编号等;文档结构如中文摘要、中文关键词、
% 英文摘要、英文关键词、作者简历、致谢等。
%
% \subsubsection{格式控制命令}
%
% \myentry{中文字体}
% \DescribeMacro{\songti}
% \DescribeMacro{\heiti}
% \DescribeMacro{\kaishu}
% \DescribeMacro{\fangsong}
% 可采用下述命令选择中文字体
% \begin{itemize}
% \item \cs{songti} 切换宋体
% \item \cs{heiti} 切换黑体
% \item \cs{kaishu} 切换楷书
% \item \cs{fangsong} 切换仿宋体
% \end{itemize}
%
% \begin{example}
% {\songti 乾:元,亨,利贞}
% {\heiti 九二,见龙在田,利见大人}
% {\kaishu 九三,君子终日乾乾,夕惕若,厉,无咎}
% {\fangsong 九四,或跃在渊,无咎}
% \end{example}
%
% \myentry{字号}
% \DescribeMacro{\zihao}
% \cs{zihao}命令可用于选择字号。其语法为:
% \begin{syntax}
% \cs{zihao}\marg{n}
% \end{syntax}
% 其中参数\meta{n}为要使用的字号;在\meta{n}前加负号$-$表示小号字体。目前提供的字号包括:
% \begin{itemize}
% \item 初号(|\zihao{0}|)、小初号(|\zihao{-0}|)
% \item 一号(|\zihao{1}|)、小一号(|\zihao{-1}|)
% \item 二号(|\zihao{2}|)、小二号(|\zihao{-2}|)
% \item 三号(|\zihao{3}|)、小三号(|\zihao{-3}|)
% \item 四号(|\zihao{4}|)、小四号(|\zihao{-4}|)
% \item 五号(|\zihao{5}|)、小五号(|\zihao{-5}|)
% \item 六号(|\zihao{6}|)、小六号(|\zihao{-6}|)
% \item 七号(|\zihao{7}|)
% \item 八号(|\zihao{8}|)
% \end{itemize}
%
% \begin{example}
% {\zihao{2} 二号} {\zihao{3} 三号} {\zihao{4} 四号} {\zihao{-4} 小四}
% \end{example}
%
% \myentry{字距}
% \DescribeMacro{\ziju}
% \cs{ziju}命令可用于更改汉字之间默认的距离。其语法为:
% \begin{syntax}
% \cs{ziju}\marg{width}
% \end{syntax}
% 其中的参数\meta{width}只要是合格的{\TeX}距离即可。
%
% \begin{example}
% {\ziju{4bp}调整字距示例}
% \end{example}
%
% \myentry{缩进}
% \DescribeMacro{\indent}
% \DescribeMacro{\noindent}
% \cs{indent}命令将当前行正常的缩进两个汉字字宽的距离,同时在汉字大小和字距改变的情况都
% 可以自动修改缩进距离。
%
% \cs{noindent}则取消缩进。
%
% \myentry{破折号}
% \DescribeMacro{\zhdash}
% 中文破折号在CJK-{\LaTeX}里没有很好的处理,我们平时输入的都是两个小短线,比如这样,
% ``{中国——中华人民共和国}''。这不符合中文习惯。所以这里定义了一个命令生成更好看的破折号。
% 不过这似乎不是一个好的解决办法,比如不能用在\cs{section}等命令中使用。简单的办法是可以
% 提供一个不带破折号的段标题:
% \begin{syntax}
% \cs{section}\oarg{没有破折号精简标题}\marg{带破折号的标题}
% \end{syntax}
%
% \begin{example}
% 测试--中文破折号
% 测试{\zhdash}中文破折号
% \end{example}
%
% 上述例子的显示效果分别如下:
% \begin{itemize}
% \item 测试--中文破折号
% \item 测试{\zhdash}中文破折号
% \end{itemize}
%
% \subsubsection{国家图书馆封面内容替换命令}
%
% 本节描述论文的国家图书馆封面的内容替换命令。只有博士学位论文才需要提供国家图书
% 馆封面。若申请的学位为硕士或学士,则可完全忽略本节所描述的命令。
%
% \myentry{分类号}
% \DescribeMacro{\classification}
% 命令\cs{classification}用于设置论文按照《中国图书资料分类法》的分类编号。此属性
% 必须被设置。具体的分类号需咨询学校图书馆的老师。
%
% \begin{example}
%   \classification{O175.2}
% \end{example}
%
% \myentry{密级}
% \DescribeMacro{\securitylevel}
% \DescribeMacro{\openlevel}
% \DescribeMacro{\controllevel}
% \DescribeMacro{\confidentiallevel}
% \DescribeMacro{\clasifiedlevel}
% \DescribeMacro{\mostconfidentiallevel}
% 命令\cs{securitylevel}设置论文的密级。论文的密级必须按照\std{GB/T 7156-2003}标准
% 进行填写。
%
% 根据\std{GB/T 7156-2003}标准,文献保密等级分为$5$级,即“公开级”、“限制级”、
% “秘密级”、“机密级”、“绝密级”。本文档类中预定义了该标准中文献保密等级的五个等级的
% 代码常量:
% \begin{enumerate}
% \item \cs{openlevel},表示公开级:此级别的文献可在国内外发行和交换。
% \item \cs{controllevel},表示限制级:此级别的文献内容不涉及国家秘密,但在一定时间内
% 限制其交流和使用范围。
% \item \cs{confidentiallevel},表示秘密级:此级别的文献内容涉及一般国家秘密。
% \item \cs{clasifiedlevel},表示机密级:此级别的文献内容涉及重要的国家秘密 。
% \item \cs{mostconfidentiallevel},表示绝密级:此级别的文献内容涉及最重要的国家秘密。
% \end{enumerate}
%
% 如果未设置\cs{securitylevel},其默认值将被设置为\cs{openlevel},即“公开级”。
%
% \begin{example}
% \securitylevel{\controllevel}
% \end{example}
%
% \myentry{UDC编号}
% \DescribeMacro{\udc}
% 命令\cs{udc}用于设置论文按照《国际十进分类法UDC》的分类编号。此属性可选,默认值为空白。
%
% 国际十进分类法(Universal Decimal Classification,简称UDC),又称为通用十进制
% 分类法,是世界上规模最大、用户最多、影响最广泛的一部文献资料分类法。自
% 1899--1905年比利时学者奥特勒和拉封丹共同主编、出版UDC法文第一版以来,现已有20
% 多种语言的各种详略版本。近百年来,UDC已被世界上几十个国家的10多万个图书馆和情
% 报机构采用。UDC目前已成为名符其实的国际通用文献分类法。
%
% 论文的具体UDC编号需咨询学校图书馆的老师,或在下面网址查询:
% \begin{center}
% \url{http://www.udcc.org/udcsummary/php/index.php?lang=chi}
% \end{center}
%
% \begin{example}
% \udc{004.72}
% \end{example}
%
% \myentry{论文标题及副标题}
% \DescribeMacro{\nlctitlea}
% \DescribeMacro{\nlctitleb}
% \DescribeMacro{\nlctitlec}
% 命令\cs{nlctitlea}、\cs{nlctitleb}和\cs{nlctitlec}分别用于设置国家图书馆封面的
% 论文标题及副标题的第一行、第二行和第三行。其中,\cs{nlctitlea}为可选,默认值为
% 用户通过\cs{title}命令设置的中文标题;\cs{nlctitleb}和\cs{nlctitlec}亦为可选,
% 其默认值为空白。这三个命令是为了让用户在论文标题较长时手动进行分割换行。
%
% \begin{example}
% \nlctitlea{基于小世界理论的}
% \nlctitleb{数据中心网络模型研究}
% \end{example}
%
% \begin{note}
% \cs{nlctitlea}、\cs{nlctitleb}和\cs{nlctitlec}命令的参数中都不能再出现换行。
% \end{note}
%
% \myentry{导师信息}
% \DescribeMacro{\supervisorinfo}
% 命令\cs{supervisorinfo}用于设置论文作者的导师的单位名称及联系地址。此属性必须被设置。
%
% \begin{example}
% \supervisorinfo{江南大学计算机科学与技术系,南京市汉口路22号,210093}
% \end{example}
%
% \myentry{答辩委员会主席}
% \DescribeMacro{\chairman}
% 命令\cs{chairman}用于设置论文答辩委员会主席的姓名和职称。此属性必须被设置。
%
% \begin{example}
% \chairman{王重阳\hspace{1em}教授}
% \end{example}
%
% \myentry{评阅人}
% \DescribeMacro{\reviewera}
% \DescribeMacro{\reviewerb}
% \DescribeMacro{\reviewerc}
% \DescribeMacro{\reviewerd}
% 命令\cs{reviewera}、\cs{reviewerb}、\cs{reviewerc}、\cs{reviewerd}分别用于设置
% 论文的第一、第二、第三和第四评阅人的姓名和职称。这四个命令为可选,默认值为空白。
%
% \begin{example}
% \reviewera{张三丰~~教授}
% \reviewerb{张无忌~~副教授}
% \reviewerc{黄裳~~教授}
% \reviewerd{郭靖~~研究员}
% \end{example}
%
% \subsubsection{中文封面内容替换命令}
%
% 本节描述论文中文封面的内容替换命令。
%
% \myentry{论文标题}
% \DescribeMacro{\title}
% 命令\cs{title}用于设置论文的中文标题。
%
% \begin{example}
% \title{基于小世界理论的数据中心网络模型}
% \end{example}
%
% \begin{note}
% \cs{title}的参数中不可换行,也不能使用\cs{thanks}脚注。
% \end{note}
%
% \myentry{论文标题分行}
% \DescribeMacro{\titlea}
% \DescribeMacro{\titleb}
% 命令\cs{titlea}和\cs{titleb}用于在论文标题很长时,设置分行的论文中文标题。
% 其中\cs{titlea}设置论文标题的第一行,\cs{titleb}设置论文标题的第二行。
%
% \begin{example}
% \titlea{半轻衰变$D^+\to \omega(\phi)e^+\nu_e$的研究}
% \titleb{和弱衰变$J/\psi \to D_s^{(*)-}e^+\nu_e$的寻找}
% \end{example}
%
% \begin{note}
% \cs{title}的参数中不可换行,也不能使用\cs{thanks}脚注。
% \end{note}
%
% \myentry{作者姓名}
% \DescribeMacro{\author}
% 命令\cs{author}用于设置论文作者的姓名。此属性必须被设置。
%
% \begin{example}
% \author{张三}
% \end{example}
%
% \begin{note}
% \cs{author}的参数中不可换行,也不能使用\cs{thanks}脚注。
% \end{note}
%
% \myentry{作者电话}
% \DescribeMacro{\telphone}
% 命令\cs{telphone}用于设置论文作者的电话号码。此属性必须被设置。
%
% \begin{example}
% \telphone{13671413272}
% \end{example}
%
% \myentry{作者邮件}
% \DescribeMacro{\email}
% 命令\cs{email}用于设置论文作者的电子邮件地址。此属性必须被设置。
%
% \begin{example}
% \email{san.zhang@gmail.com}
% \end{example}
%
% \myentry{作者学号}
% \DescribeMacro{\studentnum}
% 命令\cs{studentnum}用于设置论文作者的学号。此属性必须被设置。
%
% \begin{example}
% \studentnum{MGXXXXXXX}
% \end{example}
%
% \myentry{入学年份}
% \DescribeMacro{\grade}
% 命令\cs{grade}用于设置论文作者的入学年份(即年级),用一个阿拉伯数字表示。此属性
% 必须被设置。
%
% \begin{example}
% \grade{2012}
% \end{example}
%
% \myentry{导师姓名职称}
% \DescribeMacro{\supervisor}
% 命令\cs{supervisor}用于设置论文作者的导师的姓名和职称。此属性必须被设置。
%
% \begin{example}
% \supervisorname{李四~~教授}
% \end{example}
%
% \myentry{导师电话}
% \DescribeMacro{\supervisortelphone}
% 命令\cs{supervisortelphone}用于设置论文作者的导师的姓名和职称。此属性必须被设置。
%
% \begin{example}
% \supervisortelphone{13671607471}
% \end{example}
%
% \myentry{学科专业}
% \DescribeMacro{\major}
% 命令\cs{major}用于设置论文作者的学科与专业方向。此属性必须被设置。
%
% \begin{example}
% \major{计算机软件与理论}
% \end{example}
%
% \begin{note}
% \cs{major}的参数中不可换行。
% \end{note}
%
% \myentry{研究方向}
% \DescribeMacro{\researchfield}
% 命令\cs{researchfield}用于设置论文作者的研究方向。此属性必须被设置。
%
% \begin{example}
% \major{计算机网络与信息安全}
% \end{example}
%
% \begin{note}
% \cs{researchfield}的参数中不可换行。
% \end{note}
%
% \myentry{院系名称}
% \DescribeMacro{\department}
% 命令\cs{department}用于设置论文作者所在院系的中文名称。此属性必须被设置。
%
% \begin{example}
% \department{计算机科学与技术系}
% \end{example}
%
% \begin{note}
% \cs{department}的参数中不可换行。
% \end{note}
%
% \myentry{学校名称}
% \DescribeMacro{\institute}
% 命令\cs{institute}用于设置论文作者所在学校或机构的名称,该学校或机构也是所申请学
% 位的颁发机构。此命令为可选,默认值为``江南大学''。
%
% \begin{example}
% \institute{江南大学}
% \end{example}
%
% \begin{note}
% \cs{institute}的参数中不可换行。
% \end{note}
%
% \myentry{提交日期}
% \DescribeMacro{\submitdate}
% 命令\cs{submitdate}用于设置论文的提交日期,需设置年、月、日。此属性必须被设置。
%
% \begin{example}
% \submitdate{2013年6月10日}
% \end{example}
%
% \myentry{答辩日期}
% \DescribeMacro{\defenddate}
% 命令\cs{defenddate}用于设置论文的答辩日期,需设置年、月、日。此属性必须被设置。
%
% \begin{example}
% \defenddate{2013年6月27日}
% \end{example}
%
% \myentry{定稿日期}
% \DescribeMacro{\date}
% 命令\cs{date}用于设置论文的定稿日期,该日期将出现在中文封面下方以及书脊下方。需设
% 置年、月、日。此属性可选,默认值为最后一次编译时的日期,精确到日。
%
% \begin{example}
% \date{2013年5月27日}
% \end{example}
%
% \subsubsection{英文封面内容替换命令}
%
% 本节描述论文的英文封面的内容替换命令。
%
% \myentry{论文标题}
% \DescribeMacro{\englishtitle}
% 命令\cs{englishtitle}用于设置论文的英文标题。此属性必须被设置。
%
% \begin{example}
% \englishtitle{Network Models of Data Centers based on the Small World Theory}
% \end{example}
%
% \begin{note}
% \cs{englishtitle}的参数中不可换行,也不能使用\cs{thanks}脚注。
% \end{note}
%
% \myentry{作者姓名}
% \DescribeMacro{\englishauthor}
% 命令\cs{englishauthor}用于设置论文的作者姓名的汉语拼音,此属性必须被设置。
% \begin{suggestion}
% 作者姓名的汉语拼音必须遵循\std{GB/T 16159-2012}标准。
% \end{suggestion}
%
% \begin{example}
% \englishauthor{Wei Xiaobao}
% \end{example}
%
% \begin{note}
% \cs{englishauthor}的参数中不可换行,也不能使用\cs{thanks}脚注。
% \end{note}
%
% \myentry{导师姓名职称}
% \DescribeMacro{\englishsupervisor}
% 命令\cs{englishsupervisor}用于设置论文作者的导师姓名的汉语拼音和导师职称的英文翻译。
% 此属性必须被设置。
% \begin{suggestion}
% 导师姓名的汉语拼音必须遵循\std{GB/T 16159-2012}标准。
% \end{suggestion}
%
% \begin{example}
% \englishsupervisor{Professor CHEN Jin-Nan}
% \end{example}
%
% \myentry{作者专业}
% \DescribeMacro{\englishmajor}
% 命令\cs{englishmajor}用于设置论文作者的学科与专业方向的英文名。此属性必须被设置。
%
% \begin{example}
% \englishmajor{Compuer Software and Theory}
% \end{example}
%
% \begin{note}
% \cs{englishmajor}的参数中不可换行。
% \end{note}
%
% \myentry{院系名称}
% \DescribeMacro{\englishdepartment}
% 命令\cs{englishdepartment}用于设置论文作者所在院系的英文名称。此属性必须被设置。
%
% \begin{example}
% \englishdepartment{Department of Computer Science and Technology}
% \end{example}
%
% \begin{note}
% \cs{englishdepartment}的参数中不可换行。
% \end{note}
%
% \myentry{学校名称}
% \DescribeMacro{\englishinstitute}
% 命令\cs{englishinstitute}用于设置论文作者所在学校或机构的英文名称,此学校或机构
% 也是所申请学位的颁发机构。此属性可选,默认值为``Nanjing University''。
%
% \begin{example}
% \englishinstitute{Nanjing University}
% \end{example}
%
% \begin{note}
% \cs{englishinstitute}的参数中不可换行。
% \end{note}
%
% \myentry{完成日期}
% \DescribeMacro{\englishdate}
% 命令\cs{englishdate}用于设置论文完成日期的英文形式,它将出现在英文封面下方。需
% 设置年、月、日。日期格式使用美国的日期格式,即``Month day, year'',其中
% ``Month''为月份的英文名全称,首字母大写;``day''为该月中日期的阿拉伯数字表示;
% ``year''为年份的四位阿拉伯数字表示。此属性可选,默认值为最后一次编译时的日期。
%
% \begin{example}
% \englishdate{May 1, 2013}
% \end{example}
%
% \subsubsection{中文摘要页内容替换命令}
%
% 本节描述论文的中文摘要页的内容替换命令。
%
% \myentry{标题及副标题}
% \DescribeMacro{\abstracttitlea}
% \DescribeMacro{\abstracttitleb}
% 命令\cs{abstracttitlea}和\cs{abstracttitleb}分别用于设置中文摘要页面的论文标题
% 及副标题的第一行和第二行。\cs{abstracttitlea}命令为可选,其默认值为使用\cs{title}
% 命令所设置的论文标题;\cs{abstracttitleb}命令为可选,其默认值为空白。这两个命令
% 是为了让用户在论文标题较长时手动进行分割换行。
%
% \begin{example}
% \abstracttitlea{基于小世界理论的}
% \abstracttitleb{数据中心网络模型研究}
% \end{example}
%
% \begin{note}
% \cs{abstracttitlea}和\cs{abstracttitleb}命令的参数中都不能出现换行。
% \end{note}
%
% \subsubsection{英文摘要页内容替换命令}
%
% 本节描述论文的英文摘要页的内容替换命令。
%
% \myentry{标题及副标题}
% \DescribeMacro{\englishabstracttitlea}
% \DescribeMacro{\englishabstracttitleb}
% 命令\cs{abstracttitlea}和\cs{abstracttitleb}分别用于设置英文摘要页面的论文标题
% 及副标题的第一行和第二行。\cs{englishabstracttitlea}命令为可选,其默认值为使用
% \cs{englishtitle}命令所设置的论文英文标题;\cs{englishabstracttitleb}命令为可
% 选,其默认值为空白。这两个命令是为了让用户在论文标题较长时手动进行分割换行。
%
% \begin{example}
% \englishabstracttitlea{A Network Model of Data Centers}
% \englishabstracttitleb{Based on the Small World Theory}
% \end{example}
%
% \begin{note}
% \cs{englishabstracttitlea}和\cs{englishabstracttitleb}命令的参数中都不能换行。
% \end{note}
%
% \subsubsection{文档结构命令和环境}
%
% 本节描述论文中可能用到的其他文档结构命令和环境。
%
% \myentry{生成国家图书馆封面}
% \DescribeMacro{\makenlctitle}
% 命令\cs{makenlctitle}用于生成论文的国家图书馆封面。此命令必须被用在{\TeX}文档
% 的\cs{begin{document}}命令之后和\cs{frontmatter}命令之前。目前只有博士学位论文要
% 求制作国家图书馆封面,硕士学位论文和学士学位论文不需要。
%
% \begin{example}
% \makenlctitle
% \end{example}
%
% \myentry{生成中文封面}
% \DescribeMacro{\maketitle}
% 命令\cs{maketitle}用于生成论文的中文封面。此命令必须被用在{\TeX}文档的
% \cs{begin{document}}命令之后和\cs{frontmatter}命令之前。
%
% \begin{example}
% \maketitle
% \end{example}
%
% \myentry{生成英文封面}
% \DescribeMacro{\makeenglishtitle}
% 命令\cs{makeenglishtitle}用于生成论文的英文封面。此命令必须被用在{\TeX}文档的
% \cs{begin{document}}命令之后和\cs{frontmatter}命令之前。
%
% \begin{example}
% \makeenglishtitle
% \end{example}
%
% \myentry{中文摘要}
% \DescribeEnv{abstract}
% \env{abstract}为中文摘要环境。此环境必须被用在{\TeX}文档的\cs{frontmatter}命令之后和
% \cs{mainmatter}命令之前。
%
% \begin{example}
% \begin{abstract}
% 本文基于小世界理论,研究了数据中心的网络模型。………………
% \end{abstract}
% \end{example}
%
% \myentry{中文关键词}
% \DescribeMacro{\keywords}
% 命令\cs{keywords}用于设置中文关键词。此命令必须被用在\env{abstract}环境中。关键词
% 之间用中文全角分号隔开。
%
% \begin{example}
% \begin{abstract}
% 本文基于小世界理论,研究了数据中心的网络模型。………………
% \keywords{数据中心;网络模型;小世界理论}
% \end{abstract}
% \end{example}
%
% \myentry{英文摘要}
% \DescribeEnv{englishabstract}
% \env{englishabstract}为英文摘要环境。此环境必须被用在{\TeX}文档的
% \env{abstract}环境之后和\cs{mainmatter}命令之前。
%
% \begin{example}
% \begin{englishabstract}
% In this paper, we studied the network model of data centers,
% based on the theory of small worlds. ....
% \end{englishabstract}
% \end{example}
%
% \myentry{英文关键词}
% \DescribeMacro{\englishkeywords}
% 命令\cs{englishkeywords}用于设置英文关键词。此命令必须被用在\env{englishabstract}
% 环境中。关键词之间用英文半角逗号隔开。
%
% \begin{example}
% \begin{englishabstract}
% In this paper, we studied the network model of data centers,
% based on the theory of small worlds. ....
% \englishkeywords{Data Center, Network Model, Small World}
% \end{englishabstract}
% \end{example}
%
% \myentry{前言}
% \DescribeEnv{preface}
% \env{preface}为论文前言环境。此环境必须被用在{\TeX}文档的
% \env{englishabstract}环境之后和\cs{tableofcontents}命令之前。
%
% \begin{example}
% \begin{preface}
%  复杂网络的研究可上溯到20世纪60年代对ER网络的研究。90年后代随着Internet
%  的发展,以及对人类社会、通信网络、生物网络、社交网络等各领域研究的深入,
%  发现了小世界网络和无尺度现象等普适现象与方法。对复杂网络的定性定量的科
%  学理解和分析,已成为如今网络时代科学研究的一个重点课题。
%
%  在此背景下,由于云计算时代的到来,本文针对面向云计算的数据中心网络基础
%  设施设计中的若干问题,进行了几方面的研究。本文的创造性研究成果主要如下
%  几方面:
%
%  ………
%
%
%  \vspace{1cm}
%  \begin{flushright}
%   韦小宝\\
%   2013年夏于江南大学南苑
%  \end{flushright}
% \end{preface}
% \end{example}
%
% \myentry{目录}
% \DescribeMacro{\tableofcontents}
% 命令\cs{tableofcontents}用于生成论文目录。此命令必须被用在{\TeX}文档的
% \env{preface}环境之后和\cs{mainmatter}命令之前。
%
% \begin{example}
% \tableofcontents
% \end{example}
%
% \myentry{附表清单}
% \DescribeMacro{\listoftables}
% 命令\cs{listoftables}用于生成论文的附表清单。此命令为可选命令。此命令必须被用在
% {\TeX}文档的\cs{tableofcontents}命令之后和\cs{mainmatter}命令之前。
%
% \begin{example}
% \listoftables
% \end{example}
%
% \myentry{插图清单}
% \DescribeMacro{\listoffigures}
% 命令\cs{listoffigures}用于生成论文插图清单。此命令为可选命令。此命令必须被用在
% {\TeX}文档的\cs{tableofcontents}命令之后和\cs{mainmatter}命令之前。
%
% \begin{example}
% \listoffigures
% \end{example}
%
% \myentry{致谢章节}
% \DescribeEnv{acknowledgement}
% \env{acknowledgement}环境用于生成致谢章节。此环境必须被用在论文的最后一章(通
% 常是“结论”章节)之后以及{\TeX}文档的\cs{appendix}命令和\cs{backmatter}命令之前。
%
% \begin{example}
% \begin{acknowledgement}
% 首先感谢我的母亲韦春花对我的支持。其次感谢我的导师陈近南对我的精心指导和热心帮助。接
% 下来,感谢我的师兄茅十八和风际中,他们阅读了我的论文草稿并提出了很有价值的修改建议。
%
% 最后,感谢我亲爱的老婆们:双儿、苏荃、阿珂、沐剑屏、曾柔、建宁公主、方怡,感谢你们在
% 生活上对我无微不至的关怀和照顾。
% \end{acknowledgement}
% \end{example}
%
% \myentry{简历与科研成果}
% \DescribeEnv{resume}
% \DescribeEnv{authorinfo}
% \DescribeEnv{education}
% \DescribeEnv{publications}
% \DescribeEnv{projects}
% \env{resume}环境用于生成致谢章节。此环境必须被放在{\TeX}文档的\cs{backmatter}
% 命令之后。\env{authorinfo}环境用于生成论文作者简介;\env{education}环境用于生
% 成论文作者教育经历列表;\env{publications}环境用于生成论文作者在攻读学位期间发
% 表的论文的列表;\env{projects}环境用于生成论文作者在攻读学位期间参与的科研课题
% 的列表。
%
% \begin{example}
% \begin{resume}
% % 论文作者身份简介,一句话即可。
% \begin{authorinfo}
% \noindent 韦小宝,男,汉族,1985年11月出生,江苏省扬州人。
% \end{authorinfo}
% % 论文作者教育经历列表,按日期从近到远排列,不包括将要申请的学位。
% \begin{education}
% \item[2007.9 --- 2010.6] 江南大学计算机科学与技术系 \hfill 硕士
% \item[2003.9 --- 2007.6] 江南大学计算机科学与技术系 \hfill 本科
% \end{education}
% % 论文作者在攻读学位期间所发表的文章的列表,按发表日期从近到远排列。
% \begin{publications}
% \item Xiaobao Wei, Jinnan Chen, ``Voting-on-Grid Clustering for Secure
%   Localization in Wireless Sensor Networks,'' in \textsl{Proc. IEEE
%   International Conference on Communications (ICC) 2010}, May. 2010.
% \item Xiaobao Wei, Shiba Mao, Jinnan Chen, ``Protecting Source Location
%   Privacy in Wireless Sensor Networks with Data Aggregation,'' in
%   \textsl{Proc. 6th International Conference on Ubiquitous Intelligence
%   and Computing (UIC) 2009}, Oct. 2009.
% \end{publications}
% % 论文作者在攻读学位期间参与的科研课题的列表,按照日期从近到远排列。
% \begin{projects}
% \item 国家自然科学基金面上项目``无线传感器网络在知识获取过程中的若干安全问题研究''
% (课题年限~2010.1 --- 2012.12),负责位置相关安全问题的研究。
% \item 江苏省知识创新工程重要方向项目下属课题``下一代移动通信安全机制研究''
% (课题年限~2010.1 --- 2010.12),负责LTE/SAE认证相关的安全问题研究。
% \end{projects}
% \end{resume}
% \end{example}
%
% \myentry{生成《学位论文出版授权书》}
% \DescribeMacro{\makelicense}
% 命令\cs{makelicense}用于生成《学位论文出版授权书》。该授权书中的一些字段将根据
% 用户所设置的文档属性自动填写,其他字段需由作者将论文打印出来后用笔手工填写。此命令应该
% 用于{\TeX}文档的\cs{end{document}}命令之前。
% \begin{example}
% \makelicense
% \end{example}
%
% \subsubsection{其它命令和环境}
%
% \myentry{列表环境}
% \DescribeEnv{itemize}
% \DescribeEnv{enumerate}
% \DescribeEnv{description}
% 为了适合中文习惯,{\jnuthesis}文档类使用|paralist|宏包重新定义了|itemize|、
% |enumerate|和|description|这三个常用的列表环境。一方面满足了多余空间的清楚,另
% 一方面可以自己指定标签的样式和符号。
%
% 使用的细节请参看|paralist|文档,此处不再赘述。
%
% \subsection{数学环境}
%
% {\jnuthesis}宏包预定义了一些数学定理环境,如表\ref{table:math-env}所示。
%
% \begin{table}
% \noindent\centering
% \begin{tabular}{*{7}{l}}
%   \hline
%   axiom     & theorem   & definition & proposition & lemma      & conjecture & notation \\
%   公理       & 定理      & 定义        & 命题         & 引理       & 猜想        & 记号 \\
%   \hline
%   proof     & corollary & example    & exercise    & assumption & remark     & problem \\
%   证明       & 推论      & 例子        & 练习         & 假设        & 评注       & 问题\\
%   \hline
%   postulate &  hypothesis  &  principle  &  algorithm &           &      &  \\
%   公设       &  假说        &   定律       &  算法       &         &        & \\
%   \hline
% \end{tabular}
% \caption{预定义的数学定理环境}\label{table:math-env}
% \end{table}
%
% 例如:
% \begin{example}
% \begin{definition}
% 小世界网络是指其平均路径长度和其节点总数成对数关系的网络。
% \end{definition}
% \end{example}
% 上述代码将产生(自动编号):
% \begin{flushleft}
% {\heiti 定义~1.1~~~} {小世界网络是指其平均路径长度和其节点总数成对数关系的网络。}
% \end{flushleft}
%
% 列举出来的数学环境毕竟是有限的,如果想用{\heiti 胡说}这样的数学环境,那么很容易定义:
% \begin{example}
% \newtheorem{nonsense}{胡说}[chapter]
% \end{example}
%
% 然后这样使用:
% \begin{example}
% \begin{nonsense}
% 契丹武士要来中原夺武林秘笈。\zhdash 慕容博
% \end{nonsense}
% \end{example}
% 上述代码将产生(自动编号):
% \begin{flushleft}
% {\heiti 胡说~1.1~~~} {契丹武士要来中原夺武林秘笈。\zhdash 慕容博}
% \end{flushleft}
%
% \subsection{自定义以及其它}
%
% 文档类的配置文件|jnuthesis.cfg|中定义了很多固定词汇,一般无须修改。如果有特殊需求,
% 推荐在导言区使用\cs{renewcommand}。当然,导言区里可以直接使用中文。
%
% \section{实现细节}
%
% \subsection{定义选项}
%
% {\jnuthesis}宏包的默认选项为|adobefonts|。
%    \begin{macrocode}
%<*cls>
\newif\ifjnut@adobefonts\jnut@adobefontstrue
\newif\ifjnut@winfonts\jnut@winfontsfalse
\newif\ifjnut@linuxfonts\jnut@linuxfontsfalse
\newif\ifjnut@macfonts\jnut@macfontsfalse
\newif\ifjnut@backinfo\jnut@backinfotrue
\newif\ifjnut@phd\jnut@phdfalse
\newif\ifjnut@master\jnut@masterfalse
\newif\ifjnut@bachelor\jnut@bachelorfalse
\DeclareOption{adobefonts}{\jnut@adobefontstrue
  \jnut@winfontsfalse
  \jnut@linuxfontsfalse
  \jnut@macfontsfalse}
\DeclareOption{winfonts}{\jnut@winfontstrue
  \jnut@adobefontsfalse
  \jnut@linuxfontsfalse
  \jnut@macfontsfalse}
\DeclareOption{linuxfonts}{\jnut@linuxfontstrue
  \jnut@adobefontsfalse
  \jnut@winfontsfalse
  \jnut@macfontsfalse}
\DeclareOption{macfonts}{\jnut@macfontstrue
  \jnut@adobefontsfalse
  \jnut@winfontsfalse
  \jnut@linuxfontsfalse}
\DeclareOption{nobackinfo}{\jnut@backinfofalse}
\DeclareOption{phd}{\jnut@phdtrue
  \jnut@masterfalse
  \jnut@bachelorfalse}
\DeclareOption{master}{\jnut@mastertrue
  \jnut@phdfalse
  \jnut@bachelorfalse}
\DeclareOption{bachelor}{\jnut@bachelortrue
  \jnut@phdfalse
  \jnut@masterfalse}
%    \end{macrocode}
%
% 把没有定义的选项传递给底层的文档类,在这里为|book|。
%
%    \begin{macrocode}
\DeclareOption*{\PassOptionsToClass{\CurrentOption}{book}}
%    \end{macrocode}
%
% 处理选项:
%    \begin{macrocode}
\ProcessOptions\relax
%    \end{macrocode}
%
% \subsection{底层文档类}
%
% 文档基于{\LaTeX}的标准|book|类。正文使用小四字号(对应于12.05pt,这里近似使用12pt),
% 纸张使用A4,双面打印。
%    \begin{macrocode}
\LoadClass[12pt,a4paper,doubleside]{book}
%    \end{macrocode}
%
% \subsection{装载宏包}
%
% 使用本文档类所写的文档需要使用{\XeLaTeX}引擎处理,因此首先要检查引擎是否正确。
%    \begin{macrocode}
\RequirePackage{ifxetex}
\RequireXeTeX
%    \end{macrocode}
%
% 使用|lastpage|宏包来获得最后一页的页码,从而生成“第3页,共20页”这样的页码标签。
%    \begin{macrocode}
\RequirePackage{lastpage}
%    \end{macrocode}
%
% 使用|geometry|宏包定义页面布局,定义段间距。
%    \begin{macrocode}
\RequirePackage{geometry}
%    \end{macrocode}
%
% 使用|titlesec|宏包设置标题格式。
%    \begin{macrocode}
\RequirePackage{titlesec}
%    \end{macrocode}
%
% 使用|graphicx|宏包支持插入图片。
%    \begin{macrocode}
\RequirePackage{graphicx}
%    \end{macrocode}
%
% 如果插入的图片没有指定扩展名,那么依次搜索下面的扩展名所对应的文件
%    \begin{macrocode}
\DeclareGraphicsExtensions{.pdf,.eps,.jpg,.png}
%    \end{macrocode}
%
% |caption2|宏包已经不再推荐使用,改用新的|caption|宏包处理浮动图形和表格的标题
% 样式。
%    \begin{macrocode}
\RequirePackage{caption}
%    \end{macrocode}
%
% |float|宏包为浮动图形和表格环境提供了一个H选项,强制将其放在当前位置。
%    \begin{macrocode}
\RequirePackage{float}
%    \end{macrocode}
%
% |subfigure|宏包已经不再推荐使用,改用新的|subfig|宏包支持插入子图和子表。
%    \begin{macrocode}
\RequirePackage{subfig}
%    \end{macrocode}
%
% 使用|array|宏包扩展表格的列选项。
%    \begin{macrocode}
\RequirePackage{array}
%    \end{macrocode}
%
% 使用|longtable|宏包处理长表格。
%    \begin{macrocode}
\RequirePackage{longtable}
%    \end{macrocode}
%
% |booktabs|宏包可生成三线表,支持\cs{toprule},\cs{midrule},\cs{bottomrulle}等命令。
%    \begin{macrocode}
\RequirePackage{booktabs}
%    \end{macrocode}
%
% |multirow|宏包支持在表格中跨行。
%    \begin{macrocode}
\RequirePackage{multirow}
%    \end{macrocode}
%
% |enumitem|宏包支持自定义列表环境。
%    \begin{macrocode}
\RequirePackage{enumitem}
%    \end{macrocode}
%
% |xcolor|宏包提供色彩控制。
%    \begin{macrocode}
\RequirePackage{xcolor}
%    \end{macrocode}
%
% |amsmath|宏包提供数学公式支持。
%    \begin{macrocode}
\RequirePackage{amsmath}
%    \end{macrocode}
%
% |amsthm|宏包支持自定义数学定理环境。
%    \begin{macrocode}
\RequirePackage{amsthm}
%    \end{macrocode}
%
% |amsfonts|宏包、|amssymb|宏包、|bm|宏包和|mathrsfs|宏包提供数学符号和字体支持。
%    \begin{macrocode}
\RequirePackage{amsfonts}
\RequirePackage{amssymb}
\RequirePackage{bm}
\RequirePackage{mathrsfs}
%    \end{macrocode}
%
% |wasysym|宏包提供特殊符号支持。
%    \begin{macrocode}
\RequirePackage{wasysym}
%    \end{macrocode}
%
% |pifont|宏包提供带圈的数字符号。
%    \begin{macrocode}
\RequirePackage{pifont}
%    \end{macrocode}
%
% |txfonts|宏包用自己的typewriter字体替换系统Courier字体,它必须在{\AmSTeX}之后引入。
%    \begin{macrocode}
\RequirePackage{txfonts}
%    \end{macrocode}
%
% |setspace|宏包支持行距控制。它需要在|hyperref|宏包之前加载,避免脚注超链接失效。
%    \begin{macrocode}
\RequirePackage{setspace}
%    \end{macrocode}
%
% |fancyhdr|宏包支持自定义页眉页脚。
%    \begin{macrocode}
\RequirePackage{fancyhdr}
%    \end{macrocode}
%
% |shortvrb|提供了一个\cs{MakeShortVerb}命令,可将某个符号定义为\cs{verb}命令的缩写。
%    \begin{macrocode}
\RequirePackage{shortvrb}
%    \end{macrocode}
%
% 使用|xltxtra|宏包来获取{\XeLaTeX}的符号。
%    \begin{macrocode}
\RequirePackage{xltxtra}
%    \end{macrocode}
%
% 使用|xeCJK|宏包处理中文。宏包选项|CJKnumber|表示调用|CJKnumber|宏包处理中文数
% 字;|CJKchecksingle|表示避免单个汉字单独占一行。|xeCJK|宏包必须放在|amssymb|之后,
% 否则会有冲突。
% \begin{note}
%   因为我们将使用黑体作为粗体,使用楷体作为斜体,因此载入|xeCJK|宏包时不需要开启
%   |BoldFont|选项和|SlantFont|;否则的话,|xeCJK|会自动生成宋体的粗体和斜体,而那会
%   非常难看。然而江南大学的格式要求在很多地方使用了粗体宋体,我们只得将|AutoFakeBold|
%   打开。
% \end{note}
% \begin{note}
% 由于TeX Live升级到2014版后,直接用|xeCJK|的|CJKnumber|选项会出现bug,我们需单独导入
% |CJKnumb|宏包;但|xeCJK|的|CJKnumber|选项依然需要,否则在Tex Live 2012下编译会报错。
% \end{note}
%    \begin{macrocode}
\RequirePackage[CJKnumber,CJKchecksingle,AutoFakeBold=true]{xeCJK}
\RequirePackage{CJKnumb}
%    \end{macrocode}
%
% 让{\XeLaTeX}能够处理dash的惯例(使用"--"和"---"获得较长的dash)。
%    \begin{macrocode}
\defaultfontfeatures{Mapping=tex-text}
%    \end{macrocode}
%
% 设置中文标点格式,使用|plain|方案。其他可选方案参见|xeCJK|文档。
%    \begin{macrocode}
\punctstyle{plain}
%    \end{macrocode}
%
% |xeCJKfntef|宏包提供了中文下划线命令\cs{CJKunderline},它将在制作论文封面时用到。
%    \begin{macrocode}
\RequirePackage{xeCJKfntef}
%    \end{macrocode}
%
% 设置中文下划线颜色为黑色。
%    \begin{macrocode}
\xeCJKsetup{ underline/format = \color{black} }
%    \end{macrocode}
%
% 使用|indentfirst|宏包支持首行缩进。
%    \begin{macrocode}
%\RequirePackage{indentfirst}
%    \end{macrocode}
%
% |url|宏包支持超链接排版,我们为它提供|hyphens|选项,从而使得长链接可在连字符处自动折行。
% 注意|url|宏包必须在|hyperref|宏包之前载入,否则其选项不起作用。
%    \begin{macrocode}
\RequirePackage[hyphens]{url}
%    \end{macrocode}
%
% |hyperref|宏包可根据交叉引用生成超链接,同时生成PDF文档的书签。
%    \begin{macrocode}
\RequirePackage{hyperref}
%    \end{macrocode}
%
% 设置|hyperref|宏包参数。|hyperref|配合{\XeTeX}使用时不能开启Unicode选项。
%    \begin{macrocode}
\hypersetup{%
    unicode=false,
    hyperfootnotes=true,
    hyperindex=true,
    pageanchor=true,
    CJKbookmarks=true,
    bookmarksnumbered=true,
    bookmarksopen=true,
    bookmarksopenlevel=0,
    breaklinks=true,
    colorlinks=false,
    plainpages=false,
    pdfpagelabels,
    pdfborder=0 0 0%
}
%    \end{macrocode}
%
% 设置URL样式,使其与上下文一致。
%    \begin{macrocode}
\urlstyle{same}
%    \end{macrocode}
%
% 美化参考文献排序和引用格式的宏包|natbib|。
%    \begin{macrocode}
\RequirePackage[sort&compress,numbers]{natbib}
%    \end{macrocode}
%
% |hypernat|可以让|hyperref|和|natbib|混合使用,但它需要放在这两者之后。
%    \begin{macrocode}
\RequirePackage{hypernat}
%    \end{macrocode}
%
% |tabularx|宏包支持自动扩展的列宽,但它需要在|hyperref|之后引入才不会导致正文
% 的footnote的超链接失效。
%    \begin{macrocode}
\RequirePackage{tabularx}
%    \end{macrocode}
%
% |makeidx|宏包支持建立索引。
%    \begin{macrocode}
\RequirePackage{makeidx}
%    \end{macrocode}
%
% |glossaries|宏包可用于制作术语表。但该宏包必须在|hyperref|之后载入。
%    \begin{macrocode}
\RequirePackage{glossaries}
%    \end{macrocode}
%
% |ifthen|宏包提供了\cs{ifthenelse}命令,本文档类将使用该命令定义一些其他命令。
%    \begin{macrocode}
\RequirePackage{ifthen}
%    \end{macrocode}
%
% |footmisc|宏包提供了对脚注样式的控制功能。
%    \begin{macrocode}
\RequirePackage[perpage,symbol*]{footmisc}
%</cls>
%    \end{macrocode}
%
% |etoolbox|宏包提供了一些工具宏。
%    \begin{macrocode}
\RequirePackage{etoolbox}
%    \end{macrocode}
%
% \subsection{字符串常量定义}
%
% 定义论文中各章节的中文标题名称字符串常量:
%    \begin{macrocode}
%<*cfg>
\newcommand*{\jnut@cap@abstractname}{摘\hspace{2em}要}
\newcommand*{\jnut@cap@contentsname}{目\hspace{2em}次}
\newcommand*{\jnut@cap@revisionhistory}{修订历史}
\newcommand*{\jnut@cap@listfigurename}{插图清单}
\newcommand*{\jnut@cap@listtablename}{附表清单}
\newcommand*{\jnut@cap@listsymbolname}{符号清单}
\newcommand*{\jnut@cap@listequationname}{公式清单}
\newcommand*{\jnut@cap@equationname}{公式}
\newcommand*{\jnut@cap@bibname}{参考文献}
\newcommand*{\jnut@cap@glossaryname}{术\hspace{0.5em}语\hspace{0.5em}表}
\newcommand*{\jnut@cap@indexname}{索\hspace{2em}引}
\newcommand*{\jnut@cap@figurename}{图}
\newcommand*{\jnut@cap@tablename}{表}
\newcommand*{\jnut@cap@preface}{前\hspace{2em}言}
\newcommand*{\jnut@cap@acknowledgementname}{致\hspace{2em}谢}
\newcommand*{\jnut@cap@appendixname}{附录\thechapter}
%    \end{macrocode}
%
% 定义用于重定义\cs{chaptername}命令的常量。若当前所处位置是文档的|mainmatter|部分,ze
% 将其定义为``第XX章''的形式,否则将其定义为空字符串。
%    \begin{macrocode}
\newcommand*{\jnut@cap@chaptername}{%
  \if@mainmatter{第\CJKnumber{\thechapter}章}\fi%
}
%    \end{macrocode}
%
% 定义常用数学定理环境的字符串常量:
%    \begin{macrocode}
\newcommand*{\jnut@cap@definition}{定义}
\newcommand*{\jnut@cap@notation}{记号}
\newcommand*{\jnut@cap@theorem}{定理}
\newcommand*{\jnut@cap@lemma}{引理}
\newcommand*{\jnut@cap@corollary}{推论}
\newcommand*{\jnut@cap@proposition}{命题}
\newcommand*{\jnut@cap@fact}{事实}
\newcommand*{\jnut@cap@assumption}{假设}
\newcommand*{\jnut@cap@conjecture}{猜想}
\newcommand*{\jnut@cap@hypothesis}{假说}
\newcommand*{\jnut@cap@axiom}{公理}
\newcommand*{\jnut@cap@postulate}{公设}
\newcommand*{\jnut@cap@principle}{定律}
\newcommand*{\jnut@cap@problem}{问题}
\newcommand*{\jnut@cap@exercise}{练习}
\newcommand*{\jnut@cap@example}{例}
\newcommand*{\jnut@cap@remark}{评注}
\newcommand*{\jnut@cap@proof}{证明}
\newcommand*{\jnut@cap@solution}{解}
\newcommand*{\jnut@cap@algorithm}{算法}
%    \end{macrocode}
%
% 定义自定义列表环境的字符串常量:
%    \begin{macrocode}
\newcommand*{\jnut@cap@case}{情况}
\newcommand*{\jnut@cap@subcase}{子情况}
\newcommand*{\jnut@cap@step}{步骤}
\newcommand*{\jnut@cap@substep}{子步骤}
%    \end{macrocode}
%
% 定义日期中的中文字符:
%    \begin{macrocode}
\newcommand*{\jnut@cap@year}{年}
\newcommand*{\jnut@cap@month}{月}
\newcommand*{\jnut@cap@day}{日}
\newcommand*{\jnut@cap@to}{至}
%    \end{macrocode}
%
% 定义学位名称的中英文字符串常量:
%    \begin{macrocode}
\newcommand*{\jnut@cap@phd}{博士}
\newcommand*{\jnut@cap@master}{硕士}
\newcommand*{\jnut@cap@bachelor}{学士}
\newcommand*{\jnut@cap@en@phd}{Doctor of Philosophy}
\newcommand*{\jnut@cap@en@master}{Master}
\newcommand*{\jnut@cap@en@bachelor}{Bachelor}
%    \end{macrocode}
%
% 定义国家图书馆(NLC)封面的字符串常量:
%    \begin{macrocode}
\newcommand*{\jnut@cap@nlc}{国家图书馆封面}
\newcommand*{\jnut@cap@nlc@classification}{分类号}
\newcommand*{\jnut@cap@nlc@securitylevel}{密级}
\newcommand*{\jnut@cap@nlc@udc}{UDC}
\newcommand*{\jnut@cap@nlc@title}{%
学\hspace{1em}位\hspace{1em}论\hspace{1em}文%
}
\newcommand*{\jnut@cap@nlc@quotetitle}{(题名和副题名)}
\newcommand*{\jnut@cap@nlc@author}{(作者姓名)}
\newcommand*{\jnut@cap@nlc@supervisor}{%
指导教师姓名、职务、职称、学位、单位名称及地址%
}
\newcommand*{\jnut@cap@nlc@degree}{申请学位级别}
\newcommand*{\jnut@cap@nlc@major}{专业名称}
\newcommand*{\jnut@cap@nlc@submitdate}{论文提交日期}
\newcommand*{\jnut@cap@nlc@defenddate}{论文答辩日期}
\newcommand*{\jnut@cap@nlc@institute}{学位授予单位和日期}
\newcommand*{\jnut@cap@nlc@chairman}{答辩委员会主席:}
\newcommand*{\jnut@cap@nlc@reviwer}{评阅人:}
%    \end{macrocode}
%
% 定义标准的文献密级汉字代码:
%    \begin{macrocode}
\newcommand*{\jnut@cap@nlc@openlevel}{公开}
\newcommand*{\jnut@cap@nlc@controllevel}{限制}
\newcommand*{\jnut@cap@nlc@confidentiallevel}{秘密}
\newcommand*{\jnut@cap@nlc@clasifiedlevel}{机密}
\newcommand*{\jnut@cap@nlc@mostconfidentiallevel}{绝密}
%    \end{macrocode}
%
% 定义江南大学学位论文中文封面的字符串常量:
%    \begin{macrocode}
\newcommand*{\jnut@cap@cover}{中文封面}
\newcommand*{\jnut@cap@cover@thesis}{{\jnut@value@degree}学位论文}
\newcommand*{\jnut@cap@cover@apply}{}
\newcommand*{\jnut@cap@cover@title}{题    目}
\newcommand*{\jnut@cap@cover@entitle}{英文并列题目}
\newcommand*{\jnut@cap@cover@author}{研  究  生}
\newcommand*{\jnut@cap@cover@supervisor}{导    师}
\newcommand*{\jnut@cap@cover@major}{专    业}
\newcommand*{\jnut@cap@cover@researchfield}{研 究 方 向}
\newcommand*{\jnut@cap@cover@department}{院系}
\newcommand*{\jnut@cap@cover@directormembers}{指导小组成员}
\newcommand*{\jnut@cap@cover@degreedate}{学位授予日期}
\newcommand*{\jnut@cap@cover@institute}{江南大学}
%    \end{macrocode}
%
% 定义江南大学学位论文中文封面背面的字符串常量:
%    \begin{macrocode}
\newcommand*{\jnut@cap@coverback@supervisor}{指导教师}
\newcommand*{\jnut@cap@coverback@studentnum}{学号}
\newcommand*{\jnut@cap@coverback@defenddate}{论文答辩日期}
\newcommand*{\jnut@cap@coverback@sign}{\hspace{10em}(签字)}
%    \end{macrocode}
%
% 定义江南大学学位论文英文封面的字符串常量:
%    \begin{macrocode}
\newcommand*{\jnut@cap@cover@en@by}{by}
\newcommand*{\jnut@cap@cover@en@in}{in}
\newcommand*{\jnut@cap@cover@en@supervisor}{Supervised by}
\newcommand*{\jnut@cap@cover@en@statement}{%
A dissertation submitted to\\
the graduate school of {\jnut@value@en@institute}\\
in partial fulfilment of the requirements for the degree of\\
{\textsc{\jnut@value@en@degree}}\\
in\\
{\jnut@value@en@major}
}
%    \end{macrocode}
%
% 定义江南大学学位论文中文摘要页的字符串常量:
%    \begin{macrocode}
\newcommand*{\jnut@cap@abstract}{中文摘要}
\newcommand*{\jnut@cap@abstract@abstractname}{摘 要}
\newcommand*{\jnut@cap@abstract@keywordsname}{关键词}
%    \end{macrocode}
%
% 定义江南大学学位论文英文摘要页的字符串常量:
%    \begin{macrocode}
\newcommand*{\jnut@cap@abstract@en}{英文摘要}
\newcommand*{\jnut@cap@abstract@en@abstractname}{Abstract}
\newcommand*{\jnut@cap@abstract@en@keywordsname}{keywords}
%    \end{macrocode}
%
% 定义江南大学学位论文中论文作者简历与科研成果页的字符串常量:
%    \begin{macrocode}
%% TODO: 完善此处格式
\newcommand*{\jnut@cap@resume@chaptername}{附  录: 作者在攻读{\jnut@value@degree}学位期间发表的论文}
\newcommand*{\jnut@cap@resume@authorinfo}{基本信息}
\newcommand*{\jnut@cap@resume@education}{教育背景}
\newcommand*{\jnut@cap@resume@publications}{%
攻读{\jnut@value@degree}学位期间完成的学术成果%
}
\newcommand*{\jnut@cap@resume@projects}{%
攻读{\jnut@value@degree}学位期间参与的科研课题%
}
%    \end{macrocode}
%
% 定义《学位论文出版授权书》中的字符串常量:
%    \begin{macrocode}
\newcommand*{\jnut@cap@license@chaptername}{学位论文出版授权书}
\newcommand*{\jnut@cap@license@declaration}{%
本人完全同意《中国优秀博硕士学位论文全文数据库出版章程》(以下简称“章程”),%
愿意将本人的学位论文提交“中国学术期刊(光盘版)电子杂志社”在《中国博士学位论%
文全文数据库》、《中国优秀硕士学位论文全文数据库》中全文发表。《中国博士学位论%
文全文数据库》、《中国优秀硕士学位论文全文数据库》可以以电子、网络及其他数字媒%
体形式公开出版,并同意编入《中国知识资源总库》,在《中国博硕士学位论文评价数据%
库》中使用和在互联网上传播,同意按“章程”规定享受相关权益。%
}
\newcommand*{\jnut@cap@license@sign}{作者签名:}
\newcommand*{\jnut@cap@license@securitylevel}{论文涉密情况:}
\newcommand*{\jnut@cap@license@public}{不保密}
\newcommand*{\jnut@cap@license@secret}{保密,保密期:}
\newcommand*{\jnut@cap@license@title}{论文题名}
\newcommand*{\jnut@cap@license@studentnum}{研究生学号}
\newcommand*{\jnut@cap@license@department}{所在院系}
\newcommand*{\jnut@cap@license@grade}{学位年度}
\newcommand*{\jnut@cap@license@category}{论文级别}
\newcommand*{\jnut@cap@license@telphone}{作者电话}
\newcommand*{\jnut@cap@license@email}{作者Email}
\newcommand*{\jnut@cap@license@supervisorname}{第一导师姓名}
\newcommand*{\jnut@cap@license@supervisortelphone}{导师电话}
\newcommand*{\jnut@cap@license@categoryhint}{(请在方框内画勾)}
\newcommand*{\jnut@cap@license@categorymaster}{硕士}
\newcommand*{\jnut@cap@license@categoryphd}{博士}
\newcommand*{\jnut@cap@license@categorymasterspec}{硕士专业学位}
\newcommand*{\jnut@cap@license@categoryphdspec}{博士专业学位}
\newcommand*{\jnut@cap@license@remark}{%
注:请将该授权书填写后装订在学位论文最后一页(南大封面)。%
}
%    \end{macrocode}
%
% 定义学位颁发机构的校徽和校名图片文件名:
%    \begin{macrocode}
\newcommand*{\jnut@cap@institute@logo}{jnulogo.eps}
\newcommand*{\jnut@cap@institute@name}{jnuname.eps}
%    \end{macrocode}
%
% \subsection{字段默认值定义}
%
% 定义国家图书馆(NLC)封面中要填写的字段的默认值:
%    \begin{macrocode}
\newcommand*{\jnut@value@nlc@classification}{(分类)}
\newcommand*{\jnut@value@nlc@securitylevel}{\openlevel}
\newcommand*{\jnut@value@nlc@udc}{}
\newcommand*{\jnut@value@nlc@titlea}{\jnut@value@title}
\newcommand*{\jnut@value@nlc@titleb}{}
\newcommand*{\jnut@value@nlc@titlec}{}
\newcommand*{\jnut@value@nlc@supervisorinfo}{%
(导师的职务、职称、学位、单位名称及地址)%
}
\newcommand*{\jnut@value@nlc@chairman}{(答辩主席)}
\newcommand*{\jnut@value@nlc@reviewera}{(评审人)}
\newcommand*{\jnut@value@nlc@reviewerb}{}
\newcommand*{\jnut@value@nlc@reviewerc}{}
\newcommand*{\jnut@value@nlc@reviewerd}{}
%    \end{macrocode}
%
% 定义江南大学学位论文中文封面中要填写的字段的默认值:
%    \begin{macrocode}
\ifjnut@phd
    \newcommand*{\jnut@value@degree}{\jnut@cap@phd}
\else
    \ifjnut@master
        \newcommand*{\jnut@value@degree}{\jnut@cap@master}
    \else
       \ifjnut@bachelor
          \newcommand*{\jnut@value@degree}{\jnut@cap@bachelor}
       \else
          \ClassError{jnuthesis}{No degree was selected.}{}
       \fi
    \fi
\fi
\newcommand*{\jnut@value@title}{(论文标题)}
\newcommand*{\jnut@value@titlea}{(论文长标题第一行)}
\newcommand*{\jnut@value@titleb}{(论文长标题第二行)}
\newcommand*{\jnut@value@author}{(作者姓名)}
\newcommand*{\jnut@value@telphone}{(作者电话号码)}
\newcommand*{\jnut@value@email}{(作者电子邮件)}
\newcommand*{\jnut@value@studentnum}{XXXXXXXX}
\newcommand*{\jnut@value@grade}{XXXX}
\newcommand*{\jnut@value@supervisor}{(导师姓名和职称)}
\newcommand*{\jnut@value@supervisortelphone}{(导师电话号码)}
\newcommand*{\jnut@value@major}{(作者专业)}
\newcommand*{\jnut@value@researchfield}{(作者研究方向)}
\newcommand*{\jnut@value@department}{(作者所属院系)}
\newcommand*{\jnut@value@directormembers}{(指导小组成员)}
\newcommand*{\jnut@value@degreedate}{(学位授予日期)}
\newcommand*{\jnut@value@institute}{江南大学}
\newcommand*{\jnut@value@submitdate}{xxxx年xx月xx日}
\newcommand*{\jnut@value@defenddate}{xxxx年xx月xx日}
\newcommand*{\jnut@value@date}{%
{\number\year}年{\number\month}月{\number\day}日%
}
%    \end{macrocode}
%
% 定义江南大学学位论文英文封面中要填写的字段的默认值:
%    \begin{macrocode}
\ifjnut@phd
    \newcommand*{\jnut@value@en@degree}{\jnut@cap@en@phd}
\else
    \ifjnut@master
        \newcommand*{\jnut@value@en@degree}{\jnut@cap@en@master}
    \else
       \ifjnut@bachelor
          \newcommand*{\jnut@value@en@degree}{\jnut@cap@en@bachelor}
       \else
          \ClassError{jnuthesis}{No degree was selected.}{}
       \fi
    \fi
\fi
\newcommand*{\jnut@value@en@titlea}{(English Title of Thesis)}
\newcommand*{\jnut@value@en@titleb}{(English Title of Thesis)}
\newcommand*{\jnut@value@en@author}{(Author's Name)}
\newcommand*{\jnut@value@en@supervisor}{Professor (Supervisor's Name)}
\newcommand*{\jnut@value@en@major}{Author's Major}
\newcommand*{\jnut@value@en@department}{(Department's Name)}
\newcommand*{\jnut@value@en@institute}{Nanjing University}
\newcommand*{\jnut@value@en@date}{
\ifcase\month\or
January\or
February\or
March\or
April\or
May\or
June\or
July\or
August\or
September\or
October\or
November\or
December\fi
\number\day, \number\year%
}
%    \end{macrocode}
%
% 定义江南大学学位论文中文摘要页中要填写的字段的默认值:
%    \begin{macrocode}
\newcommand*{\jnut@value@abstract@titlea}{\jnut@value@title}
\newcommand*{\jnut@value@abstract@titleb}{}
\newcommand*{\jnut@value@abstract@keywords}{}
%    \end{macrocode}
%
% 定义江南大学学位论文英文摘要页中要填写的字段的默认值:
%    \begin{macrocode}
\newcommand*{\jnut@value@abstract@en@titlea}{\jnut@value@en@titlea}
\newcommand*{\jnut@value@abstract@en@titleb}{\jnut@value@en@titleb}
\newcommand*{\jnut@value@abstract@en@keywords}{}
%    \end{macrocode}
%
% \subsection{格式控制常量定义}
%
% 定义Windows下宋体、黑体、楷书和仿宋体四种中文字体的名称。默认采用微软字体。
%    \begin{macrocode}
\newcommand*{\jnut@zhfn@songti@win}{SimSun}
\newcommand*{\jnut@zhfn@heiti@win}{SimHei}
\newcommand*{\jnut@zhfn@kaishu@win}{KaiTi}
\newcommand*{\jnut@zhfn@fangsong@win}{FangSong}
%    \end{macrocode}
%
% 定义Windows下英文字体的名称。默认采用Windows自带的字体。
%    \begin{macrocode}
\newcommand*{\jnut@enfn@main@win}{Times New Roman}
\newcommand*{\jnut@enfn@sans@win}{Arial}
\newcommand*{\jnut@enfn@mono@win}{Courier New}
%    \end{macrocode}
%
% 定义Linux下宋体、黑体、楷书和仿宋体四种中文字体的名称。默认采用文鼎宋体、楷体;
% 文泉黑体;以及华文仿宋体(需要单独安装)。
%    \begin{macrocode}
\newcommand*{\jnut@zhfn@songti@linux}{SimSun}
\newcommand*{\jnut@zhfn@heiti@linux}{SimHei}
	\newcommand*{\jnut@zhfn@kaishu@linux}{KaiTi}
\newcommand*{\jnut@zhfn@fangsong@linux}{FangSong}
%    \end{macrocode}
%
% 定义Linux下英文字体的名称。默认采用的字体若未安装请自行安装。
%    \begin{macrocode}
\newcommand*{\jnut@enfn@main@linux}{Times New Roman}
\newcommand*{\jnut@enfn@sans@linux}{Arial}
\newcommand*{\jnut@enfn@mono@linux}{Courier New}
%    \end{macrocode}
%
% 定义Mac下宋体、黑体、楷书和仿宋体四种中文字体的名称。默认采用华文字体。
%    \begin{macrocode}
\newcommand*{\jnut@zhfn@songti@mac}{STSong}
\newcommand*{\jnut@zhfn@heiti@mac}{STHeiti}
\newcommand*{\jnut@zhfn@kaishu@mac}{STKaiti}
\newcommand*{\jnut@zhfn@fangsong@mac}{STFangsong}
%    \end{macrocode}
%
% 定义Mac下英文字体的名称。默认采用Mac自带的字体。
%    \begin{macrocode}
\newcommand*{\jnut@enfn@main@mac}{Times}
\newcommand*{\jnut@enfn@sans@mac}{Helvetica}
\newcommand*{\jnut@enfn@mono@mac}{Courier}
%    \end{macrocode}
%
% 定义Adoble提供的宋体、黑体、楷书和仿宋体四种中文字体的名称。Adoble的宋体、黑体和
% 仿宋体可以在其网站免费下载,地址为
% \begin{center}
%  \url{http://www.adobe.com/support/downloads/detail.jsp?ftpID=4421}
% \end{center}
% 但Adobe的楷体只随Adobe Creative Suite软件提供。不过,所有Adobe中文字体都可以在这里
% 打包下载:
% \begin{center}
% \url{http://tinker-bot.googlecode.com/files/cfonts.tar.gz}
% \end{center}
%    \begin{macrocode}
\newcommand*{\jnut@zhfn@songti@adobe}{Adobe Song Std}
\newcommand*{\jnut@zhfn@heiti@adobe}{Adobe Heiti Std}
\newcommand*{\jnut@zhfn@kaishu@adobe}{Adobe Kaiti Std}
\newcommand*{\jnut@zhfn@fangsong@adobe}{Adobe Fangsong Std}
%    \end{macrocode}
%
% 定义英文字体的名称。默认采用Mac自带的字体。
%    \begin{macrocode}
\newcommand*{\jnut@enfn@main@adobe}{Times New Roman}
\newcommand*{\jnut@enfn@sans@adobe}{Arial}
\newcommand*{\jnut@enfn@mono@adobe}{Courier New}
%</cfg>
%    \end{macrocode}
%
% \subsection{载入字符串常量配置}
%
% 在进行其他配置之前先载入预定义的字符串常量配置。
%    \begin{macrocode}
%<*cls>
%%
%% This is file `jnuthesis.cfg',
%% generated with the docstrip utility.
%%
%% The original source files were:
%%
%% jnuthesis.dtx  (with options: `cfg')
%% This is a generated file.
%% 
%% Copyright (C) 2013-2016, Haixing Hu.
%% Department of Computer Science and Technology, Nanjing University.
%% 
%% Home Page of the Project: http://haixing-hu.github.io/nju-thesis/
%% 
%% It may be distributed and/or modified under the conditions of the LaTeX Project
%% Public License, either version 1.2 of this license or (at your option) any
%% later version.  The latest version of this license is in
%% 
%%    http://www.latex-project.org/lppl.txt
%% 
%% and version 1.2 or later is part of all distributions of LaTeX version
%% 1999/12/01 or later.
%% 
%% This is the configuration file of the njuthesis package with XeLaTeX.
\ProvidesFile{jnuthesis.cfg}
%% {\jnuthesis}所遵循的江南大学研究生院的要求如下:
%% \begin{itemize}
%% \item \textsl{江南大学硕士答辩流程及相关材料下载},\\
%% \url{http://gs.nju.edu.cn/content/xw/ss3.htm}
%% \item \textsl{江南大学博士答辩流程及相关材料下载},\\
%% \url{http://gs.nju.edu.cn/content/xw/bs3.htm}
%% \item \textsl{江南大学博士(硕士)学位论文编写格式规定(试行)}\\
%% \url{http://grawww.nju.edu.cn/content/xw/lwgf.doc}
%% \end{itemize}
\RequirePackage{etoolbox}
\RequirePackage{xstring}
\RequirePackage{xifthen}
\RequirePackage{xparse}
\newcommand*{\jnut@cap@abstractname}{摘 要}
\newcommand*{\jnut@cap@contentsname}{目 录}
\newcommand*{\jnut@cap@contentsname@bachelorhead}{目录}
\newcommand*{\jnut@cap@revisionhistory}{修订历史}
\newcommand*{\jnut@cap@listfigurename}{插图清单}
\newcommand*{\jnut@cap@listtablename}{附表清单}
\newcommand*{\jnut@cap@listsymbolname}{符号清单}
\newcommand*{\jnut@cap@listequationname}{公式清单}
\newcommand*{\jnut@cap@equationname}{公式}
\newcommand*{\jnut@cap@bibname}{参考文献}
\newcommand*{\jnut@cap@glossaryname}{术\hspace{0.5em}语\hspace{0.5em}表}
\newcommand*{\jnut@cap@indexname}{索 引}
\newcommand*{\jnut@cap@figurename}{图}
\newcommand*{\jnut@cap@tablename}{表}
\newcommand*{\jnut@cap@preface}{前 言}
\newcommand*{\jnut@cap@acknowledgementname}{致 谢}
\newcommand*{\jnut@cap@appendixname}{附录\arabic{chapter}}
\newcommand*{\jnut@cap@appendixname@head}{附录}
\newcommand*{\jnut@cap@chaptername}{%
  \if@mainmatter{第{\ifjnut@bachelor\arabic{chapter}\else\CJKnumber{\arabic{chapter}}\fi}章}\fi%
}
\newcommand*{\jnut@cap@definition}{定义}
\newcommand*{\jnut@cap@notation}{记号}
\newcommand*{\jnut@cap@theorem}{定理}
\newcommand*{\jnut@cap@lemma}{引理}
\newcommand*{\jnut@cap@corollary}{推论}
\newcommand*{\jnut@cap@proposition}{命题}
\newcommand*{\jnut@cap@fact}{事实}
\newcommand*{\jnut@cap@assumption}{假设}
\newcommand*{\jnut@cap@conjecture}{猜想}
\newcommand*{\jnut@cap@hypothesis}{假说}
\newcommand*{\jnut@cap@axiom}{公理}
\newcommand*{\jnut@cap@postulate}{公设}
\newcommand*{\jnut@cap@principle}{定律}
\newcommand*{\jnut@cap@problem}{问题}
\newcommand*{\jnut@cap@exercise}{练习}
\newcommand*{\jnut@cap@example}{例}
\newcommand*{\jnut@cap@remark}{评注}
\newcommand*{\jnut@cap@proof}{证明}
\newcommand*{\jnut@cap@solution}{解}
\newcommand*{\jnut@cap@algorithm}{算法}
\newcommand*{\jnut@cap@case}{情况}
\newcommand*{\jnut@cap@subcase}{子情况}
\newcommand*{\jnut@cap@step}{步骤}
\newcommand*{\jnut@cap@substep}{子步骤}
\newcommand*{\jnut@cap@year}{年}
\newcommand*{\jnut@cap@month}{月}
\newcommand*{\jnut@cap@day}{日}
\newcommand*{\jnut@cap@to}{至}
\newcommand*{\jnut@cap@phd}{博士}
\newcommand*{\jnut@cap@master}{硕士}
\newcommand*{\jnut@cap@bachelor}{学士}
\newcommand*{\jnut@cap@en@phd}{Doctor of Philosophy}
\newcommand*{\jnut@cap@en@master}{Master}
\newcommand*{\jnut@cap@en@bachelor}{Bachelor}
\newcommand*{\jnut@cap@nlc}{国家图书馆封面}
\newcommand*{\jnut@cap@nlc@classification}{分类号}
\newcommand*{\jnut@cap@nlc@securitylevel}{密级}
\newcommand*{\jnut@cap@nlc@udc}{UDC}
\newcommand*{\jnut@cap@nlc@title}{%
学\hspace{1em}位\hspace{1em}论\hspace{1em}文%
}
\newcommand*{\jnut@cap@nlc@quotetitle}{(题名和副题名)}
\newcommand*{\jnut@cap@nlc@author}{(作者姓名)}
\newcommand*{\jnut@cap@nlc@supervisor}{%
指导教师姓名、职务、职称、学位、单位名称及地址%
}
\newcommand*{\jnut@cap@nlc@degree}{申请学位级别}
\newcommand*{\jnut@cap@nlc@major}{专业名称}
\newcommand*{\jnut@cap@nlc@submitdate}{论文提交日期}
\newcommand*{\jnut@cap@nlc@defenddate}{论文答辩日期}
\newcommand*{\jnut@cap@nlc@institute}{学位授予单位和日期}
\newcommand*{\jnut@cap@nlc@chairman}{答辩委员会主席:}
\newcommand*{\jnut@cap@nlc@reviwer}{评阅人:}
\newcommand*{\jnut@cap@nlc@openlevel}{公开}
\newcommand*{\jnut@cap@nlc@controllevel}{限制}
\newcommand*{\jnut@cap@nlc@confidentiallevel}{秘密}
\newcommand*{\jnut@cap@nlc@clasifiedlevel}{机密}
\newcommand*{\jnut@cap@nlc@mostconfidentiallevel}{绝密}
\newcommand*\jnut@circled[1]{\tikz[baseline=(char.base)]{
  \node[shape=circle,draw,inner sep=1pt] (char) {#1};}}
\newcommand*{\jnut@cap@cover}{中文封面}
\newcommand*{\jnut@cap@cover@prodegree@fullday}{全日制专业学位}
\newcommand*{\jnut@cap@cover@prodegree@maybe}{\ifjnut@prodegree\jnut@cap@cover@prodegree@fullday\fi}
\newcommand*{\jnut@cap@cover@thesis}{\jnut@cap@cover@prodegree@maybe{\jnut@value@degree}学位论文}
\newcommand*{\jnut@cap@cover@bachelorthesis@normal}{本科生毕业设计(论文)}
\newcommand*{\jnut@cap@cover@bachelorthesis@related}{毕业设计(论文)}
\newcommand*{\jnut@cap@cover@bachelorthesis@related@b}{相关资料}
\newcommand*{\jnut@cap@cover@bachelorthesis@translation@b}{外文资料翻译}
\newcommand*{\jnut@cap@cover@bachelorthesis@translation@c}{及原文}
\newcommand*{\jnut@cap@cover@bachelorthesis@coursework@a}{《{\jnut@value@coursename}》}
\newcommand*{\jnut@cap@cover@bachelorthesis@coursework@b}{课程设计}
\newcommand*{\jnut@cap@cover@masterthesis@blindreview}{{\jnut@cap@cover@thesis}抽检封面}
\newcommand*{\jnut@cap@cover@apply}{}
\newcommand*{\jnut@cap@cover@title}{题    目}
\newcommand*{\jnut@cap@cover@bachelortitle}{题目:}
\newcommand*{\jnut@cap@cover@entitle}{英文并列题目}
\newcommand*{\jnut@cap@cover@blindreview@number@a}{盲审论文编号}
\newcommand*{\jnut@cap@cover@blindreview@number@b}{[由学位办填写]}
\newcommand*{\jnut@cap@cover@author}{研  究  生}
\newcommand*{\jnut@cap@cover@bachelorauthor}{学生姓名}
\newcommand*{\jnut@cap@cover@bachelorstudentnum}{学  号}
\newcommand*{\jnut@cap@cover@supervisor}{导    师}
\newcommand*{\jnut@cap@cover@supervisor@prodegree@a}{导师~\jnut@circled{1}~姓名}
\newcommand*{\jnut@cap@cover@supervisor@prodegree@b}{导师~\jnut@circled{2}~姓名}
\newcommand*{\jnut@cap@cover@supervisor@prodegree@pos}{职    称}
\newcommand*{\jnut@cap@cover@bachelorsupervisor}{指导教师}
\newcommand*{\jnut@cap@cover@major}{专    业}
\newcommand*{\jnut@cap@cover@major@prodegree}{专 业 名 称}
\newcommand*{\jnut@cap@cover@bachelormajor}{专业}
\newcommand*{\jnut@cap@cover@researchfield}{研 究 方 向}
\newcommand*{\jnut@cap@cover@department}{院系}
\newcommand*{\jnut@cap@cover@bachelordepartment}{学院}
\newcommand*{\jnut@cap@cover@directormembers}{指导小组成员}
\newcommand*{\jnut@cap@cover@degreedate}{学位授予日期}
\newcommand*{\jnut@cap@cover@institute}{江南大学}
\newcommand*{\jnut@cap@coverback@supervisor}{指导教师}
\newcommand*{\jnut@cap@coverback@studentnum}{学号}
\newcommand*{\jnut@cap@coverback@defenddate}{论文答辩日期}
\newcommand*{\jnut@cap@coverback@sign}{\hspace{10em}(签字)}
\newcommand*{\jnut@cap@cover@en@by}{by}
\newcommand*{\jnut@cap@cover@en@in}{in}
\newcommand*{\jnut@cap@cover@en@supervisor}{Supervised by}
\newcommand*{\jnut@cap@cover@en@statement}{%
A dissertation submitted to\\
the graduate school of {\jnut@value@en@institute}\\
in partial fulfilment of the requirements for the degree of\\
{\textsc{\jnut@value@en@degree}}\\
in\\
{\jnut@value@en@major}
}
\newcommand*{\jnut@cap@abstract}{中文摘要}
\newcommand*{\jnut@cap@abstract@abstractname}{摘 要}
\newcommand*{\jnut@cap@abstract@abstractname@bachelorhead}{摘要}
\newcommand*{\jnut@cap@abstract@bachelordesign}{设计总说明}
\newcommand*{\jnut@cap@abstract@keywordsname}{关键词}
\newcommand*{\jnut@cap@abstract@en}{英文摘要}
\newcommand*{\jnut@cap@abstract@en@abstractname@bachelor}{ABSTRACT}
\newcommand*{\jnut@cap@abstract@en@abstractname@bachelorhead}{Abstract}
\newcommand*{\jnut@cap@abstract@en@abstractname@other}{Abstract}
\newcommand*{\jnut@cap@abstract@en@abstractname}{%
  \ifjnut@bachelor%
  \jnut@cap@abstract@en@abstractname@bachelor%
  \else%
  \jnut@cap@abstract@en@abstractname@other%
  \fi%
}
\newcommand*{\jnut@cap@abstract@en@keywordsname}{Keywords}
%% TODO: 完善此处格式
\newcommand*{\jnut@cap@resume@chaptername}{附  录: 作者在攻读{\jnut@value@degree}学位期间发表的论文}
\newcommand*{\jnut@cap@resume@authorinfo}{基本信息}
\newcommand*{\jnut@cap@resume@education}{教育背景}
\newcommand*{\jnut@cap@resume@publications}{%
攻读{\jnut@value@degree}学位期间完成的学术成果%
}
\newcommand*{\jnut@cap@resume@projects}{%
攻读{\jnut@value@degree}学位期间参与的科研课题%
}
\newcommand*{\jnut@cap@institute@name}{jnuname}
\newcommand*{\jnut@value@nlc@classification}{(分类)}
\newcommand*{\jnut@value@nlc@securitylevel}{\openlevel}
\newcommand*{\jnut@value@nlc@udc}{}
\newcommand*{\jnut@value@nlc@titlea}{\jnut@value@title}
\newcommand*{\jnut@value@nlc@titleb}{}
\newcommand*{\jnut@value@nlc@titlec}{}
\newcommand*{\jnut@value@nlc@supervisorinfo}{%
(导师的职务、职称、学位、单位名称及地址)%
}
\newcommand*{\jnut@value@nlc@chairman}{(答辩主席)}
\newcommand*{\jnut@value@nlc@reviewera}{(评审人)}
\newcommand*{\jnut@value@nlc@reviewerb}{}
\newcommand*{\jnut@value@nlc@reviewerc}{}
\newcommand*{\jnut@value@nlc@reviewerd}{}
\ifjnut@phd
    \newcommand*{\jnut@value@degree}{\jnut@cap@phd}
\else
    \ifjnut@master
        \newcommand*{\jnut@value@degree}{\jnut@cap@master}
    \else
       \ifjnut@bachelor
          \newcommand*{\jnut@value@degree}{\jnut@cap@bachelor}
       \else
          \ClassError{jnuthesis}{No degree was selected.}{}
       \fi
    \fi
\fi
\newcommand*{\jnut@value@coursename}{(本科课程设计课程名称)}
\newcommand*{\jnut@value@title}{(论文标题)}
\newcommand*{\jnut@value@titlea}{(论文长标题第一行)}
\newcommand*{\jnut@value@titleb}{(论文长标题第二行)}
\newcommand*{\jnut@value@author}{(作者姓名)}
\newcommand*{\jnut@value@telphone}{(作者电话号码)}
\newcommand*{\jnut@value@email}{(作者电子邮件)}
\newcommand*{\jnut@value@studentnum}{XXXXXXXX}
\newcommand*{\jnut@value@grade}{XXXX}
\newcommand*{\jnut@value@supervisor}{(导师姓名)}
\newcommand*{\jnut@value@supervisorpos}{(导师职称)}
\newcommand*{\jnut@value@supervisorb}{(第二行导师姓名)}
\newcommand*{\jnut@value@supervisorbpos}{(第二行导师职称)}
\newcommand*{\jnut@value@supervisortelphone}{(导师电话号码)}
\newcommand*{\jnut@value@major}{(作者专业)}
\newcommand*{\jnut@value@researchfield}{(作者研究方向)}
\newcommand*{\jnut@value@department}{(作者所属院系)}
\newcommand*{\jnut@value@directormembers}{(指导小组成员)}
\newcommand*{\jnut@value@degreedate}{(学位授予日期)}
\newcommand*{\jnut@value@bachelordegreeyear}{\number\year}
\newcommand*{\jnut@value@bachelordegreemonth}{\number\month}
\newcommand*{\jnut@value@institute}{江南大学}
\newcommand*{\jnut@value@submitdate}{xxxx年xx月xx日}
\newcommand*{\jnut@value@defenddate}{xxxx年xx月xx日}
\newcommand*{\jnut@value@date}{%
\CJKdigits{\number\year}年\CJKnumber{\number\month}月%
}
\ifjnut@phd
    \newcommand*{\jnut@value@en@degree}{\jnut@cap@en@phd}
\else
    \ifjnut@master
        \newcommand*{\jnut@value@en@degree}{\jnut@cap@en@master}
    \else
       \ifjnut@bachelor
          \newcommand*{\jnut@value@en@degree}{\jnut@cap@en@bachelor}
       \else
          \ClassError{jnuthesis}{No degree was selected.}{}
       \fi
    \fi
\fi
\newcommand*{\jnut@value@en@titlea}{(English Title of Thesis)}
\newcommand*{\jnut@value@en@titleb}{(English Title of Thesis)}
\newcommand*{\jnut@value@en@author}{(Author's Name)}
\newcommand*{\jnut@value@en@supervisor}{Professor (Supervisor's Name)}
\newcommand*{\jnut@value@en@major}{Author's Major}
\newcommand*{\jnut@value@en@department}{(Department's Name)}
\newcommand*{\jnut@value@en@institute}{Nanjing University}
\newcommand*{\jnut@value@en@date}{
\ifcase\month\or
January\or
February\or
March\or
April\or
May\or
June\or
July\or
August\or
September\or
October\or
November\or
December\fi
\number\day, \number\year%
}
\newcommand*{\jnut@value@abstract@titlea}{\jnut@value@title}
\newcommand*{\jnut@value@abstract@titleb}{}
\newcommand*{\jnut@value@abstract@keywords}{}
\newcommand*{\jnut@value@abstract@en@titlea}{\jnut@value@en@titlea}
\newcommand*{\jnut@value@abstract@en@titleb}{\jnut@value@en@titleb}
\newcommand*{\jnut@value@abstract@en@keywords}{}
\newcommand*{\jnut@zhfn@songti@win}{SimSun}
\newcommand*{\jnut@zhfn@heiti@win}{SimHei}
\newcommand*{\jnut@zhfn@kaishu@win}{KaiTi}
\newcommand*{\jnut@zhfn@fangsong@win}{FangSong}
\newcommand*{\jnut@enfn@main@win}{Times New Roman}
\newcommand*{\jnut@enfn@sans@win}{Arial}
\newcommand*{\jnut@enfn@mono@win}{Courier New}
\newcommand*{\jnut@zhfn@songti@linux}{SimSun}
\newcommand*{\jnut@zhfn@heiti@linux}{SimHei}
\newcommand*{\jnut@zhfn@kaishu@linux}{KaiTi}
\newcommand*{\jnut@zhfn@fangsong@linux}{FangSong}
\newcommand*{\jnut@enfn@main@linux}{Times New Roman}
\newcommand*{\jnut@enfn@sans@linux}{Arial}
\newcommand*{\jnut@enfn@mono@linux}{Courier New}
\newcommand*{\jnut@zhfn@songti@mac}{STSong}
\newcommand*{\jnut@zhfn@heiti@mac}{STHeiti}
\newcommand*{\jnut@zhfn@kaishu@mac}{STKaiti}
\newcommand*{\jnut@zhfn@fangsong@mac}{STFangsong}
\newcommand*{\jnut@enfn@main@mac}{Times}
\newcommand*{\jnut@enfn@sans@mac}{Helvetica}
\newcommand*{\jnut@enfn@mono@mac}{Courier}
\newcommand*{\jnut@zhfn@songti@adobe}{Adobe Song Std}
\newcommand*{\jnut@zhfn@heiti@adobe}{Adobe Heiti Std}
\newcommand*{\jnut@zhfn@kaishu@adobe}{Adobe Kaiti Std}
\newcommand*{\jnut@zhfn@fangsong@adobe}{Adobe Fangsong Std}
\newcommand*{\jnut@enfn@main@adobe}{Times New Roman}
\newcommand*{\jnut@enfn@sans@adobe}{Arial}
\newcommand*{\jnut@enfn@mono@adobe}{Courier New}
\newcommand*{\jnut@zhfn@songti@source}{Source Han Serif CN}
\newcommand*{\jnut@zhfn@heiti@source}{Source Han Sans CN}
\newcommand*{\jnut@zhfn@kaishu@source}{AR PL UKai CN}
\newcommand*{\jnut@zhfn@fangsong@source}{\jnut@zhfn@fangsong@adobe}
\newcommand*{\jnut@enfn@main@source}{\jnut@enfn@main@adobe}
\newcommand*{\jnut@enfn@sans@source}{\jnut@enfn@sans@adobe}
\newcommand*{\jnut@enfn@mono@source}{\jnut@enfn@mono@adobe}
\endinput
%%
%% End of file `jnuthesis.cfg'.

%    \end{macrocode}
%
% \subsection{字体设置}
%
% 首先根据文档选项选择正确的中文字体名称。
%    \begin{macrocode}
\ifjnut@adobefonts
  \newcommand*{\jnut@zhfn@songti}{\jnut@zhfn@songti@adobe}
  \newcommand*{\jnut@zhfn@heiti}{\jnut@zhfn@heiti@adobe}
  \newcommand*{\jnut@zhfn@kaishu}{\jnut@zhfn@kaishu@adobe}
  \newcommand*{\jnut@zhfn@fangsong}{\jnut@zhfn@fangsong@adobe}
  \newcommand*{\jnut@enfn@main}{\jnut@enfn@main@adobe}
  \newcommand*{\jnut@enfn@sans}{\jnut@enfn@sans@adobe}
  \newcommand*{\jnut@enfn@mono}{\jnut@enfn@mono@adobe}
\else
  \ifjnut@winfonts
      \newcommand*{\jnut@zhfn@songti}{\jnut@zhfn@songti@win}
      \newcommand*{\jnut@zhfn@heiti}{\jnut@zhfn@heiti@win}
      \newcommand*{\jnut@zhfn@kaishu}{\jnut@zhfn@kaishu@win}
      \newcommand*{\jnut@zhfn@fangsong}{\jnut@zhfn@fangsong@win}
      \newcommand*{\jnut@enfn@main}{\jnut@enfn@main@win}
      \newcommand*{\jnut@enfn@sans}{\jnut@enfn@sans@win}
      \newcommand*{\jnut@enfn@mono}{\jnut@enfn@mono@win}
  \else
    \ifjnut@linuxfonts
      \newcommand*{\jnut@zhfn@songti}{\jnut@zhfn@songti@linux}
      \newcommand*{\jnut@zhfn@heiti}{\jnut@zhfn@heiti@linux}
      \newcommand*{\jnut@zhfn@kaishu}{\jnut@zhfn@kaishu@linux}
      \newcommand*{\jnut@zhfn@fangsong}{\jnut@zhfn@fangsong@linux}
      \newcommand*{\jnut@enfn@main}{\jnut@enfn@main@linux}
      \newcommand*{\jnut@enfn@sans}{\jnut@enfn@sans@linux}
      \newcommand*{\jnut@enfn@mono}{\jnut@enfn@mono@linux}
    \else
       \ifjnut@macfonts
          \newcommand*{\jnut@zhfn@songti}{\jnut@zhfn@songti@mac}
          \newcommand*{\jnut@zhfn@heiti}{\jnut@zhfn@heiti@mac}
          \newcommand*{\jnut@zhfn@kaishu}{\jnut@zhfn@kaishu@mac}
          \newcommand*{\jnut@zhfn@fangsong}{\jnut@zhfn@fangsong@mac}
          \newcommand*{\jnut@enfn@main}{\jnut@enfn@main@mac}
          \newcommand*{\jnut@enfn@sans}{\jnut@enfn@sans@mac}
          \newcommand*{\jnut@enfn@mono}{\jnut@enfn@mono@mac}
       \else
         \ClassError{jnuthesis}{No fonts was selected.}{}
       \fi
    \fi
  \fi
\fi
%    \end{macrocode}
%
% 接下来定义文档使用的中文字体:
%    \begin{macrocode}
\setCJKfamilyfont{song}{\jnut@zhfn@songti}
\setCJKfamilyfont{hei}{\jnut@zhfn@heiti}
\setCJKfamilyfont{kai}{\jnut@zhfn@kaishu}
\setCJKfamilyfont{fangsong}{\jnut@zhfn@fangsong}
\setCJKmainfont[BoldFont={\jnut@zhfn@heiti},%
                ItalicFont={\jnut@zhfn@kaishu}]%
               {\jnut@zhfn@songti}
\setCJKsansfont{\jnut@zhfn@heiti}
\setCJKmonofont{\jnut@zhfn@fangsong}
%    \end{macrocode}
%
% 定义文档使用的英文字体。
%    \begin{macrocode}
\setmainfont{\jnut@enfn@main}
\setsansfont{\jnut@enfn@sans}
\setmonofont{\jnut@enfn@mono}
%    \end{macrocode}
%
% 定义中文字体选择命令。
%    \begin{macrocode}
\newcommand*{\songti}{\CJKfamily{song}}
\newcommand*{\heiti}{\CJKfamily{hei}}
\newcommand*{\kaishu}{\CJKfamily{kai}}
\newcommand*{\fangsong}{\CJKfamily{fangsong}}
%    \end{macrocode}
%
% \begin{table}
%   \centering
%   \subtable[科学出版社编写的《著译编辑手册》(1994年)中定义的中文字号大小]{
%     \label{table:fontsize:standard}
%     \noindent
%     \begin{tabular}{ccc}
%       \toprule
%       \textbf{字号}  &   \textbf{大小(pt)} & \textbf{大小(mm)}   \\
%       \midrule
%       七号  &    5.25  &    1.845 \\
%       六号  &    7.875 &    2.768 \\
%       小五  &    9     &    3.163 \\
%       五号  &    10.5  &    3.69  \\
%       小四  &    12    &    4.2175 \\
%       四号  &    13.75 &    4.83   \\
%       三号  &    15.75 &    5.53  \\
%       二号  &    21    &    7.38  \\
%       一号  &    27.5  &    9.48  \\
%       小初  &    36    &    12.65 \\
%       初号  &    42    &    14.76 \\
%       \bottomrule
%     \end{tabular}
%   }
%   \qquad
%   \subtable[Microsoft Word中定义的中文字号大小,其中$1$bp=$72.27/72$pt]{
%     \label{table:fontsize:word}
%     \noindent
%     \begin{tabular}{cccc}
%       \toprule
%       \textbf{字号}  & \textbf{大小(bp)} & \textbf{大小(mm)} & \textbf{大小(pt)}   \\
%       \midrule
%       初号  & 42     & 14.82 & 42.1575  \\
%       小初  & 36     & 12.70 & 36.135   \\
%       一号  & 26     & 9.17  & 26.0975  \\
%       小一  & 24     & 8.47  & 24.09    \\
%       二号  & 22     & 7.76  & 22.0825  \\
%       小二  & 18     & 6.35  & 18.0675  \\
%       三号  & 16     & 5.64  & 16.06    \\
%       小三  & 15     & 5.29  & 15.05625 \\
%       四号  & 14     & 4.94  & 14.0525  \\
%       小四  & 12     & 4.23  & 12.045   \\
%       五号  & 10.5   & 3.70  & 10.59375 \\
%       小五  & 9      & 3.18  & 9.03375  \\
%       六号  & 7.5    & 2.56  &            \\
%       小六  & 6.5    & 2.29  &            \\
%       七号  & 5.5    & 1.94  &            \\
%       八号  & 5      & 1.76  &            \\
%       \bottomrule
%     \end{tabular}
%   }
%   \caption{中文字号对应的字体大小}
%   \label{table:fontsize}
% \end{table}
%
% 下面定义中文字号对应的大小,其标准参见表\ref{table:fontsize:standard}和
% 表\ref{table:fontsize:word}。
%    \begin{macrocode}
\newcommand*{\jnut@fs@eight}{5.02} % 八号字 5bp
\newcommand*{\jnut@fs@eightskip}{6.02}
\newcommand*{\jnut@fs@seven}{5.52} % 七号字 5.5bp
\newcommand*{\jnut@fs@sevenskip}{6.62}
\newcommand*{\jnut@fs@ssix}{6.52} % 小六号 6.5bp
\newcommand*{\jnut@fs@ssixskip}{7.83}
\newcommand*{\jnut@fs@six}{7.53} % 六号字 7.5bp
\newcommand*{\jnut@fs@sixskip}{9.03}
\newcommand*{\jnut@fs@sfive}{9.03} % 小五号 9bp
\newcommand*{\jnut@fs@sfiveskip}{10.84}
\newcommand*{\jnut@fs@five}{10.54} % 五号 10bp
\newcommand*{\jnut@fs@fiveskip}{12.65}
\newcommand*{\jnut@fs@sfour}{12.05} % 小四号 12bp
\newcommand*{\jnut@fs@sfourskip}{14.45}
\newcommand*{\jnut@fs@four}{14.05} % 四号字 14bp
\newcommand*{\jnut@fs@fourskip}{16.86}
\newcommand*{\jnut@fs@sthree}{15.06} % 小三号 15bp
\newcommand*{\jnut@fs@sthreeskip}{18.07}
\newcommand*{\jnut@fs@three}{16.06} % 三号字 16bp
\newcommand*{\jnut@fs@threeskip}{19.27}
\newcommand*{\jnut@fs@stwo}{18.07} % 小二号 18bp
\newcommand*{\jnut@fs@stwoskip}{21.68}
\newcommand*{\jnut@fs@two}{22.08} % 二号字 22bp
\newcommand*{\jnut@fs@twoskip}{26.50}
\newcommand*{\jnut@fs@sone}{24.09} % 小一号 24bp
\newcommand*{\jnut@fs@soneskip}{28.91}
\newcommand*{\jnut@fs@one}{26.10} % 一号字 26bp
\newcommand*{\jnut@fs@oneskip}{31.32}
\newcommand*{\jnut@fs@szero}{36.14} % 小初号 36bp
\newcommand*{\jnut@fs@szeroskip}{43.36}
\newcommand*{\jnut@fs@zero}{42.16} % 初号字 42bp
\newcommand*{\jnut@fs@zeroskip}{50.59}
%    \end{macrocode}
%
% 声明不同字号下的数学字体大小。
%    \begin{macrocode}
\DeclareMathSizes{\jnut@fs@eight}
                 {\jnut@fs@eight}
                 {5}
                 {5}
\DeclareMathSizes{\jnut@fs@seven}
                 {\jnut@fs@seven}
                 {5}
                 {5}
\DeclareMathSizes{\jnut@fs@ssix}
                 {\jnut@fs@ssix}
                 {5}
                 {5}
\DeclareMathSizes{\jnut@fs@six}
                 {\jnut@fs@six}
                 {5}
                 {5}
\DeclareMathSizes{\jnut@fs@sfive}
                 {\jnut@fs@sfive}
                 {6}
                 {5}
\DeclareMathSizes{\jnut@fs@five}
                 {\jnut@fs@five}
                 {7}
                 {5}
\DeclareMathSizes{\jnut@fs@sfour}
                 {\jnut@fs@sfour}
                 {8}
                 {6}
\DeclareMathSizes{\jnut@fs@four}
                 {\jnut@fs@four}
                 {\jnut@fs@five}
                 {\jnut@fs@six}
\DeclareMathSizes{\jnut@fs@sthree}
                 {\jnut@fs@sthree}
                 {\jnut@fs@sfour}
                 {\jnut@fs@sfive}
\DeclareMathSizes{\jnut@fs@three}
                 {\jnut@fs@three}
                 {\jnut@fs@four}
                 {\jnut@fs@five}
\DeclareMathSizes{\jnut@fs@stwo}
                 {\jnut@fs@stwo}
                 {\jnut@fs@sthree}
                 {\jnut@fs@sfour}
\DeclareMathSizes{\jnut@fs@two}
                 {\jnut@fs@two}
                 {\jnut@fs@three}
                 {\jnut@fs@four}
\DeclareMathSizes{\jnut@fs@sone}
                 {\jnut@fs@sone}
                 {\jnut@fs@stwo}
                 {\jnut@fs@sthree}
\DeclareMathSizes{\jnut@fs@one}
                 {\jnut@fs@one}
                 {\jnut@fs@two}
                 {\jnut@fs@three}
\DeclareMathSizes{\jnut@fs@szero}
                 {\jnut@fs@szero}
                 {\jnut@fs@sone}
                 {\jnut@fs@stwo}
\DeclareMathSizes{\jnut@fs@zero}
                 {\jnut@fs@zero}
                 {\jnut@fs@one}
                 {\jnut@fs@two}
%    \end{macrocode}
%
% 定义字号选择命令。字号前面加负号表示采用对应的小体字号,例如|\zihao{-3}|表示小
% 三号。
% \begin{note}
% 为了让|\zihao{-0}|能正确表示小初号,在判断参数正负的时候把参数后面再接一个字符`1',从
% 而将``-0''变为``-01'',而``-01''转换为数字为$-1$,故可正确判断其是否小于零。
% \end{note}
%    \begin{macrocode}
\def\jnut@zihao{}
\DeclareRobustCommand*{\zihao}[1]{%
  \def\jnut@zihao{#1}%
  \ifnum #11<0%
    \@tempcnta=-#1
    \ifcase\@tempcnta%
        \fontsize\jnut@fs@szero\jnut@fs@szeroskip%
    \or \fontsize\jnut@fs@sone\jnut@fs@soneskip%
    \or \fontsize\jnut@fs@stwo\jnut@fs@stwoskip%
    \or \fontsize\jnut@fs@sthree\jnut@fs@sthreeskip%
    \or \fontsize\jnut@fs@sfour\jnut@fs@sfourskip%
    \or \fontsize\jnut@fs@sfive\jnut@fs@sfiveskip%
    \or \fontsize\jnut@fs@ssix\jnut@fs@ssixskip%
    \else \ClassError{jnuthesis}{%
            Undefined Chinese font size in command \protect\zihao}{%
            The old font size is used if you continue.}%
    \fi%
  \else%
    \@tempcnta=#1
    \ifcase\@tempcnta%
        \fontsize\jnut@fs@zero\jnut@fs@zeroskip%
    \or \fontsize\jnut@fs@one\jnut@fs@oneskip%
    \or \fontsize\jnut@fs@two\jnut@fs@twoskip%
    \or \fontsize\jnut@fs@three\jnut@fs@threeskip%
    \or \fontsize\jnut@fs@four\jnut@fs@fourskip%
    \or \fontsize\jnut@fs@five\jnut@fs@fiveskip%
    \or \fontsize\jnut@fs@six\jnut@fs@sixskip%
    \or \fontsize\jnut@fs@seven\jnut@fs@sevenskip%
    \or \fontsize\jnut@fs@eight\jnut@fs@eightskip%
    \else \ClassError{jnuthesis}{%
            Undefined Chinese font size in command \protect\zihao}{%
            The old font size is used if you continue.}%
    \fi%
  \fi%
  \selectfont\ignorespaces}
%    \end{macrocode}
%
% 修改常用字体大小选择命令。
%
%    \begin{macrocode}
\renewcommand{\tiny}{% 小六号 6.5bp
  \@setfontsize\tiny{\jnut@fs@ssix}{\jnut@fs@ssixskip}}
\renewcommand{\scriptsize}{% 六号字 7.5bp
  \@setfontsize\scriptsize{\jnut@fs@six}{\jnut@fs@sixskip}}
\renewcommand{\footnotesize}{% 小五号 9bp
  \@setfontsize\footnotesize{\jnut@fs@sfive}{\jnut@fs@sfiveskip}%
  \abovedisplayskip 6\p@ \@plus2\p@ \@minus4\p@
  \abovedisplayshortskip \z@ \@plus\p@
  \belowdisplayshortskip 3\p@ \@plus\p@ \@minus2\p@
  \def\@listi{\leftmargin\leftmargini
    \topsep 3\p@ \@plus\p@ \@minus\p@
    \parsep 2\p@ \@plus\p@ \@minus\p@
    \itemsep \parsep}%
  \belowdisplayskip \abovedisplayskip}
\renewcommand{\small}{% 五号 10bp
  \@setfontsize\small{\jnut@fs@five}{\jnut@fs@fiveskip}%
  \abovedisplayskip 8.5\p@ \@plus3\p@ \@minus4\p@
  \abovedisplayshortskip \z@ \@plus2\p@
  \belowdisplayshortskip 4\p@ \@plus2\p@ \@minus2\p@
  \def\@listi{\leftmargin\leftmargini
    \topsep 4\p@ \@plus2\p@ \@minus2\p@
    \parsep 2\p@ \@plus\p@ \@minus\p@
    \itemsep \parsep}%
  \belowdisplayskip \abovedisplayskip}
\renewcommand{\normalsize}{% 小四号 12bp
  \@setfontsize\normalsize{\jnut@fs@sfour}{\jnut@fs@sfourskip}%
  \abovedisplayskip 10\p@ \@plus2\p@ \@minus5\p@
  \abovedisplayshortskip \z@ \@plus3\p@
  \belowdisplayshortskip 6\p@ \@plus3\p@ \@minus3\p@
  \belowdisplayskip \abovedisplayskip
  \let\@listi\@listI}
\renewcommand{\large}{% 小三号 15bp
  \@setfontsize\large{\jnut@fs@sthree}{\jnut@fs@sthreeskip}}
\renewcommand{\Large}{% 小二号 18bp
  \@setfontsize\Large{\jnut@fs@stwo}{\jnut@fs@stwoskip}}
\renewcommand{\LARGE}{% 小一号 24bp
  \@setfontsize\LARGE{\jnut@fs@sone}{\jnut@fs@soneskip}}
\renewcommand{\huge}{% 一号 26bp
  \@setfontsize\huge{\jnut@fs@one}{\jnut@fs@oneskip}}
\renewcommand{\Huge}{% 小初号 36bp
  \@setfontsize\Huge{\jnut@fs@szero}{\jnut@fs@szeroskip}}
%    \end{macrocode}
%
% 定义中文字距修改命令,直接修改\cs{CJKglue}即可。
%    \begin{macrocode}
\newcommand*{\ziju}[1]{\renewcommand*{\CJKglue}{\hskip {#1}}}
%    \end{macrocode}
%
% 修改\cs{textsc}命令,使其可在中文编码下正常工作。
%    \begin{macrocode}
\renewcommand{\textsc}[1]{{\usefont{OT1}{cmr}{m}{sc}{#1}}}
%    \end{macrocode}
%
% \subsection{数学公式和定理}
%
% 按照\std{CY/T 35-2001}规范的要求,重定义公式、图、表的编号格式。例如:
% \begin{itemize}
% \item 图\dashnumber{1}{2}
% \item 表\dashnumber{2}{3}
% \item 附注 1)
% \item 文献[4]
% \item 式(\dashnumber{6}{3})
% \end{itemize}
% 子图和子表的应用序号外加小括号,例如
% \begin{itemize}
% \item 图\dashnumber{1}{2}(a)
% \item 表\dashnumber{2}{3}(b)
% \end{itemize}
%    \begin{macrocode}
\newcommand{\dashnumber}[2]%
  {{#1}\kern.07em\rule[.5ex]{.4em}{.15ex}\kern.07em{#2}}
\renewcommand*{\thefigure}{\dashnumber{\thechapter}{\arabic{figure}}}
\renewcommand*{\thetable}{\dashnumber{\thechapter}{\arabic{table}}}
\renewcommand*{\theequation}{\dashnumber{\thechapter}{\arabic{equation}}}
\renewcommand*{\thesubfigure}{(\alph{subfigure})}
\renewcommand*{\thesubtable}{(\alph{subtable})}
%    \end{macrocode}
%
% 定义常用的数学定理环境及其样式。
%    \begin{macrocode}
\newtheoremstyle{plain}% name
                {1em}%      Space above, empty = `usual value'
                {1em}%      Space below
                {\normalfont}% Body font
                {}%         Indent amount
                {\normalfont\bfseries}% Thm head font
                {}%         Punctuation after thm head
                {1em}%      Space after thm head: \newline = linebreak
                {}%         Thm head spec
\newtheorem{definition}{\jnut@cap@definition}[chapter]
\newtheorem{notation}[definition]{\jnut@cap@notation}
\newtheorem{theorem}{\jnut@cap@theorem}[chapter]
\newtheorem{lemma}[theorem]{\jnut@cap@lemma}
\newtheorem{corollary}[theorem]{\jnut@cap@corollary}
\newtheorem{proposition}[theorem]{\jnut@cap@proposition}
\newtheorem{fact}[theorem]{\jnut@cap@fact}
\newtheorem{assumption}[theorem]{\jnut@cap@assumption}
\newtheorem{conjecture}[theorem]{\jnut@cap@conjecture}
\newtheorem{hypothesis}{\jnut@cap@hypothesis}[chapter]
\newtheorem{axiom}{\jnut@cap@axiom}[chapter]
\newtheorem{postulate}{\jnut@cap@postulate}[chapter]
\newtheorem{principle}{\jnut@cap@principle}[chapter]
\newtheorem{problem}{\jnut@cap@problem}[chapter]
\newtheorem{exercise}{\jnut@cap@exercise}[chapter]
\newtheorem{example}{\jnut@cap@example}[chapter]
\newtheorem{remark}{\jnut@cap@remark}[chapter]

\renewenvironment{proof}[1][\jnut@cap@proof]{\par
  \pushQED{\qed}%
  \normalfont \topsep6\p@\@plus6\p@\relax
  \trivlist
  \item[\hskip\labelsep\textbf{#1}\@addpunct{:}]\ignorespaces
}{\popQED\endtrivlist\@endpefalse}

\newenvironment{solution}[1][\jnut@cap@solution]{\par
  \normalfont \topsep6\p@\@plus6\p@\relax
  \trivlist
  \item[\hskip\labelsep\textbf{#1}\@addpunct{:}]\ignorespaces
}{\endtrivlist\@endpefalse}

\newtheorem{algorithm}{\jnut@cap@algorithm}[chapter]
%    \end{macrocode}
%
% 修改上面定义的各定理环境的编号样式:
%    \begin{macrocode}
\renewcommand*{\thedefinition}{\dashnumber{\thechapter}{\arabic{definition}}}
\renewcommand*{\thetheorem}{\dashnumber{\thechapter}{\arabic{theorem}}}
\renewcommand*{\theaxiom}{\dashnumber{\thechapter}{\arabic{axiom}}}
\renewcommand*{\theproblem}{\dashnumber{\thechapter}{\arabic{problem}}}
\renewcommand*{\theexercise}{\dashnumber{\thechapter}{\arabic{exercise}}}
\renewcommand*{\theexample}{\dashnumber{\thechapter}{\arabic{example}}}
\renewcommand*{\theremark}{\dashnumber{\thechapter}{\arabic{remark}}}
%    \end{macrocode}
%
% \subsection{设置浮动环境格式}
%
% 默认情况下,{\LaTeX}要求每页的文字至少占据$20\%$,否则该页就只单独放置一个浮动环境。而
% 这通常不是我们想要的。我们将这个要求降低到$5\%$。
%    \begin{macrocode}
\renewcommand*{\textfraction}{0.05}
%    \end{macrocode}
% 有时如果多个浮动环境连续放在一起,{\LaTeX}会将它们分在几个不同页,即使它们可在同一页放
% 得下。我们可以通过修改\cs{topfraction}和\cs{bottomfraction}分别设置顶端和底端的浮动
% 环境的最大比例。
%    \begin{macrocode}
\renewcommand*{\topfraction}{0.9}
\renewcommand*{\bottomfraction}{0.8}
%    \end{macrocode}
% 有时{\LaTeX}会把一个浮动环境单独放在一页,我们要求这个环境至少要占据$85\%$才能单独放在
% 一页。
% \begin{note}
% \cs{floatpagefraction}的数值必须小于\cs{topfraction}。
% \end{note}
%    \begin{macrocode}
\renewcommand*{\floatpagefraction}{0.85}
%    \end{macrocode}
%
% \subsection{中文标题名称}
%
% 设置常见的中文标题名称。
%    \begin{macrocode}
\newcommand*{\abstractname}{\jnut@cap@abstractname}
\renewcommand*{\contentsname}{\jnut@cap@contentsname}
\renewcommand*{\listfigurename}{\jnut@cap@listfigurename}
\renewcommand*{\listtablename}{\jnut@cap@listtablename}
\newcommand*{\listsymbolname}{\jnut@cap@listsymbolname}
\newcommand*{\listequationname}{\jnut@cap@listequationname}
\renewcommand*{\glossaryname}{\jnut@cap@glossaryname}
\renewcommand*{\indexname}{\jnut@cap@indexname}
\newcommand*{\equationname}{\jnut@cap@equationname}
\renewcommand*{\bibname}{\jnut@cap@bibname}
\renewcommand*{\figurename}{\jnut@cap@figurename}
\renewcommand*{\tablename}{\jnut@cap@tablename}
\renewcommand*{\chaptername}{\jnut@cap@chaptername}
\renewcommand*{\appendixname}{\jnut@cap@appendixname}
%    \end{macrocode}
%
% \subsection{中文标题格式}
%
% 设置章节格式如下:
% \begin{description}
% \item[零级节标题] 命令为\cs{chapter},格式为三号黑体,居中排列,段前空4ex,段后空3ex;
% \item[一级节标题] 命令为\cs{section},格式为四号加粗宋体,左排列,段前空3.5ex,段后空2.3ex;
% \item[二级节标题] 命令为\cs{subsection},格式为小四号加粗宋体,左排列,段前空3.0ex,段后空1.5ex;
% \item[三级节标题] 命令为\cs{subsubsection},格式为小三号黑体,左排列,段前空2.5ex,段后空1.5ex;
% \item[四级节标题] 命令为\cs{paragraph},格式为四号黑体,左排列,段前空2.0ex,段后空1ex;
% \item[五级节标题] 命令为\cs{subparagraph},格式为小四号黑体,左排列,段前空1.5ex,段后空1ex;
% \end{description}
%
% 使用|titlesec|宏包提供的\cs{titleformat}和\cs{titlespacing}命令可以方便地设置标题
% 的样式:
%    \begin{macrocode}
\titleformat{\chapter}[hang]
            {\centering\zihao{3}\bfseries}
            {\chaptertitlename}{1em}{}
\titlespacing{\chapter}
             {0pt}
             {*4}
             {*3}
\titleformat{\section}[hang]
            {\zihao{4}\songti\bfseries}
            {\thesection}{1em}{}
\titlespacing{\section}
             {0pt}
             {*3.5}
             {*2.3}
\titleformat{\subsection}[hang]
            {\zihao{-4}\songti\bfseries}
            {\thesubsection}{1em}{}
\titlespacing{\subsection}
             {0pt}
             {*3}
             {*1.5}
%% NOTE: 以下格式江南大学未使用。
\titleformat{\subsubsection}[hang]
            {\zihao{-3}\bfseries}
            {\thesubsubsection}{1em}{}
\titlespacing{\subsubsection}
             {0pt}
             {*2.5}
             {*1.5}
\titleformat{\paragraph}[hang]
            {\zihao{4}\bfseries}
            {}{0em}{}
\titlespacing{\paragraph}
             {0pt}
             {*2}
             {*1}
\titleformat{\subparagraph}[hang]
            {\zihao{-4}\bfseries}
            {}{0em}{}
\titlespacing{\subparagraph}
             {0pt}
             {*1.5}
             {*1}
%    \end{macrocode}
%
% 设置章节标题编号最多到第4层(即\cs{subsubsection}),超过第四层的章节不再自动编号。
%    \begin{macrocode}
\setcounter{secnumdepth}{4}
%    \end{macrocode}
%
% 修改章节编号的样式:
%    \begin{macrocode}
\renewcommand{\thechapter}{\arabic{chapter}}
\renewcommand{\thesection}{\thechapter\thinspace.\thinspace\arabic{section}}
\renewcommand{\thesubsection}{\thesection\thinspace.\thinspace\arabic{subsection}}
\renewcommand{\thesubsubsection}{\thesubsection\thinspace.\thinspace\arabic{subsubsection}}
%    \end{macrocode}
%
% \subsection{浮动环境}
%
% 设置浮动环境标题的字体大小。根据学位论文格式要求,插图和表格标题字体需要比正文字体略小。
%    \begin{macrocode}
\captionsetup{font=small}
%    \end{macrocode}
%
% 根据学位论文格式要求,表格的标题必须位于表格上方,插图的标题必须位于插图下方。
%    \begin{macrocode}
\captionsetup[table]{position=above}
\captionsetup[figure]{position=below}
\floatstyle{plaintop}
\restylefloat{table}
%    \end{macrocode}
%
% \subsection{页幅设置}
%
% 正文统一用小四号字,间距为固定值20pt。\cs{linestrech}的值为$1$时为单倍行距, $1.2$时是
% 一倍半行距, 而为$1.6$时是双倍行距。其实不同尺寸的字体行间距都不相同,而是成比例关系。这
% 个20pt是对正文主要字体来说的。
%
% 在{\TeX}中基本的行间距是\cs{baselineskip}, 对于12pt的字体,这个值等于14.5pt,
% 而真正的行间距是\cs{baselineskip}$\times$\cs{baselinestretch},
% \cs{baselinestretch}默认为$1$, 但我们可以重新设置它的值,如
% |\renewcommand{\baselinestretch}{1.38}|就得到真正的行间距为14.5pt*1.38≈20pt。
% 而这样定义之后,对不同尺寸的字体都会按同样的比例因子1.38放大行间距,使得全文排
% 版能协调一致。
%    \begin{macrocode}
\renewcommand*{\baselinestretch}{1.38}
%    \end{macrocode}
%
% 修改|tabular|环境,设置表格中的行间距为正文行间距。
%    \begin{macrocode}
\let\convertergy@oldtabular\tabular
\let\convertergy@endoldtabular\endtabular
\renewenvironment{tabular}%
{\bgroup%
\renewcommand{\arraystretch}{1.38}%
\convertergy@oldtabular}%
{\convertergy@endoldtabular\egroup}
%    \end{macrocode}
%
% 文章用A4纸标准大小的白纸打印,页眉页脚:1cm,页边距上下:2cm,左右:2.5cm。
%    \begin{macrocode}
\geometry{headheight=1cm,headsep=3mm,footskip=12mm}
\geometry{top=2cm,bottom=2cm,left=2.5cm,right=2.5cm}
%    \end{macrocode}
%
% 设置每一段的首行缩进两个汉字。
% \begin{note}
% 直接将|parindent|设置为|2em|并不能正确地设置段首缩进为恰好两个中文字符。因此我们采用下面的
% 网页提供办法:\\
% \url{https://github.com/ElegantLaTeX/ElegantLaTeX/blob/master/CJKindent.md}
% \end{note}
%
% 首先,我们需要计算出当前字符的宽度:
%    \begin{macrocode}
\def\jnut@CJK@charwidth{\hskip \f@size \p@}
%    \end{macrocode}
% 接下来,我们需要考虑到字符间距,计算出当前相邻两字符中心的距离:
%    \begin{macrocode}
\newdimen\jnut@CJK@chardimen
\settowidth\jnut@CJK@chardimen{\jnut@CJK@charwidth\CJKglue}
%    \end{macrocode}
% 最后,我们设置段首缩进长度:
%    \begin{macrocode}
\newcommand{\jnut@CJK@setfontspace}{%
\settowidth\jnut@CJK@chardimen{\jnut@CJK@charwidth\CJKglue}%
\ifdim\parindent=0pt\relax\else\parindent2\jnut@CJK@chardimen\fi%
}
\renewcommand*{\indent}{\jnut@CJK@setfontspace\parindent2\jnut@CJK@chardimen}
\AtBeginDocument{\indent}
%    \end{macrocode}
%
% \subsection{页眉页脚}
%
% 我们使用|fancyhdr|宏包来设置页眉页脚。|fancyhdr|宏包提供了一个|fancy|页面风格,
% 在该风格下,章节的起始页(即包含“第XX章”标题的页面)的页眉页脚将使用|plain|风
% 格,而章节的后继页面的页眉页脚将使用|fancy|风格的默认定义或用户通过
% \cs{fancyhead}或\cs{fancyfoot}命令定义的样式。
%
% 首先,我们需要修改|plain|风格的页眉页脚,将其默认的分割线去掉。
%    \begin{macrocode}
\fancypagestyle{plain}{%
   \renewcommand{\headrulewidth}{0pt} % and the header line
   \renewcommand{\footrulewidth}{0pt} % and the footer line
}
%    \end{macrocode}
%
% 接下来我们按照如下规则修改|fancy|风格的页眉页脚设置,注意学位论文始终是双面打印的:
% \begin{itemize}
%    \item 令偶数页的页眉如下:
%      \begin{itemize}
%      \item 若当前不位于|mainmatter|中,则显示当前节(section)的标题;
%      \item 若当前位于|mainmatter|中,则显示“江南大学XX学位论文”。
%      \end{itemize}
%    \item 令奇数页的页眉如下:
%      \begin{itemize}
%      \item 无条件显示当前节(section)的编号和标题。
%      \end{itemize}
%    \item 无论奇偶页,页眉下都有一条分割线;
%    \item 无论奇偶页,页脚都为页码,页脚上都无分割线。
% \end{itemize}
%
% 设置|fancy|风格下的页脚,令页脚分割线宽度为$0$:
%    \begin{macrocode}
\renewcommand{\footrulewidth}{0pt}
%    \end{macrocode}
%
% 设置|fancy|风格的页眉,令偶数页按如上要求显示,奇数页显示当前节
% 信息;但若当前页面尚未开始本章的第一节(即\cs{rightmark}为空),则奇数页左上角也
% 显示当前章信息(即\cs{leftmark})。
%    \begin{macrocode}
\fancyhead{}
\fancyhead[C]{%
\ifthenelse{\isodd{\value{page}}}%
            {\leftmark}%
            {\if@mainmatter\jnut@value@institute\jnut@cap@cover@thesis\else\rightmark\fi}%
}
%    \end{macrocode}
%
% 设置全局使用|fancy|风格。
%    \begin{macrocode}
\pagestyle{fancy}
%    \end{macrocode}
%
% 重新定义|chaptermark|,让其显示当前章信息和当前节信息。注意下面的重定义必须放
% 在第一次调用|\pagestyle{fancy}|之后,因为第一次调用该命令会设置\cs{chaptermark}。
%    \begin{macrocode}
\renewcommand{\chaptermark}[1]{\markboth{%
 \if@mainmatter\chaptertitlename\hspace{1em}\fi{#1}%
}{}}
%    \end{macrocode}
%
% 重新定义|sectionmark|,让其显示当前节信息。注意下面的重定义必须放在第一次调用
% |\pagestyle{fancy}|之后,因为第一次调用该命令会设置\cs{sectionmark}。
%    \begin{macrocode}
\renewcommand{\sectionmark}[1]{\markright{%
 \if@mainmatter\thesection\hspace{1em}\fi{#1}%
}}
%    \end{macrocode}
%
% 另一个麻烦的问题是:默认的|fancy|风格会在每一章最后的空白页(由于是双面打印)
% 也加上页眉页脚,江南大学也如此要求,但我们不希望包含正文部分之前的空白页。
% 解决方法是修改{\LaTeX}内部的\cs{cleardoublepage}与\cs{mainmatter}命令的定义如下:
%    \begin{macrocode}
\def\jnut@origcleardoublepage\cleardoublepage
\def\cleardoublepage{\clearpage\if@twoside \ifodd\c@page\else
  \hbox{}\thispagestyle{empty}\newpage\if@twocolumn\hbox{}\newpage\fi\fi\fi}

\def\jnut@origmainmatter\mainmatter
\def\mainmatter{\clearpage\jnut@origmainmatter\def\cleardoublepage\jnut@origcleardoublepage}
%    \end{macrocode}
%
% \subsection{列表环境}
%
% {\LaTeX}默认的列表:|enumerate|,|itemize|,和|description|都不符合中文习惯。
% 符合中文习惯的列表需要满足:
% \begin{enumerate}
% \item 列表标签要与正文的左边界对齐;
% \item 列表文本左侧要和左边界对齐;
% \item 列表项的间距应当等于正文中的段落间距,通常为$0$;
% \item 列表文本的右侧与正文的右边界对齐。
% \end{enumerate}
% 因此需要重新设置默认的列表的格式。
%    \begin{macrocode}
\setlist{%
  topsep=0.3em,             % 列表顶端的垂直空白
  partopsep=0pt,            % 列表环境前面紧接着一个空白行时其顶端的额外垂直空白
  itemsep=0ex plus 0.1ex,   % 列表项之间的额外垂直空白
  parsep=0pt,               % 列表项内的段落之间的垂直空白
  leftmargin=1.5em,         % 环境的左边界和列表之间的水平距离
  rightmargin=0em,          % 环境的右边界和列表之间的水平距离
  labelsep=0.5em,           % 包含标签的盒子与列表项的第一行文本之间的间隔
  labelwidth=2em,           % 包含标签的盒子的正常宽度;若实际宽度更宽,则使用实际宽度。
}
%    \end{macrocode}
%
% 设置无序列表的标签符号。
%    \begin{macrocode}
\setlist[itemize,1]{label=$\medbullet$}
\setlist[itemize,2]{label=$\blacksquare$}
\setlist[itemize,3]{label=$\Diamondblack$}
%    \end{macrocode}
%
% \subsection{引用}
%
% 默认的引用环境|quote|和|quotation|都不符合中文习惯,我们将其重新定义如下:
%    \begin{macrocode}
\renewenvironment{quote}%
                 {\list{}{\leftmargin=4em\rightmargin=4em}\item[]}%
                 {\endlist}
\renewenvironment{quotation}%
                 {\list{}{\leftmargin=4em\rightmargin=4em}\item[]}%
                 {\endlist}
%    \end{macrocode}
%
% \subsection{目录}
%
% 前置部分的封面在后面详细介绍,首先看目录。其具体要求为:目录页由论文的章、节、条、项、
% 附录等的序号、名称和页码组成,另页排在序之后。目录页标注学位论文的前三级目录。
% 标题统一用“目录”,黑体3字号字居中,段前、段后间距为1行; 各章(一级目录)名称用
% 黑体5号字,段前间距为0.5行,段后间距为0行; 其它(二、三级目录)用宋体5号字,
% 段前、段后间距为0行。
%
% \begin{macro}{\nchapter}
% 用于产生没有编号但在目录中列出的章。
%    \begin{macrocode}
\newcommand\nchapter[1]{%
  \if@mainmatter%
    \@mainmatterfalse%
    \chapter{#1}%
    \@mainmattertrue%
  \else
    \chapter{#1}%
  \fi
}
%    \end{macrocode}
% \end{macro}
%
% \begin{macro}{\@dottedtocline}
% 改变缺省的目录中的点线为中文习惯。
%    \begin{macrocode}
\def\@dottedtocline#1#2#3#4#5{%
  \ifnum #1>\c@tocdepth \else
    \vskip \z@ \@plus.2\p@
    {\leftskip #2\relax \rightskip \@tocrmarg \parfillskip -\rightskip
     \parindent #2\relax\@afterindenttrue
     \interlinepenalty\@M
     \leavevmode
     \@tempdima #3\relax
     \advance\leftskip \@tempdima \null\nobreak\hskip -\leftskip
     {#4}\nobreak
     \leaders\hbox{$\m@th\mkern 1.5mu\cdot\mkern 1.5mu$}\hfill
     \nobreak
     \hb@xt@\@pnumwidth{\hfil\normalfont \normalcolor #5}%
     \par}%
  \fi}
%    \end{macrocode}
% \end{macro}
%
% \begin{macro}{\l@part}
% 改变缺省的目录中的点线为中文习惯。
%    \begin{macrocode}
\renewcommand*{\l@part}[2]{%
  \ifnum \c@tocdepth >-2\relax
    \addpenalty{-\@highpenalty}%
    \addvspace{2.25em \@plus\p@}%
    \setlength\@tempdima{3em}%
    \begingroup
      \parindent \z@ \rightskip \@pnumwidth
      \parfillskip -\@pnumwidth
      {\leavevmode
       \large \bfseries #1
       \leaders\hbox{$\m@th\mkern 1.5mu\cdot\mkern 1.5mu$}
       \hfil \hb@xt@\@pnumwidth{\hss #2}}\par
       \nobreak
         \global\@nobreaktrue
         \everypar{\global\@nobreakfalse\everypar{}}%
    \endgroup
  \fi}
%    \end{macrocode}
% \end{macro}
%
% \begin{macro}{\l@chapter}
% 改变缺省的目录中的点线为中文习惯。
%    \begin{macrocode}
\renewcommand*{\l@chapter}[2]{%
  \ifnum \c@tocdepth >\m@ne
    \addpenalty{-\@highpenalty}%
    \vskip 1.0em \@plus\p@
    \setlength\@tempdima{1.5em}%
    \begingroup
      \parindent \z@ \rightskip \@pnumwidth
      \parfillskip -\@pnumwidth
      \leavevmode \bfseries
      \advance\leftskip\@tempdima
      \hskip -\leftskip
      #1\nobreak
      \leaders\hbox{$\m@th\mkern 1.5mu\cdot\mkern 1.5mu$}
      \hfil \nobreak\hb@xt@\@pnumwidth{\hss #2}\par
      \penalty\@highpenalty
    \endgroup
  \fi}
%    \end{macrocode}
% \end{macro}
%
% \begin{macro}{\tableofcontents}
% 修改\cs{tableofcontents}命令用于生成目录页,并将目录页本身也被加入目录中。
%    \begin{macrocode}
\renewcommand*{\tableofcontents}{%
    \if@twocolumn
      \@restonecoltrue\onecolumn
    \else
      \@restonecolfalse
    \fi
    \nchapter{\contentsname}%
    \@mkboth{\MakeUppercase\contentsname}{\MakeUppercase\contentsname}%
    \@starttoc{toc}%
    \if@restonecol\twocolumn\fi
}
%    \end{macrocode}
% \end{macro}
%
% \begin{macro}{\listoftables}
% 修改\cs{listoftables}命令,使得附表清单被加入目录中。
%    \begin{macrocode}
\renewcommand*{\listoftables}{%
    \if@twocolumn
      \@restonecoltrue\onecolumn
    \else
      \@restonecolfalse
    \fi
    \nchapter{\listtablename}%
    \@mkboth{\MakeUppercase\listtablename}{\MakeUppercase\listtablename}%
    \@starttoc{lot}%
    \if@restonecol\twocolumn\fi
}
%    \end{macrocode}
% \end{macro}
%
% \begin{macro}{\listoffigures}
% 修改\cs{listoffigures}命令,使得插图清单被加入目录中。
%    \begin{macrocode}
\renewcommand*{\listoffigures}{%
    \if@twocolumn
      \@restonecoltrue\onecolumn
    \else
      \@restonecolfalse
    \fi
    \nchapter{\listfigurename}%
    \@mkboth{\MakeUppercase\listfigurename}{\MakeUppercase\listfigurename}%
    \@starttoc{lof}%
    \if@restonecol\twocolumn\fi
}
%    \end{macrocode}
% \end{macro}
%
% \subsection{参考文献}
%
% \begin{environment}{thebibliography}
% 修改|thebibliography|环境用于在目录中加入参考文献页。
%    \begin{macrocode}
\renewenvironment{thebibliography}[1]
     {\nchapter{\bibname}%
      \@mkboth{\MakeUppercase\bibname}{\MakeUppercase\bibname}%
      \list{\@biblabel{\@arabic\c@enumiv}}%
           {\settowidth\labelwidth{\@biblabel{#1}}%
            \leftmargin\labelwidth
            \advance\leftmargin\labelsep
            \@openbib@code
            \usecounter{enumiv}%
            \let\p@enumiv\@empty
            \renewcommand\theenumiv{\@arabic\c@enumiv}}%
      \sloppy
      \clubpenalty4000
      \@clubpenalty \clubpenalty
      \widowpenalty4000%
      \sfcode`\.\@m}
     {\def\@noitemerr
       {\@latex@warning{Empty `thebibliography' environment}}%
      \endlist}
%    \end{macrocode}
% \end{environment}
%
% 使用|gbt7714-2005.bst|作为参考文献样式。
%    \begin{macrocode}
\bibliographystyle{gbt7714-2005}
%    \end{macrocode}
%
% 使用符合\std{GB/T 7714-2005}规范的参考文献引用样式。
%    \begin{macrocode}
\setcitestyle{super,square}
%    \end{macrocode}
%
% 修改|natbib|内部的\cs{NAT@citesuper}命令,使其生成的上标引用编号可以正确地把
% \cs{cite}命令的可选参数(通常是引文页码)也作为上标放在引文编号方框之后。
%    \begin{macrocode}
\renewcommand\NAT@citesuper[3]{%
\ifNAT@swa%
  \if*#2*\else#2\NAT@spacechar\fi%
  \unskip\kern\p@\textsuperscript{\NAT@@open#1\NAT@@close#3}%
\else #1\fi\endgroup%
}
%    \end{macrocode}
%
% 重新定义\cs{ref}命令,使其前面自动加一个``\textasciitilde''。因为|hyperref|宏包会通
% 过\cs{AtBeginDocument}修改\cs{ref}的定义,因此我们对\cs{ref}的修改也必须使用
% \cs{AtBeginDocument}命令进行。同时我们需要修改\cs{eqref},使其括号前后不出现空隙。
%    \begin{macrocode}
\AtBeginDocument{%
\let\oldref\ref%
\renewcommand*{\ref}[1]{\thinspace\oldref{#1}}%
\renewcommand*{\eqref}[1]{(\oldref{#1})}
}
%    \end{macrocode}
%
% \subsection{脚注}
%
% 使用|footmisc|宏包和|pifont|宏包设置符合\std{GB/T 7713.1-2006}规范的脚注样式。注意,
% 由于|pifont|宏包提供的特殊符号的限制,一页之中最多只能有$10$个脚注。
%    \begin{macrocode}
\DefineFNsymbols*{circlednumber}[text]{%
   {\ding{192}} %
   {\ding{193}} %
   {\ding{194}} %
   {\ding{195}} %
   {\ding{196}} %
   {\ding{197}} %
   {\ding{198}} %
   {\ding{199}} %
   {\ding{200}} %
   {\ding{201}} %
}%
\setfnsymbol{circlednumber}
%    \end{macrocode}
%
% \subsection{封面字段设置}
%
% 国家图书馆封面字段设置:
%    \begin{macrocode}
\newcommand*{\classification}[1]{%
  \renewcommand*{\jnut@value@nlc@classification}{#1}}
\newcommand*{\securitylevel}[1]{%
  \renewcommand*{\jnut@value@nlc@securitylevel}{#1}}
\newcommand*{\openlevel}{\jnut@cap@nlc@openlevel}
\newcommand*{\controllevel}{\jnut@cap@nlc@controllevel}
\newcommand*{\confidentiallevel}{\jnut@cap@nlc@confidentiallevel}
\newcommand*{\clasifiedlevel}{\jnut@cap@nlc@clasifiedlevel}
\newcommand*{\mostconfidentiallevel}{\jnut@cap@nlc@mostconfidentiallevel}
\newcommand*{\udc}[1]{%
  \renewcommand*{\jnut@value@nlc@udc}{#1}}
\newcommand*{\nlctitlea}[1]{%
  \renewcommand{\jnut@value@nlc@titlea}{#1}}
\newcommand*{\nlctitleb}[1]{%
  \renewcommand{\jnut@value@nlc@titleb}{#1}}
\newcommand*{\nlctitlec}[1]{%
  \renewcommand{\jnut@value@nlc@titlec}{#1}}
\newcommand*{\supervisorinfo}[1]{%
  \renewcommand{\jnut@value@nlc@supervisorinfo}{#1}}
\newcommand*{\chairman}[1]{%
  \renewcommand{\jnut@value@nlc@chairman}{#1}}
\newcommand*{\reviewera}[1]{%
  \renewcommand{\jnut@value@nlc@reviewera}{#1}}
\newcommand*{\reviewerb}[1]{%
  \renewcommand{\jnut@value@nlc@reviewerb}{#1}}
\newcommand*{\reviewerc}[1]{%
  \renewcommand{\jnut@value@nlc@reviewerc}{#1}}
\newcommand*{\reviewerd}[1]{%
  \renewcommand{\jnut@value@nlc@reviewerd}{#1}}
\newcommand*{\nlcdate}[1]{%
  \renewcommand{\jnut@value@nlc@date}{#1}}
%    \end{macrocode}
%
% 中文封面字段设置:
%    \begin{macrocode}
\renewcommand*{\title}[1]{%
  \renewcommand{\jnut@value@title}{#1}}
\newcommand*{\titlea}[1]{%
  \renewcommand{\jnut@value@titlea}{#1}}
\newcommand*{\titleb}[1]{%
  \renewcommand{\jnut@value@titleb}{#1}}
\renewcommand*{\author}[1]{%
  \renewcommand{\jnut@value@author}{#1}}
\newcommand*{\telphone}[1]{%
  \renewcommand{\jnut@value@telphone}{#1}}
\newcommand*{\email}[1]{%
  \renewcommand{\jnut@value@email}{#1}}
\newcommand*{\studentnum}[1]{%
  \renewcommand{\jnut@value@studentnum}{#1}}
\newcommand*{\grade}[1]{%
  \renewcommand{\jnut@value@grade}{#1}}
\newcommand*{\supervisor}[1]{%
  \renewcommand{\jnut@value@supervisor}{#1}}
\newcommand*{\supervisortelphone}[1]{%
  \renewcommand{\jnut@value@supervisortelphone}{#1}}
\newcommand*{\major}[1]{%
  \renewcommand{\jnut@value@major}{#1}}
\newcommand*{\researchfield}[1]{%
  \renewcommand{\jnut@value@researchfield}{#1}}
\newcommand*{\department}[1]{%
  \renewcommand{\jnut@value@department}{#1}}
\newcommand*{\institute}[1]{%
  \renewcommand{\jnut@value@institute}{#1}}
\newcommand*{\directormembers}[1]{%
  \renewcommand{\jnut@value@directormembers}{#1}}
\newcommand*{\submitdate}[1]{%
  \renewcommand{\jnut@value@submitdate}{#1}}
\newcommand*{\defenddate}[1]{%
  \renewcommand{\jnut@value@defenddate}{#1}}
\newcommand*{\degreedate}[1]{%
  \renewcommand{\jnut@value@degreedate}{#1}}
\renewcommand*{\date}[1]{%
  \renewcommand{\jnut@value@date}{#1}}
%    \end{macrocode}
%
% 英文封面字段设置:
%    \begin{macrocode}
\newcommand*{\englishtitlea}[1]{%
  \renewcommand{\jnut@value@en@titlea}{#1}}
\newcommand*{\englishtitleb}[1]{%
  \renewcommand{\jnut@value@en@titleb}{#1}}
\newcommand*{\englishauthor}[1]{%
  \renewcommand{\jnut@value@en@author}{#1}}
\newcommand{\englishsupervisor}[1]{%
  \renewcommand{\jnut@value@en@supervisor}{#1}}
\newcommand{\englishmajor}[1]{%
  \renewcommand{\jnut@value@en@major}{#1}}
\newcommand{\englishdepartment}[1]{%
  \renewcommand{\jnut@value@en@department}{#1}}
\newcommand{\englishinstitute}[1]{%
  \renewcommand{\jnut@value@en@institute}{#1}}
\newcommand*{\englishdate}[1]{%
  \renewcommand{\jnut@value@en@date}{#1}}
%    \end{macrocode}
%
% \subsection{生成封面}
%
% \begin{macro}{\jnutunderline}
% 定义封面中用到的生成下划线的宏。
%    \begin{macrocode}
\newcommand{\jnut@underline}[2][\textwidth]%
           {\CJKunderline{\makebox[#1]{#2}}}
\def\jnutunderline{\@ifnextchar[\jnut@underline\CJKunderline}
%    \end{macrocode}
% \end{macro}
%
% \begin{macro}{\makenlctitle}
% 定义生成国家图书馆封面的命令。注意我们使用了前面修改过的\cs{cleardoublepage}命令来插入
% 无页眉页脚的空白页。
%    \begin{macrocode}
\newcommand*{\makenlctitle}{%
  \thispagestyle{empty}
  \pdfbookmark[0]{\jnut@cap@nlc}{nlc}
  {\songti\zihao{-4}
    \makebox[40pt][l]{\jnut@cap@nlc@classification}
    \jnutunderline[150pt]{\jnut@value@nlc@classification}
    \hfill
    \makebox[40pt][r]{\jnut@cap@nlc@securitylevel}
    \jnutunderline[150pt]{\jnut@value@nlc@securitylevel}
    \vskip 10pt
    \makebox[40pt][l]{\jnut@cap@nlc@udc}
    \jnutunderline[150pt]{\jnut@value@nlc@udc}
  }
  \vskip\stretch{2}
  \begin{center}
    \def\ULthickness{1pt}
    {\kaishu\zihao{-0} \jnut@cap@nlc@title}
    {\kaishu\zihao{1}
    \vskip \stretch{1}
    \jnutunderline[12em]{\jnut@value@nlc@titlea}\\
    \jnutunderline[12em]{\jnut@value@nlc@titleb}\\
    \jnutunderline[12em]{\jnut@value@nlc@titlec}\\
    }
    \vskip \stretch{1}
    {\kaishu\zihao{4}\jnut@cap@nlc@quotetitle}
    \vskip \stretch{1}
    {\kaishu\zihao{1}\jnutunderline{\jnut@value@author}}
    \vskip \stretch{1}
    {\kaishu\zihao{4}\jnut@cap@nlc@author}
  \end{center}
  \vskip\stretch{1}
  {\kaishu\zihao{4}
    \noindent\jnut@cap@nlc@supervisor%
    \jnutunderline[94pt]{\jnut@value@supervisor}\par
    \noindent\jnutunderline[\textwidth]{%
      \jnut@value@nlc@supervisorinfo}\par
    \noindent\jnut@cap@nlc@degree%
    \jnutunderline[8em]{\jnut@value@degree}%
    \noindent\jnut@cap@nlc@major%
    \jnutunderline[164pt]{\jnut@value@major}\par
    \noindent\jnut@cap@nlc@submitdate%
    \jnutunderline[8em]{\jnut@value@submitdate}%
    \jnut@cap@nlc@defenddate%
    \jnutunderline[134pt]{\jnut@value@defenddate}\par
    \noindent\jnut@cap@nlc@institute\jnutunderline[290pt]{}\par
    \noindent\hfill\jnut@cap@nlc@chairman%
    \jnutunderline[9em]{\jnut@value@nlc@chairman}\par
    \noindent\hfill\jnut@cap@nlc@reviwer%
    \jnutunderline[9em]{\jnut@value@nlc@reviewera}\par
    \noindent\hfill\jnutunderline[9em]{\jnut@value@nlc@reviewerb}\par
    \noindent\hfill\jnutunderline[9em]{\jnut@value@nlc@reviewerc}\par
    \noindent\hfill\jnutunderline[9em]{\jnut@value@nlc@reviewerd}\par
  }
  \cleardoublepage
}
%    \end{macrocode}
% \end{macro}
%
% \begin{macro}{\maketitle}
% 重新定义{\LaTeX}提供的\cs{maketitle}命令,使其生成江南大学学术论文所需的中文封面。
% 注意我们使用了前面修改过的\cs{cleardoublepage}命令来插入无页眉页脚的空白页。
%    \begin{macrocode}
\newcommand*{\jnut@covertable@onelinetitle}{
  \begin{tabular}{p{7.3em}c}
      \makebox[7em][s]{\jnut@cap@cover@title:}
      & \jnutunderline[310pt]{\jnut@value@title}\\
      \makebox[7em][s]{\jnut@cap@cover@entitle:}
      & \jnutunderline[310pt]{\jnut@value@en@titlea}\\
      \makebox[7em][s]{}
      & \jnutunderline[310pt]{\jnut@value@en@titleb}\\
      \makebox[7em][s]{\jnut@cap@cover@author:}
      & \jnutunderline[310pt]{\jnut@value@author}\\
      \makebox[7em][s]{\jnut@cap@cover@major:}
      & \jnutunderline[310pt]{\jnut@value@major}\\
      \makebox[7em][s]{\jnut@cap@cover@researchfield:}
      & \jnutunderline[310pt]{\jnut@value@researchfield}\\
      \makebox[7em][s]{\jnut@cap@cover@supervisor:}
      & \jnutunderline[310pt]{\jnut@value@supervisor}\\
      \makebox[7em][s]{\jnut@cap@cover@directormembers:}
      & \jnutunderline[310pt]{\jnut@value@directormembers}\\
      \makebox[7em][s]{\jnut@cap@cover@degreedate:}
      & \jnutunderline[310pt]{\jnut@value@degreedate}\\
  \end{tabular}
}
\newcommand*{\jnut@covertable@twolinetitle}{
  \begin{tabular}{p{7.3em}c}
      \makebox[7em][s]{\jnut@cap@cover@title:}
      & \jnutunderline[310pt]{\jnut@value@titlea}\\
      \makebox[7em][s]{}
      & \jnutunderline[310pt]{\jnut@value@titleb}\\
      \makebox[7em][s]{\jnut@cap@cover@entitle:}
	  & \jnutunderline[310pt]{\zihao{5}\jnut@value@en@titlea}\\
      \makebox[7em][s]{}
	  & \jnutunderline[310pt]{\zihao{5}\jnut@value@en@titleb}\\
      \makebox[7em][s]{\jnut@cap@cover@author:}
      & \jnutunderline[310pt]{\jnut@value@author}\\
      \makebox[7em][s]{\jnut@cap@cover@major:}
      & \jnutunderline[310pt]{\jnut@value@major}\\
      \makebox[7em][s]{\jnut@cap@cover@researchfield:}
      & \jnutunderline[310pt]{\jnut@value@researchfield}\\
      \makebox[7em][s]{\jnut@cap@cover@supervisor:}
      & \jnutunderline[310pt]{\jnut@value@supervisor}\\
      \makebox[7em][s]{\jnut@cap@cover@directormembers:}
      & \jnutunderline[310pt]{\jnut@value@directormembers}\\
      \makebox[7em][s]{\jnut@cap@cover@degreedate:}
      & \jnutunderline[310pt]{\jnut@value@degreedate}\\
  \end{tabular}
}
\renewcommand*{\maketitle}{%
  \thispagestyle{empty}
  \pdfbookmark[0]{\jnut@cap@cover}{cover}
  \begin{center}
	%% TODO: 分类号,密级
    \vskip 10mm
    \includegraphics[height=1.46cm,width=5.20cm]{\jnut@cap@institute@name} \\
    {\bf\heiti\zihao{1}\makebox[10em][s]{\jnut@cap@cover@thesis}}\\
    \vskip\stretch{1}
    {\bgroup
    \songti\zihao{4}
    \def\tabcolsep{1pt}
    \def\arraystretch{1.5}
    \ifdefempty{\jnut@cap@cover@title}{\jnut@covertable@onelinetitle}{\jnut@covertable@twolinetitle}
    \egroup}\\
    \vskip \stretch{1}
    {\bf\kaishu\zihao{4}\jnut@value@date}
  \end{center}
  \cleardoublepage
}
%    \end{macrocode}
%  \end{macro}
%
% \begin{macro}{\makeenglishtitle}
% 定义生成英文封面的命令。注意我们使用了前面修改过的\cs{cleardoublepage}命令来插入无页眉
% 页脚的空白页。
%    \begin{macrocode}
\newcommand*{\makeenglishtitle}{%
  \thispagestyle{empty}
  \begin{center}
    \vspace*{20pt}
	%% TODO: 两行英文标题;查证江南大学是否完全不需要英文封面
    \bf\sffamily\zihao{2}\jnut@value@en@titlea
    \vskip \stretch{1}
    \normalfont\rmfamily\zihao{4}\jnut@cap@cover@en@by
    \vskip 3pt
    \bf\sffamily\zihao{4}\jnut@value@en@author
    \vskip\stretch{1}
    \normalfont\rmfamily\zihao{4}\jnut@cap@cover@en@supervisor
    \vskip 3pt
    \normalfont\sffamily\zihao{4}\jnut@value@en@supervisor
    \vskip\stretch{1}
    \normalsize\rmfamily\jnut@cap@cover@en@statement
    \vskip\stretch{2}
    \includegraphics[width=2.5cm]{\jnut@cap@institute@logo} \\
    \vskip 3mm
    \normalfont\jnut@value@en@department\\
    \jnut@value@en@institute
    \vskip 30pt
    \normalfont\normalsize\jnut@value@en@date
  \end{center}
  \normalfont
  \cleardoublepage
}
%    \end{macrocode}
% \end{macro}
%
% \subsection{独创性声明、关于论文使用授权的说明}
%
% \begin{macro}{\makeoriginalitypage}
% 生成固定格式的独创性声明与使用授权说明页面。
%	 \begin{macrocode}
\newcommand*{\makeoriginalitypage}{%
  \thispagestyle{empty}
  \bf\kaishu\zihao{4}
  \begin{center}
    \heiti\zihao{3}独 创 性 声 明
  \end{center}
  \vskip 3pt
  本人声明所呈交的学位论文是本人在导师指导下进行的研究工作及取得的研究成果。
  尽我所知,除了文中特别加以标注和致谢的地方外,论文中不包含其他人已经发表或撰写过的研究成果,
  也不包含本人为获得江南大学或其它教育机构的学位或证书而使用过的材料。
  与我一同工作的同志对本研究所做的任何贡献均已在论文中作了明确的说明并表示谢意。
  \vskip 1pt
  %% TODO: 此处格式
  签名:日期:

  \vskip 10pt
  \begin{center}
    \heiti\zihao{3}关于论文使用授权的说明
  \end{center}

  本学位论文作者完全了解江南大学有关保留、使用学位论文的规定:
  江南大学有权保留并向国家有关部门或机构送交论文的复印件和磁盘,
  允许论文被查阅和借阅,可以将学位论文的全部或部分内容编入有关数据库进行检索,
  可以采用影印、缩印或扫描等复制手段保存、汇编学位论文,
  并且本人电子文档的内容和纸质论文的内容相一致。

  保密的学位论文在解密后也遵守此规定。
  \vskip 1pt
  签名:导师签名:
  日期:
  \normalfont
  \cleardoublepage
}
%    \end{macrocode}
% \end{macro}
%
% \subsection{摘要页}
%
% \begin{macro}{\abstracttitlea}
% 用于设置中文摘要页论文标题的第一行。
%    \begin{macrocode}
\newcommand*{\abstracttitlea}[1]{%
  \renewcommand{\jnut@value@abstract@titlea}{#1}%
}
%    \end{macrocode}
% \end{macro}
%
% \begin{macro}{\abstracttitleb}
% 用于设置中文摘要页论文标题的第二行。
%    \begin{macrocode}
\newcommand*{\abstracttitleb}[1]{%
  \renewcommand{\jnut@value@abstract@titleb}{#1}%
}
%    \end{macrocode}
% \end{macro}
%
% \begin{environment}{abstract}
% 定义中文摘要环境。该环境自动生成江南大学中文摘要页。注意我们使用了前面修改过的
% \cs{cleardoublepage}命令来插入无页眉页脚的空白页。
%    \begin{macrocode}
\newenvironment{abstract}{%
  \thispagestyle{empty}
  \pdfbookmark[0]{\jnut@cap@abstract}{abstract}
  \begin{center}
    {\bf\heiti\zihao{3}\jnut@cap@abstract@abstractname}
  \end{center}%
  \normalsize\par%
}{%
  \cleardoublepage
}
%    \end{macrocode}
% \end{environment}
%
% \begin{macro}{\keywords}
% 定义生成中文摘要关键词的命令。此命令必须放在\env{abstract}环境内的末尾使用。中
% 文关键词之间应以中文全角分号隔开,末尾不需要加标点。
%    \begin{macrocode}
\newcommand{\keywords}[1]{%
  \renewcommand*{\jnut@value@abstract@keywords}{#1}%
  \par\vspace{2ex}\noindent%
  {\bf\jnut@cap@abstract@keywordsname{:}}~{#1}%
}
%    \end{macrocode}
% \end{macro}
%
% \begin{macro}{\englishabstracttitlea}
% 用于设置英文摘要页论文标题的第一行。
%    \begin{macrocode}
\newcommand*{\englishabstracttitlea}[1]{%
  \renewcommand{\jnut@value@abstract@en@titlea}{#1}%
}
%    \end{macrocode}
% \end{macro}
%
% \begin{macro}{\englishabstracttitleb}
% 用于设置英文摘要页论文标题的第二行。
%    \begin{macrocode}
\newcommand*{\englishabstracttitleb}[1]{%
  \renewcommand{\jnut@value@abstract@en@titleb}{#1}%
}
%    \end{macrocode}
% \end{macro}
%
% \begin{environment}{englishabstract}
% 定义英文摘要环境。该环境自动生成江南大学英文摘要页。注意我们使用了前面修改过的
% \cs{cleardoublepage}命令来插入无页眉页脚的空白页。
%    \begin{macrocode}
\newenvironment{englishabstract}{%
  \thispagestyle{empty}
  \pdfbookmark[0]{\jnut@cap@abstract@en}{englishabstract}
  \begin{center}
    {\bf\zihao{3}\jnut@cap@abstract@en@abstractname}
  \end{center}%
  \normalsize\par%
}{%
  \cleardoublepage
}
%    \end{macrocode}
% \end{environment}
%
% \begin{macro}{\englishkeywords}
% 定义生成中文摘要关键词的命令。此命令必须放在\env{englishabstract}环境内的末尾
% 使用。英文关键词之间应以英文半角逗号隔开,末尾不需要加标点。
%    \begin{macrocode}
\newcommand{\englishkeywords}[1]{%
  \renewcommand*{\jnut@value@abstract@en@keywords}{#1}%
  \par\vspace{2ex}\noindent%
  {\bf\jnut@cap@abstract@en@keywordsname{:}}~~{#1}%
}
%    \end{macrocode}
% \end{macro}
%
% \subsection{前言章节}
%
% \begin{environment}{preface}
% 该环境用于``致谢''页。
%    \begin{macrocode}
\newenvironment{preface}{%
  \nchapter{\jnut@cap@preface}
}{}
%    \end{macrocode}
% \end{environment}
%
% \subsection{致谢章节}
%
% \begin{environment}{acknowledgement}
% 该环境用于``致谢''页。
%    \begin{macrocode}
\newenvironment{acknowledgement}{%
  \nchapter{\jnut@cap@acknowledgementname}
}{}
%    \end{macrocode}
% \end{environment}
%
% \subsection{简历与科研成果页}
%
% \begin{environment}{resume}
% 该环境用于生成作者简历与科研成果页。
%    \begin{macrocode}
\newenvironment{resume}{%
  \nchapter{\jnut@cap@resume@chaptername}
}{}
%    \end{macrocode}
% \end{environment}
%
% \begin{environment}{authorinfo}
% 定义作者基本信息环境。该环境自动生成作者基本信息段落。此环境必须被放在|resume|环境中。
%    \begin{macrocode}
\newenvironment{authorinfo}{%
  \paragraph*{\jnut@cap@resume@authorinfo}
}{}
%    \end{macrocode}
% \end{environment}
%
% \begin{environment}{education}
% 定义作者教育背景列表环境。此环境必须被放在|resume|环境中。
%    \begin{macrocode}
\newenvironment{education}{%
  \paragraph*{\jnut@cap@resume@education}
  \begin{description}[labelindent=0em, leftmargin=8em, style=sameline]
}{%
  \end{description}
}
%    \end{macrocode}
% \end{environment}
%
% \begin{environment}{publications}
% 定义作者攻读学位期间发表论文列表环境。此环境必须被放在|resume|环境中。
%    \begin{macrocode}
\newenvironment{publications}{%
  \paragraph*{\jnut@cap@resume@publications}
  \begin{enumerate}[label=\arabic*., labelindent=0em, leftmargin=*]
}{%
  \end{enumerate}
}
%    \end{macrocode}
% \end{environment}
%
% \begin{environment}{projects}
% 定义作者攻读学位期间参与的科研课题列表环境。此环境必须被放在|resume|环境中。
%    \begin{macrocode}
\newenvironment{projects}{%
  \paragraph*{\jnut@cap@resume@projects}
  \begin{enumerate}[label=\arabic*., labelindent=0em, leftmargin=*]
}{%
  \end{enumerate}
}
%    \end{macrocode}
% \end{environment}
%
% \subsection{学位论文出版授权书}
%
% \begin{macro}{\jnut@cap@datefield}
% 该命令生成一个由用户填写的日期域。
%    \begin{macrocode}
\newcommand*{\jnut@cap@datefield}{%
\jnutunderline[1cm]{}{\jnut@cap@year}%
\jnutunderline[1cm]{}{\jnut@cap@month}%
\jnutunderline[1cm]{}{\jnut@cap@day}
}
%    \end{macrocode}
% \end{macro}
%
% \begin{macro}{\jnut@license@makedeclaration}
% 该命令生成《学位论文出版授权书》中的授权声明。
%    \begin{macrocode}
\newcommand*{\jnut@license@makedeclaration}{%
\par\jnut@cap@license@declaration
\vspace{5mm}
\begin{flushright}
  \jnut@cap@license@sign\jnutunderline[6cm]{}\\
  \jnut@cap@datefield\\
\end{flushright}%
}
%    \end{macrocode}
% \end{macro}
%
% \begin{macro}{\jnut@license@maketable}
% 该命令生成《学位论文出版授权书》中的论文信息表格。
%    \begin{macrocode}
\newcommand*{\jnut@license@maketable}{%
\noindent\zihao{5}%
\begin{tabular*}{\textwidth}
    {|C{2.2cm}|C{2cm}|C{1.5cm}|C{2.1cm}|C{1.42cm}C{1.5cm}|C{1.25cm}|}
  \hline
  \cell{2.2cm}{1cm}{\jnut@cap@license@title}
  & \multicolumn{6}{c|}{\jnut@value@title} \\
  \hline
  \cell{2.2cm}{1cm}{\jnut@cap@license@studentnum}
  & {\jnut@value@studentnum}
  & {\jnut@cap@license@department}
  & \multicolumn{2}{c|}{%
    \cell{3.52cm}{1cm}{\jnut@value@department}%
  }
  & {\jnut@cap@license@grade}
  & {\jnut@value@grade} \\
  \hline
  \cell{2.2cm}{1.5cm}{\jnut@cap@license@category}
  & \multicolumn{3}{c}{
    \begin{tabular*}{5.6cm}{p{2.8cm}p{2.8cm}}
       \ifjnut@master%
           {{\zihao{-4}$\CheckedBox$}}%
       \else%
           {{\zihao{4}$\Square$}}%
       \fi%
       \jnut@cap@license@categorymaster
      & {\zihao{4}$\Square$}%
       \jnut@cap@license@categorymasterspec \\
       \ifjnut@phd%
           {{\zihao{-4}$\CheckedBox$}}%
       \else%
           {{\zihao{4}$\Square$}}%
       \fi%
       \jnut@cap@license@categoryphd
      & {\zihao{4}$\Square$}%
       \jnut@cap@license@categoryphdspec \\
    \end{tabular*}}
  & \multicolumn{3}{c|}{%
    \raisebox{-1em}{\jnut@cap@license@categoryhint}}\\
  \hline
  \cell{2.2cm}{1cm}{\jnut@cap@license@telphone}
  & \multicolumn{2}{c|}{{\jnut@value@telphone}}
  & {\jnut@cap@license@email}
  & \multicolumn{3}{c|}{{\jnut@value@email}}  \\
  \hline
  \cell{2.2cm}{1cm}{\jnut@cap@license@supervisorname}
  & \multicolumn{2}{c|}{{\jnut@value@supervisor}}
  & {\jnut@cap@license@supervisortelphone}
  & \multicolumn{3}{c|}{{\jnut@value@supervisortelphone}} \\
  \hline
\end{tabular*}
}
%    \end{macrocode}
% \end{macro}
%
% \begin{macro}{\makelicense}
% 该命令用于生成《学位论文出版授权书》。该授权书中的一些字段将根据用户所设置的文
% 档属性自动填写。
%    \begin{macrocode}
\newcommand*{\makelicense}{%
  \thispagestyle{empty}
  \nchapter{\jnut@cap@license@chaptername}
  \jnut@license@makedeclaration
  \par\vspace{1em}
  \jnut@license@maketable
  \par\vspace{1em}
  \noindent\jnut@cap@license@securitylevel\par
  \noindent\ifthenelse{\equal{\jnut@value@nlc@securitylevel}
                             {\jnut@cap@nlc@openlevel}}
                      {{\zihao{-4}$\CheckedBox$}}
                      {{\zihao{4}$\Square$}}%
  {\jnut@cap@license@public}\par
  \noindent\ifthenelse{\equal{\jnut@value@nlc@securitylevel}
                             {\jnut@cap@nlc@openlevel}}
                      {{\zihao{4}$\Square$}}
                      {{\zihao{-4}$\CheckedBox$}}%
  {\jnut@cap@license@secret}%
  \jnut@cap@datefield\hspace{0.5em}%
  \jnut@cap@to\hspace{0.5em}%
  \jnut@cap@datefield\par
  \vspace{1em}
  \noindent\jnut@cap@license@remark%
  \normalfont
  \cleardoublepage
}
%    \end{macrocode}
% \end{macro}
%
% \subsection{其他自定义命令和环境}
%
% \begin{macro}{\jnuthesis}
% 定义{\jnuthesis}文档类的logo。
%    \begin{macrocode}
\newcommand{\jnuthesis}{\texttt{JNU-Thesis}}
%    \end{macrocode}
% \end{macro}
%
% \begin{macro}{\zhdash}
% 定义中文破折号。
%    \begin{macrocode}
\newcommand{\zhdash}{\kern0.3ex\rule[0.8ex]{2em}{0.1ex}\kern0.3ex}
%    \end{macrocode}
% \end{macro}
%
% \begin{macro}{\cell}
% \cs{cell}\marg{width}\marg{height}\marg{text}用于定义一个宽度为\meta{width},
% 高度为\meta{height},内容为\meta{text}的的单元格。该单元格可放在表格中,用于控
% 制表格单元格的大小。
%    \begin{macrocode}
\newcommand{\cell}[3]{\parbox[c][#2][c]{#1}{\makebox[#1]{#3}}}
%    \end{macrocode}
% \end{macro}
%
% \begin{macro}{C}
% 定义一个新的表格列模式,|C{width}|,表示将内容居中,且列宽度为|width|。
%
% |array|环境中的\cs{centering}命令会改变\cs{newline}的定义,因此我们需要用
% \cs{arraybackslash}将其恢复;另外,我们也可能会在列内容中使用\cs{newline},因此在
% \cs{centering}后重新定义了\cs{newline}。
%
%    \begin{macrocode}
\newcolumntype{C}[1]{>{\centering\let\newline\\%
    \arraybackslash\hspace{0pt}}p{#1}}
%    \end{macrocode}
% \end{macro}
%
% \begin{environment}{arabicenum}
% 阿拉伯数字列表环境。该列表最多三层。
%    \begin{macrocode}
\newlist{arabicenum}{enumerate}{3}
\setlist[arabicenum,1]{label=\textnormal%
  {\textnormal{(\arabic*)}}}
\setlist[arabicenum,2]{label=\textnormal%
  {\textnormal{(\arabic{arabicenumi}.\arabic*)}}}
\setlist[arabicenum,3]{label=\textnormal%
  {\textnormal{(\arabic{arabicenumi}.\arabic{arabicenumii}.\arabic*)}}}
%    \end{macrocode}
% \end{environment}
%
% \begin{environment}{romanenum}
% 罗马数字列表环境。该列表最多两层。
%    \begin{macrocode}
\newlist{romanenum}{enumerate}{2}
\setlist[romanenum,1]{label={\textnormal{\roman*.}}}
\setlist[romanenum,2]{label={\textnormal{\alph*\,)}}}
%    \end{macrocode}
% \end{environment}
%
% \begin{environment}{alphaenum}
% 小写字母列表环境。该列表最多两层。
%    \begin{macrocode}
\newlist{alphaenum}{enumerate}{2}
\setlist[alphaenum,1]{label={\textnormal{\alph*\,)}}}
\setlist[alphaenum,2]{label={\textnormal{\alph{alphaenumi}.\arabic*\,)}}}
%    \end{macrocode}
% \end{environment}
%
% \begin{environment}{caseenum}
% 情况分类列表环境。该列表最多两层。
%    \begin{macrocode}
\newlist{caseenum}{enumerate}{2}
\setlist[caseenum,1]{label={\textnormal{\jnut@cap@case\arabic*.}}}
\setlist[caseenum,2]{label={\textnormal{\jnut@cap@subcase\arabic{caseenumi}.\arabic*.}}}
\setlist[caseenum]{leftmargin=*}
%    \end{macrocode}
% \end{environment}
%
% \begin{environment}{stepenum}
% 步骤列表环境。该列表最多两层。
%    \begin{macrocode}
\newlist{stepenum}{enumerate}{2}
\setlist[stepenum,1]{label={\textnormal{\jnut@cap@step\arabic*.}}}
\setlist[stepenum,2]{label={\textnormal{\jnut@cap@substep\arabic{stepenumi}.\arabic*.}}}
\setlist[stepenum]{leftmargin=*}
%    \end{macrocode}
% \end{environment}
%
% \subsection{设置PDF文档属性}
%
% \begin{macro}{\jnut@setpdfinfo}
% 此命令设置PDF文档属性,依赖于|hyperref|宏包。
%    \begin{macrocode}
\newcommand*{\jnut@setpdfinfo}{\hypersetup{%
        pdftitle={\jnut@value@title},
        pdfauthor={\jnut@value@author},
        pdfsubject={\jnut@cap@cover@apply},
        pdfkeywords={\jnut@value@abstract@keywords},
        pdfcreator={\jnut@value@author},
        pdfproducer={XeLaTeX with the JNU-Thesis document class}}
}
%    \end{macrocode}
% \end{macro}
%
% 在文档的\cs{begin{document}}之后立即调用\cs{jnut@setpdfinfo}命令设置PDF文档属性。
%    \begin{macrocode}
\AtBeginDocument{\jnut@setpdfinfo}
%</cls>
%    \end{macrocode}
% \StopEventually{\PrintIndex}
% \Finale
%
% \iffalse
%    \begin{macrocode}
%<*dtx-style>
\ProvidesPackage{dtx-style}
\RequirePackage{amssymb}
\RequirePackage{calc}
\RequirePackage{array,longtable}
\RequirePackage{fancybox,fancyvrb}
\RequirePackage{xcolor}
\RequirePackage{txfonts}
\RequirePackage{xltxtra}
\RequirePackage{subfigure}
\RequirePackage{marvosym}
\RequirePackage{booktabs}
\RequirePackage{paralist}
\RequirePackage{enumitem}
\RequirePackage{titlesec}
\RequirePackage{titling}
\RequirePackage{fancyhdr}
\RequirePackage{geometry}
\RequirePackage{indentfirst}
\RequirePackage[CJKnumber,CJKchecksingle]{xeCJK}
\RequirePackage[hyphens]{url} % must be load before hypdoc package
\RequirePackage{hypdoc} % it will load hyperref package
\RequirePackage{dtklogos}
\hypersetup{%
    unicode=false,
    hyperfootnotes=true,
    hyperindex=true,
    pageanchor=true,
    CJKbookmarks=true,
    bookmarksnumbered=true,
    bookmarksopen=true,
    bookmarksopenlevel=0,
    breaklinks=true,
    colorlinks=false,
    plainpages=false,
    pdfpagelabels,
    pdfborder=0 0 0%
}

\newcommand{\env}[1]{\texttt{#1}}

% 定义英文字体名称。
\newcommand*{\jnut@enfn@main}{Times New Roman}
\newcommand*{\jnut@enfn@sans}{Arial}
\newcommand*{\jnut@enfn@mono}{Courier New}

% 选择中文字体
\newcommand*{\jnut@zhfn@songti}{Adobe Song Std}
\newcommand*{\jnut@zhfn@heiti}{Adobe Heiti Std}
\newcommand*{\jnut@zhfn@kaishu}{Adobe Kaiti Std}
\newcommand*{\jnut@zhfn@fangsong}{Adobe Fangsong Std}

% 定义中文字体
\setCJKfamilyfont{song}{\jnut@zhfn@songti}
\setCJKfamilyfont{hei}{\jnut@zhfn@heiti}
\setCJKfamilyfont{kai}{\jnut@zhfn@kaishu}
\setCJKfamilyfont{fangsong}{\jnut@zhfn@fangsong}

\setCJKmainfont[BoldFont={\jnut@zhfn@heiti},
                ItalicFont={\jnut@zhfn@kaishu}]{\jnut@zhfn@songti}
\setCJKsansfont{\jnut@zhfn@heiti}
\setCJKmonofont{\jnut@zhfn@fangsong}

% 定义文档使用的英文字体
\setmainfont{\jnut@enfn@main}
\setsansfont{\jnut@enfn@sans}
\setmonofont{\jnut@enfn@mono}

% 定义中文字体选择命令
\newcommand*{\songti}{\CJKfamily{song}}
\newcommand*{\heiti}{\CJKfamily{hei}}
\newcommand*{\kaishu}{\CJKfamily{kai}}
\newcommand*{\fangsong}{\CJKfamily{fangsong}}

\renewcommand{\contentsname}{目\hspace{2em}录}
\renewcommand{\abstractname}{摘\hspace{2em}要}
\renewcommand{\indexname}{索\hspace{2em}引}
\renewcommand{\figurename}{图}
\renewcommand{\tablename}{表}
\renewcommand{\refname}{参考文献}

\setlength{\parskip}{4pt plus1pt minus0pt}
\setlength{\topsep}{0pt}
\setlength{\partopsep}{0pt}
\setlength{\parindent}{2em}
\addtolength{\oddsidemargin}{-1cm}
\advance\textwidth 1.5cm
\addtolength{\topmargin}{-1cm}
\addtolength{\headsep}{0.3cm}
\addtolength{\textheight}{2.3cm}

\newcommand{\zhdash}{\kern0.3ex\rule[0.8ex]{2em}{0.1ex}\kern0.3ex}

\renewcommand{\baselinestretch}{1.3}

\DefineVerbatimEnvironment{shell}{Verbatim}%
  {frame=single,framerule=0.1mm,rulecolor=\color{black},%
   framesep=2mm,fontsize=\small,gobble=1}

\DefineVerbatimEnvironment{example}{Verbatim}%
  {frame=single,framerule=0.1mm,rulecolor=\color{black},%
   framesep=2mm,baselinestretch=1.2,fontsize=\small,gobble=1}

\long\def\myentry#1{\vskip5pt\par\noindent\llap{{\color{blue}\fangsong #1}}%
  \marginpar{\strut}\hskip\parindent}

% 使用|titlesec|宏包提供的\titleformat命令设置标题格式:
\titleformat*{\section}{\Large\bfseries}
\titleformat*{\subsection}{\large\bfseries}
\titleformat*{\subsubsection}{\normalsize\bfseries}
\titleformat*{\paragraph}{\normalsize\bfseries}
\titleformat*{\subparagraph}{\normalsize\bfseries}

% 使用|titling|宏包设置标题的字体
\pretitle{\begin{center}\huge\bfseries}
\posttitle{\par\end{center}\vskip 1em}
\preauthor{\begin{center}
             \large \lineskip 0.5em}
\postauthor{\par\end{center}}
\predate{\begin{center}\large}
\postdate{\par\end{center}}

% 修改\cs{tableofcontents}命令用于生成目录页。
\renewcommand{\tableofcontents}{%
    \if@twocolumn
      \@restonecoltrue\onecolumn
    \else
      \@restonecolfalse
    \fi
    \section*{\hfill\contentsname\hfill}%
    \@mkboth{\MakeUppercase\contentsname}{\MakeUppercase\contentsname}%
    \@starttoc{toc}%
    \if@restonecol\twocolumn\fi
}

% 增加一种新的表格列对齐方式 C{width},表示该列内容居中且宽度为width
\newcolumntype{C}[1]{>{\centering\let\newline\\%
    \arraybackslash\hspace{0pt}}p{#1}}


% \dangericon 表示警告的图标
\font\manfnt=manfnt
\newcommand*{\dangericon}{\manfnt\char127}

% note 环境表示需特别注意的内容
\newenvironment{note}
               {\vskip1.5ex\par\noindent\llap{\dangericon\hskip2mm}\textbf{注意:}}
               {\vskip1.5ex}

% syntax 环境表示语法描述
\newenvironment{syntax}
               {\begin{center}}
               {\end{center}}

\newenvironment{suggestion}
               {\par\noindent\textbf{建议:}}{}

% 重新设置默认的列表的格式。
\setlist{%
  topsep=0.3em,             % 列表顶端的垂直空白
  partopsep=0pt,            % 列表环境前面紧接着一个空白行时其顶端的额外垂直空白
  itemsep=0ex plus 0.1ex,   % 列表项之间的额外垂直空白
  parsep=0pt,               % 列表项内的段落之间的垂直空白
  leftmargin=1.5em,           % 环境的左边界和列表之间的水平距离
  rightmargin=0em,          % 环境的右边界和列表之间的水平距离
  labelsep=0.5em,           % 包含标签的盒子与列表项的第一行文本之间的间隔
  labelwidth=2em,           % 包含标签的盒子的正常宽度;若实际宽度更宽,则使用实际宽度。
}

% 设置无序列表的标签符号。
\setlist[itemize,1]{label=$\bullet$}
\setlist[itemize,2]{label=$\blacksquare$}
\setlist[itemize,3]{label=$\Diamondblack$}

% 默认的|fancy|风格会在每一章最后的空白页(由于是双面打印)也加上页眉页脚,但我
% 们通常不希望如此。解决方法是修改{\LaTeX}内部的\cleardoublepage命令的定义如下:
\makeatletter
\def\cleardoublepage{\clearpage\if@twoside \ifodd\c@page\else
  \hbox{}\thispagestyle{empty}\newpage\if@twocolumn\hbox{}\newpage\fi\fi\fi}
\makeatother

% 文章用A4纸标准大小的白纸打印,页眉:2.6cm,页脚:2.4cm,页边距上下:3.5cm,左
% 右:3.2cm。
\geometry{headheight=2.6cm,headsep=3mm,footskip=13mm}
\geometry{top=3.5cm,bottom=3.5cm,left=3.2cm,right=3.2cm}


% \std{code}表示国家标准编号
\newcommand*{\std}[1]{{\normalfont #1}}

% 增加环境命令: \env{name} 表示名为 name 的环境
%% \newcommand{\env}[1]{\texttt{#1}}

% 修改\tableofcontents命令用于生成目录页,将目录页本身也被加入目录中。
\makeatletter
\renewcommand*{\tableofcontents}{%
    \if@twocolumn
      \@restonecoltrue\onecolumn
    \else
      \@restonecolfalse
    \fi
    \section*{\hfill\contentsname\hfill}%
    \@mkboth{\MakeUppercase\contentsname}{\MakeUppercase\contentsname}%
    \addcontentsline{toc}{section}{\contentsname}%
    \@starttoc{toc}%
    \if@restonecol\twocolumn\fi
}
\makeatother

% 设置索引页面的样式
\IndexPrologue{\clearpage\section*{\hfill\indexname\hfill}%
\markboth{\indexname}{\indexname}%
\addcontentsline{toc}{section}{\indexname}%
斜体数字表示对应项的描述所在页面的页码, %
带下划线的数字表示对应项的定义所在的代码行号,%
其他数字表示对应项所被引用的代码行号。%
}

% 设置索引页面的栏数
\setcounter{IndexColumns}{2}


\newcommand{\dashnumber}[2]%
  {{#1}\kern.07em\rule[.5ex]{.4em}{.1ex}\kern.07em{#2}}

%</dtx-style>
%    \end{macrocode}
% \fi
\endinput
