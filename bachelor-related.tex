%% 使用 jnuthesis 文档类生成南京大学学位论文的示例文档
%%
%% 作者:胡海星,starfish (at) gmail (dot) com
%% 项目主页: http://haixing-hu.github.io/jnu-thesis/
%%
%% 本样例文档中用到了吕琦同学的博士论文的提高和部分内容,在此对他表示感谢。
%%
\documentclass[bachelor,winfonts]{jnuthesis}
%% jnuthesis 文档类的可选参数有:
%%   nobackinfo 取消封二页导师签名信息。注意,按照南大的规定,是需要签名页的。
%%   phd/master/bachelor 选择博士/硕士/学士论文

% 使用 blindtext 宏包自动生成章节文字
% 这仅仅是用于生成样例文档,正式论文中一般用不到该宏包
\usepackage[math]{blindtext}

%%%%%%%%%%%%%%%%%%%%%%%%%%%%%%%%%%%%%%%%%%%%%%%%%%%%%%%%%%%%%%%%%%%%%%%%%%%%%%%
% 设置论文的中文封面

% 如果论文标题过长,可以分两行,第一行用\titlea{}定义,第二行用\titleb{}定义,将上面的\title{}注释掉
\titlea{半轻衰变$D^+\to \omega(\phi)e^+\nu_e$的研究}
\titleb{和弱衰变$J/\psi \to D_s^{(*)-}e^+\nu_e$的寻找}

% 论文作者姓名
\author{韦小宝}
% 论文作者学生证号
\studentnum{1234567890}
% 导师姓名职称
\supervisor{陈近南~~教授}
% 第二行导师姓名职称,仿照第一行填写,没有则留空
\supervisorb{}
% 论文作者的学科与专业方向
\major{数字媒体技术}
% 论文作者所在院系的中文名称,学士学位论文此处不带“学院”二字
\department{数字媒体}
% 论文作者所在学校或机构的名称。此属性可选,默认值为``江南大学''。
\institute{江南大学}
% 学士学位获得日期,需设置年、月。
\bachelordegreeyear{二〇一七}
\bachelordegreemonth{六}

%%%%%%%%%%%%%%%%%%%%%%%%%%%%%%%%%%%%%%%%%%%%%%%%%%%%%%%%%%%%%%%%%%%%%%%%%%%%%%%
\begin{document}

%%%%%%%%%%%%%%%%%%%%%%%%%%%%%%%%%%%%%%%%%%%%%%%%%%%%%%%%%%%%%%%%%%%%%%%%%%%%%%%

% 制作中文封面
\makebachelorrelated

% 制作外文资料翻译封面页
\makebachelortranslation

%%%%%%%%%%%%%%%%%%%%%%%%%%%%%%%%%%%%%%%%%%%%%%%%%%%%%%%%%%%%%%%%%%%%%%%%%%%%%%%
% 开始正文部分
\mainmatter

%%%%%%%%%%%%%%%%%%%%%%%%%%%%%%%%%%%%%%%%%%%%%%%%%%%%%%%%%%%%%%%%%%%%%%%%%%%%%%%
% 外文资料原文;格式仅供参考
\chapter*{原文}\label{chapter_originaltext}

\Blindtext

%%%%%%%%%%%%%%%%%%%%%%%%%%%%%%%%%%%%%%%%%%%%%%%%%%%%%%%%%%%%%%%%%%%%%%%%%%%%%%%
% 译文;格式仅供参考
\chapter*{翻译}\label{chapter_translation}

在分布式网络领域,沿着高性能集群、普世计算、网格计算的方向,现已走入云
计算时代。

云计算对信息技术架构造成了越来越大的影响。例如,借助Amazon EC2云平台,
用户借助其基础设施,可以十分方便的部署各类应用,以支持企业服务需求。用
户可以按需购买计算资源,网络带宽,存储空间等各类资源以支持他们的业务需
求,并在业务完成之后迅速的归还这些资源。通过云技术,用户可以集中在他们
擅长的核心业务之中,而不会被诸如硬件购买、安装系统、网络设置、备份和安
全等等问题干扰。

与此同时,随着计算机的普及化和微型化,现在的手持设备拥有不输于7年前台式
机的处理能力。在网络时代面前,智能终端广泛普及,每个人都可成为信息源。
在信息爆炸的时代,数据挖掘、机器学习、金融分析和模拟等行业中不断涌现新
的需求,诸如针对用户行为和社会关系的挖掘进行广告精准投放,用户行为预测
等。为了支撑PB级尺度的数据规模,需要海量的计算节点,催生并不断促进了各
行各业对云计算基础设施的建设需求。

海量的数据需要海量的处理能力,然而海量的处理能力又需要高带宽的网络IO为
承载。作为云环境中最基础的一环,IaaS层在网络、存储、计算资源的分割这几
方面,承担起整个系统的基石。虽然并非必须,但一般来说,为考虑沙盘环境,
以及对资源的细粒度切割分配,IaaS通常会伴随着虚拟化技术的运用。虚拟化具
有许多与云计算切合的特点,例如,虚拟化可以屏蔽物理环境的差异,可在多物
理节点中进行无缝迁移,可对系统进行快照和还原。这些特点都与云计算时代所
追求的灵活性、高伸缩性、快速响应等特点而吻合。

在虚拟化实现方面,目前已取得了诸如ESXi,Xen,KVM等成熟成果。然而当虚拟
化扩大的一定规模,随着节点数目的增多,在网络方面将会面临一系列取舍的问
题。例如,基于二层交换的扁平网络,当节点数目上升到千数量级时,广播报文
将会极大的拖累网络性能,必须通过划分子网,通过三层路由等形式重新规划为
多层网络结构;另一方面,除了联通之外,还需要考虑ACL控制,负载均衡,外网
通讯等各类防火墙以及NAT规则的实现。这些复杂的网络配置,在一定程度上抵消
了虚拟机带来的灵活性。例如虽然虚拟机可根据需要动态迁移,但在迁移之后,
由于网络位置的变化需要重新进行网络参数配置。虽然虚拟机在迁移过程中系统
内部状态没有变化,但站在网络角度看,该虚拟节点跟关机重启没有区别。

针对上述问题,本文站在面向云计算时代的数据中心网络建设的角度,对网络模
型进行深入研究和探讨。通过改善二层交换网络的ARP机制来解决广播风暴问题,
引入比树形网络更为复杂的复杂网络理论,指导网络节点的互联模型。从而将网
络的复杂性隐藏在节点环境之外,在节点层面仅提供简单但巨大的二层交换扁平
网络。

%%%%%%%%%%%%%%%%%%%%%%%%%%%%%%%%%%%%%%%%%%%%%%%%%%%%%%%%%%%%%%%%%%%%%%%%%%%%%%%
\end{document}
