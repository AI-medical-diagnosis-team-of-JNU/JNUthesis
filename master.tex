%% 使用 jnuthesis 文档类生成南京大学学位论文的示例文档
%%
%% 作者:胡海星,starfish (at) gmail (dot) com
%% 项目主页: http://haixing-hu.github.io/jnu-thesis/
%%
%% 本样例文档中用到了吕琦同学的博士论文的提高和部分内容,在此对他表示感谢。
%%
\documentclass[master,winfonts]{jnuthesis}
%% jnuthesis 文档类的可选参数有:
%%   nobackinfo 取消封二页导师签名信息。注意,按照南大的规定,是需要签名页的。
%%   phd/master/bachelor 选择博士/硕士/学士论文

% 使用 blindtext 宏包自动生成章节文字
% 这仅仅是用于生成样例文档,正式论文中一般用不到该宏包
\usepackage[math]{blindtext}

%%%%%%%%%%%%%%%%%%%%%%%%%%%%%%%%%%%%%%%%%%%%%%%%%%%%%%%%%%%%%%%%%%%%%%%%%%%%%%%
% 设置论文的中文封面

% 论文标题,不可换行
% \title{}
% 如果论文标题过长,可以分两行,第一行用\titlea{}定义,第二行用\titleb{}定义,将上面的\title{}注释掉
\titlea{基于Docker, Android与Web的社会关系研究}
\titleb{——以江南大学为例}
% 论文的英文标题第一行
\englishtitlea{A Research on Social Relationships based on Docker, Android and Web}
% 论文的英文标题第二行
\englishtitleb{-- Take Jiangnan University as an Example}

% 论文作者姓名
\author{王雪瑞}
% 导师姓名职称
\supervisor{刘渊~~教授}
% 论文作者的学科与专业方向
\major{数字媒体技术}
% 论文作者的研究方向
\researchfield{网络媒体应用与安全技术}
% 论文作者所在院系的中文名称
\department{}
% 论文作者所在学校或机构的名称。此属性可选,默认值为``江南大学''。
\institute{江南大学}
% 论文的指导小组成员。
\directormembers{}
% 论文的提交日期,需设置年、月、日。
\submitdate{2017年5月}
% 论文的答辩日期,需设置年、月、日。
\defenddate{2017年6月}
% 论文的定稿日期,需设置年、月、日。此属性可选,默认值为最后一次编译时的日期,精确到日。
%% \date{2013年5月1日}
% 获得学位日期。
\degreedate{}

%%%%%%%%%%%%%%%%%%%%%%%%%%%%%%%%%%%%%%%%%%%%%%%%%%%%%%%%%%%%%%%%%%%%%%%%%%%%%%%
% 设置论文的英文封面

% 论文作者姓名的拼音
\englishauthor{WANG Xuerui}
% 导师姓名职称的英文
\englishsupervisor{}
% 论文作者学科与专业的英文名
\englishmajor{}
% 论文作者所在院系的英文名称
\englishdepartment{}
% 论文作者所在学校或机构的英文名称。此属性可选,默认值为``Jiangnan University''。
\englishinstitute{Jiangnan University}
% 论文完成日期的英文形式,它将出现在英文封面下方。需设置年、月、日。日期格式使用美国的日期
% 格式,即``Month day, year'',其中``Month''为月份的英文名全称,首字母大写;``day''为
% 该月中日期的阿拉伯数字表示;``year''为年份的四位阿拉伯数字表示。此属性可选,默认值为最后
% 一次编译时的日期。
\englishdate{May 1, 2017}

%%%%%%%%%%%%%%%%%%%%%%%%%%%%%%%%%%%%%%%%%%%%%%%%%%%%%%%%%%%%%%%%%%%%%%%%%%%%%%%
% 设置论文的中文摘要

% 设置中文摘要页面的论文标题及副标题的第一行。
% 此属性可选,其默认值为使用|\title|命令所设置的论文标题
% \abstracttitlea{数据中心网络模型研究}
% 设置中文摘要页面的论文标题及副标题的第二行。
% 此属性可选,其默认值为空白
% \abstracttitleb{}

%%%%%%%%%%%%%%%%%%%%%%%%%%%%%%%%%%%%%%%%%%%%%%%%%%%%%%%%%%%%%%%%%%%%%%%%%%%%%%%
% 设置论文的英文摘要

% 设置英文摘要页面的论文标题及副标题的第一行。
% 此属性可选,其默认值为使用|\englishtitle|命令所设置的论文标题
\englishabstracttitlea{}
% 设置英文摘要页面的论文标题及副标题的第二行。
% 此属性可选,其默认值为空白
\englishabstracttitleb{}

%%%%%%%%%%%%%%%%%%%%%%%%%%%%%%%%%%%%%%%%%%%%%%%%%%%%%%%%%%%%%%%%%%%%%%%%%%%%%%%
\begin{document}

%%%%%%%%%%%%%%%%%%%%%%%%%%%%%%%%%%%%%%%%%%%%%%%%%%%%%%%%%%%%%%%%%%%%%%%%%%%%%%%

% 制作中文封面
\maketitle

% 输出独创性声明与论文使用授权说明页面
\makeoriginalitypage

%%%%%%%%%%%%%%%%%%%%%%%%%%%%%%%%%%%%%%%%%%%%%%%%%%%%%%%%%%%%%%%%%%%%%%%%%%%%%%%
% 开始前言部分
\frontmatter

%%%%%%%%%%%%%%%%%%%%%%%%%%%%%%%%%%%%%%%%%%%%%%%%%%%%%%%%%%%%%%%%%%%%%%%%%%%%%%%
% 论文的中文摘要
\begin{abstract}
复杂网络的研究可上溯到20世纪60年代对ER网络的研究。90年后代随着Internet
的发展,以及对人类社会、通信网络、生物网络、社交网络等各领域研究的深入,
发现了小世界网络和无尺度现象等普适现象与方法。对复杂网络的定性定量的科
学理解和分析,已成为如今网络时代科学研究的一个重点课题。

在此背景下,由于云计算时代的到来,本文针对面向云计算的数据中心网络基础
设施设计中的若干问题,进行了几方面的研究。………………
% 中文关键词。关键词之间用中文全角分号隔开,末尾无标点符号。
\keywords{小世界理论;网络模型;数据中心}
\end{abstract}

%%%%%%%%%%%%%%%%%%%%%%%%%%%%%%%%%%%%%%%%%%%%%%%%%%%%%%%%%%%%%%%%%%%%%%%%%%%%%%%
% 论文的英文摘要
\begin{englishabstract}
\blindtext
% 英文关键词。关键词之间用英文半角逗号隔开,末尾无符号。
\englishkeywords{Small World, Network Model, Data Center}
\end{englishabstract}

%%%%%%%%%%%%%%%%%%%%%%%%%%%%%%%%%%%%%%%%%%%%%%%%%%%%%%%%%%%%%%%%%%%%%%%%%%%%%%%
% 生成论文目次
\tableofcontents

%%%%%%%%%%%%%%%%%%%%%%%%%%%%%%%%%%%%%%%%%%%%%%%%%%%%%%%%%%%%%%%%%%%%%%%%%%%%%%%
% 开始正文部分
\mainmatter

%%%%%%%%%%%%%%%%%%%%%%%%%%%%%%%%%%%%%%%%%%%%%%%%%%%%%%%%%%%%%%%%%%%%%%%%%%%%%%%
% 学位论文的正文应以《绪论》作为第一章
\chapter{绪论}\label{chapter_introduction}
\section{研究背景}

在分布式网络领域,沿着高性能集群、普世计算、网格计算的方向,现已走入云
计算时代。

云计算对信息技术架构造成了越来越大的影响。例如,借助Amazon EC2云平台,
用户借助其基础设施,可以十分方便的部署各类应用,以支持企业服务需求。用
户可以按需购买计算资源,网络带宽,存储空间等各类资源以支持他们的业务需
求,并在业务完成之后迅速的归还这些资源。通过云技术,用户可以集中在他们
擅长的核心业务之中,而不会被诸如硬件购买、安装系统、网络设置、备份和安
全等等问题干扰。

与此同时,随着计算机的普及化和微型化,现在的手持设备拥有不输于7年前台式
机的处理能力。在网络时代面前,智能终端广泛普及,每个人都可成为信息源。
在信息爆炸的时代,数据挖掘、机器学习、金融分析和模拟等行业中不断涌现新
的需求,诸如针对用户行为和社会关系的挖掘进行广告精准投放,用户行为预测
等。为了支撑PB级尺度的数据规模,需要海量的计算节点,催生并不断促进了各
行各业对云计算基础设施的建设需求。

海量的数据需要海量的处理能力,然而海量的处理能力又需要高带宽的网络IO为
承载。作为云环境中最基础的一环,IaaS层在网络、存储、计算资源的分割这几
方面,承担起整个系统的基石。虽然并非必须,但一般来说,为考虑沙盘环境,
以及对资源的细粒度切割分配,IaaS通常会伴随着虚拟化技术的运用。虚拟化具
有许多与云计算切合的特点,例如,虚拟化可以屏蔽物理环境的差异,可在多物
理节点中进行无缝迁移,可对系统进行快照和还原。这些特点都与云计算时代所
追求的灵活性、高伸缩性、快速响应等特点而吻合。

在虚拟化实现方面,目前已取得了诸如ESXi,Xen,KVM等成熟成果。然而当虚拟
化扩大的一定规模,随着节点数目的增多,在网络方面将会面临一系列取舍的问
题。例如,基于二层交换的扁平网络,当节点数目上升到千数量级时,广播报文
将会极大的拖累网络性能,必须通过划分子网,通过三层路由等形式重新规划为
多层网络结构;另一方面,除了联通之外,还需要考虑ACL控制,负载均衡,外网
通讯等各类防火墙以及NAT规则的实现。这些复杂的网络配置,在一定程度上抵消
了虚拟机带来的灵活性。例如虽然虚拟机可根据需要动态迁移,但在迁移之后,
由于网络位置的变化需要重新进行网络参数配置。虽然虚拟机在迁移过程中系统
内部状态没有变化,但站在网络角度看,该虚拟节点跟关机重启没有区别。

针对上述问题,本文站在面向云计算时代的数据中心网络建设的角度,对网络模
型进行深入研究和探讨。通过改善二层交换网络的ARP机制来解决广播风暴问题,
引入比树形网络更为复杂的复杂网络理论,指导网络节点的互联模型。从而将网
络的复杂性隐藏在节点环境之外,在节点层面仅提供简单但巨大的二层交换扁平
网络。

\section{研究目的与意义}
\subsection{现有问题与不足}

测试一下引用\cite{newman2006structure},连续引用
\cite{newman2001random,aiello2000random,bollobas2001random},另一个连续引用
\cite{newman2001random,bollobas2001random,barabasi1999emergence}。测试一下带页码
的引用\cite[124--128]{erdHos1961strength}。

\subsection{中心观点与思想}

云计算在概念上通常被分为IaaS、PaaS、SaaS几个层面。但透过分类去理解其本
质,可认为是上世纪70年代基于大型计算机的中心控制型瘦客户端终端模式,在
如今技术水平上的一种新的表达,是在技术发展道路中,螺旋上升的结果。

与瘦客户端相比,云计算在设计结构上存在一定的相似性。

在设计思路上,两者都为了降低管理成本和硬件成本、以低能耗、高弹性等需求
为设计目标。随着技术的进步,云计算在具体实现形态上与传统的大型机也有很
大的不同:

一方面,云中心不再是传统的一台大型机,而是用大量廉价计算节点的互联来提
供海量资源。云计算更强调资源规模的无缝、平滑扩展,以及高可靠性,无单点
故障问题。另一方面,云计算时代的终端,也具备相当计算能力。随着web2.0的
整合,还有向胖客户端和智能终端发展的趋势。

总而言之,云计算在大框架中是传统的中心控制/终端的模式,但在中心与终端
两方面,都引入分布式技术加以改良。核心的思路是在低成本的前提下做到高可
靠性、高灵活性和高伸缩性。因此,云计算并不仅仅以数量换性能的表象,本质
上为低成本高性能,追求高能效比,并在实现层面讲究可实现性和可操作性。

%%%%%%%%%%%%%%%%%%%%%%%%%%%%%%%%%%%%%%%%%%%%%%%%%%%%%%%%%%%%%%%%%%%%%%%%%%%%%%%
% 学位论文的正文应以《结论》作为最后一章
\chapter{结论}\label{chapter_concludes}

本文在第\ref{chapter_smallworld}章中,通过考虑数据中心网络布局构建中的最大度限制
问题,提出了符合数据中心网络基本要求的DS小世界模型,并分析了它的性质。随后提出
SIDN,将DS模型映射到具体的网络结构中,并分析了所构成网络的平均直径、网络总带宽、
对故障的容错能力等各项网络性能。

分析与仿真实验证明,SIDN网络具有很好的扩展能力,网络总带宽与网络规模成
近似线性增长的关系;具有很强的容错能力,链路损坏与节点损坏几乎无法破坏
网络的联通性,故障率对网络性能的影响与破坏节点/链路占总资源比率线性相关。

随后在第\ref{chapter_scalefree}章中,分析了无尺度网络在数据中心网络构建应用中的
理论方面问题。对Scafida \cite{gyarmati2010scafida}文中所述在最大度限制的情况下运
用BA算法构造的网络并不会损失无尺度性质的观点,进行了深入的分析,并指出了该论点的
局限性。

在给出了在引入节点最大度限制之后,利用分治和递归的思想,对无尺度网络
进行多层构建,对所构造的网络进行度-度相关性,以及聚类性分析。

\begin{table}
  \centering
  \begin{tabular}{cccp{38mm}}
    \toprule
    \textbf{文档域类型} & \textbf{Java类型} & \textbf{宽度(字节)} & \textbf{说明} \\
    \midrule
    BOOLEAN  & boolean &  1  & \\
    CHAR     & char    &  2  & UTF-16字符 \\
    BYTE     & byte    &  1  & 有符号8位整数 \\
    SHORT    & short   &  2  & 有符号16位整数 \\
    INT      & int     &  4  & 有符号32位整数 \\
    LONG     & long    &  8  & 有符号64位整数 \\
    STRING   & String  &  字符串长度  & 以UTF-8编码存储 \\
    DATE     & java.util.Date & 8 & 距离GMT时间1970年1月1日0点0分0秒的毫秒数 \\
    BYTE\_ARRAY & byte$[]$ & 数组长度 & 用于存储二进制值 \\
    BIG\_INTEGER & java.math.BigInteger & 和具体值有关 & 任意精度的长整数 \\
    BIG\_DECIMAL & java.math.BigDecimal & 和具体值有关 & 任意精度的十进制实数 \\
    \bottomrule
  \end{tabular}
  \caption{测试表格}\label{table:test5}
\end{table}

表\ref{table:test5}用于测试表格。随后分析了无尺度网络构造过程中,交换机节点与数
据节点的角色区别,分析了两者在不同比率下形成的网络形态,以及对网络性能造成的影响。

通过理论分析和仿真实验,分析并找出比率因子q的最佳取值。此外,无尺度现象
的引入提高了网络的聚类系数,从而在不失灵活性可靠性的基础上,进一步提升
了网络的性能。

在第\ref{chapter_random}章中,将关注点转移到交换机本身。由于图论难以描述数据中心
网络中的交换设备,因此放弃基于图的抽象模型,转而基于多维簇划分的思想,提出并设计
了WarpNet网络模型。

该网络模型突破了基于图描述的局限性,并对网络的带宽等指标进行理论分析并
给出定量描述。最后对比了理论分析、仿真测试结果,并在实际物理环境中进系
真实部署,通过6节点的小规模实验以及1000节点虚拟机的大规模实验,表明该模
型的理论分析、仿真测试与实际实验吻合,并在网络性能、容错能力、伸缩性灵
活性方面得到了进一步的提升。

在第\ref{chapter_experiments}章中,针对网络模型研究这一类工作的共性,设计构造通
用验证平台系统。以海量虚拟机和虚拟分布式交换机的形式,实现了基于少量物理节点,对
大规模节点的模拟。其模拟运行的过程与真实运行在实现层面完全一致,运行的结果与真实
环境线性相关。除为本文所涉若干网络模型提供验证外,可进一步推广到更为广泛的领域,
为各种网络模型及路由算法的研究工作,提供分析、指导与验证。

%%%%%%%%%%%%%%%%%%%%%%%%%%%%%%%%%%%%%%%%%%%%%%%%%%%%%%%%%%%%%%%%%%%%%%%%%%%%%%%
% 致谢,应放在《结论》之后
\begin{acknowledgement}
  首先感谢我的母亲韦春花对我的支持。其次感谢我的导师陈近南对我的精心指导和热心帮助。接下来,
  感谢我的师兄茅十八和风际中,他们阅读了我的论文草稿并提出了很有价值的修改建议。

  最后,感谢我亲爱的老婆们:双儿、苏荃、阿珂、沐剑屏、曾柔、建宁公主、方怡,感谢
  你们在生活上对我无微不至的关怀和照顾。我爱你们!
\end{acknowledgement}

% 参考文献。应放在\backmatter之前。
% 推荐使用BibTeX,若不使用BibTeX时注释掉下面一句。
\nocite{*}
\bibliography{sample}
% 不使用 BibTeX
%\begin{thebibliography}{2}
%
%\bibitem{deng:01a}
%{邓建松,彭冉冉,陈长松}.
%\newblock {\em \LaTeXe{}科技排版指南}.
%\newblock 科学出版社,书号:7-03-009239-2/TP.1516, 北京, 2001.
%
%\bibitem{wang:00a}
%王磊.
%\newblock {\em \LaTeXe{}插图指南}.
%\newblock 2000.
%\end{thebibliography}

%%%%%%%%%%%%%%%%%%%%%%%%%%%%%%%%%%%%%%%%%%%%%%%%%%%%%%%%%%%%%%%%%%%%%%%%%%%%%%%
% 书籍附件
\backmatter
%%%%%%%%%%%%%%%%%%%%%%%%%%%%%%%%%%%%%%%%%%%%%%%%%%%%%%%%%%%%%%%%%%%%%%%%%%%%%%%
% 作者简历与科研成果页,应放在backmatter之后
\begin{resume}
% 论文作者在攻读学位期间所发表的文章的列表,按发表日期从近到远排列。
\begin{publications}
\item Xiaobao Wei, Jinnan Chen, ``Voting-on-Grid Clustering for Secure
  Localization in Wireless Sensor Networks,'' in \textsl{Proc. IEEE International
    Conference on Communications (ICC) 2010}, May. 2010.
\item Xiaobao Wei, Shiba Mao, Jinnan Chen, ``Protecting Source Location Privacy
  in Wireless Sensor Networks with Data Aggregation,'' in \textsl{Proc. 6th
    International Conference on Ubiquitous Intelligence and Computing (UIC)
    2009}, Oct. 2009.
\end{publications}
\end{resume}

%%%%%%%%%%%%%%%%%%%%%%%%%%%%%%%%%%%%%%%%%%%%%%%%%%%%%%%%%%%%%%%%%%%%%%%%%%%%%%%
\end{document}
